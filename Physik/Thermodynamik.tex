\begin{multicols}{2}
\section{Wärmedehnung}
\index{Thermodynamik!Wärmedehnung}
\begin{align*}
\rho(T)&=\rho_0(1-\beta(T-T_0))\\
V(T)&=V_0(1+\gamma(T-T_0))\\
l(T)&=l_0(1+\alpha(T-T_0))\\
\gamma&\approx 3\cdot \alpha\\
\gamma&\approx \beta
\end{align*}

\section{Wärme}
\index{Thermodynamik!Wärme}
\begin{align*}
\Delta Q&=c\cdot m(T-T_0)\\
\Delta Q&=C(T-T_0)\\
\Delta Q&=\int_{T_0}^T c\cdot m \diff T\\
\Delta Q&=c_{mol}\cdot n(T-T_0)
\end{align*}

\section{Mischtemperatur}
\index{Thermodynamik!Mischtemperatur}
\begin{align*}
T_m&=\frac{\sum_{i=1}^n T_i m_i c_i}{\sum_{i=1}^n m_i c_i} \\
\dot{Q}&\text{Ist durch einen mehrschichtiges}\\
&\text{stationäres System Konstant}
\end{align*}

\section{Wärmeleitung}
\index{Thermodynamik!Wärme!-leitung}
\begin{align*}
\dot{Q}&=\frac{\diff Q}{\diff t}=\varPhi=P\\
\vec{\dot{q}}&=\frac{\dot{Q}}{A}\cdot\vec{e_A}\\
\vec{\dot{q}}&=-\lambda\grad{T}\\
\vec{\dot{q}}&=\frac{\lambda}{s}\left(T_A-T_B\right)\cdot\vec{e_s}\\
\dot{q}&=\frac{1}{\sum_{i=1}^n\frac{s_i}{\lambda_i}}\cdot\left(T_A-T_B\right)
\end{align*}

\section{Wärmekonvektion}
\index{Thermodynamik!Wärme!-konvektion}
\begin{align*}
\dot{q}&=\alpha\left(T_A-T_B\right)\\
\dot{q}&=\frac{1}{\sum_{i=1}^n\frac{1}{\alpha_i}}\cdot\left(T_A-T_B\right)
\end{align*}
\vfill
\end{multicols}

\section{Wärmewiderstand}
\index{Thermodynamik!Wärme!-widerstand}
\[R_{th}=\frac{T_A-T_B}{\dot{q}\cdot A}=\frac{s}{\lambda A}=\frac{1}{k}=\frac{1}{\alpha
A}=\sum_{i=1}^n R_{i}\]

\subsection{Wärmeübertragung}
\index{Thermodynamik!Wärme!-übertragung}
\begin{align*}
k&=\frac{1}{\sum_{i=1}^n\frac{s_i}{\lambda_i}+\sum_{i=1}^n\frac{1}{\alpha_i}+\sum_{i=1}^n R_{i}}\\
\dot{q}&=\frac{1}{\sum_{i=1}^n\frac{s_i}{\lambda_i}+\sum_{i=1}^n\frac{1}{\alpha_i}+\sum_{i=1}^n R_{i}}\cdot\left(T_A-T_B\right)\\
\dot{q}&=k\cdot\left(T_A-T_B\right)
\end{align*}

\subsection{Wärmestrahlung}
\index{Thermodynamik!Wärme!-strahlung}
\begin{align*}
\sigma &= 5,6704\cdot 10^{-8} \frac{W}{m^2 K^4} \quad Boltzmann Konstante\\
\sigma_A &= c_A\\
\alpha&=\varepsilon\\
1&=\alpha+\tau+\vartheta\\
\dot{Q}&=\varepsilon A \sigma T^4\\
\dot{Q}_{AB}&=C_{AB}A_A\left(T_A^4-T_B^4\right)\\
C_{AB}&=\varepsilon_{AB}\sigma=\frac{\sigma}{\frac{1}{\varepsilon_A}+\frac{1}{\varepsilon_B}-1}=\frac{1}{\frac{1}{\sigma_A}+\frac{1}{\sigma_B}-\frac{1}{\sigma}}&&\text{Parallel}\\
C_{AB}&=\frac{\sigma}{\frac{1}{\varepsilon_A}+\frac{A_A}{A_B}\left(\frac{1}{\varepsilon_B}-1\right)}&&\text{$A_A$ von $A_B$ umschlossen}\\
C_{AB}&\approx\varepsilon_A\sigma&&\text{parallel ($A_A\ll A_B$)}
\end{align*}

\subsection{Zustandsänderung des idealen Gases}
Teilchen stehen nicht in Wechselwirkung, besitzen kein Volumen \\ und es kommt zu keinem Phasenübergang

\newpage
\begin{multicols}{2}
\subsubsection*{Energie}
\index{Ideales Gas!Energie}
\begin{align*}
U_{12}&=Q_{12}+W_{12}\\
&\text{Nur Isobar:}\\
\diff H&=c_pm\diff T=U+p\diff V\\
\diff S&=\frac{\diff Q}{T}
\end{align*}
\vfill

\subsubsection*{Zustandsgleichung}
\index{Ideales Gas!Zustandsgleichung}
\begin{align*}
\frac{pV}{T}&=\text{const}\\
pV&=NkT=mR_sT=nRT\\
R_s&=\frac{nR}{m}\\
R_s&=c_p-c_v
\end{align*}
\end{multicols}

\begin{multicols}{2}
\subsubsection*{Isotherm}
\index{Ideales Gas!Isotherm}
\begin{align*}
pV&=\text{const}\\
T&=\text{const}\\
U_{12}&=0\\
U_{12}&=Q_{12}+ W_{12}\\
Q_{12}&=-W_{12}\\
W_{12}&=p_1V_1\ln{\frac{V_2}{V_1}}\\
W_{12}&=p_1V_1\ln{\frac{p_1}{p_2}}\\
S_{12}&=mc_p\ln{\frac{V_2}{V_1}}+mc_V\ln{\frac{p_2}{p_1}}
\end{align*}

\subsubsection*{Isobar}
\index{Ideales Gas!Isobar}
\begin{align*}
\frac{V}{T}&=\text{const}\\
p&=\text{const}\\
Q_{12}&=mc_p\left(T_2-T_1\right)\\
W_{12}&=-p\left(V_2-V_1\right)\\
U_{12}&=Q_{12}+ W_{12}\\
S_{12}&=mc_p\ln{\frac{V_2}{V_1}}
\end{align*}

\subsubsection*{Isochor}
\index{Ideales Gas!Isochor}
\begin{align*}
\frac{p}{T}&=\text{const}\\
V&=\text{const}\\
Q_{12}&=mc_v\left(T_2-T_1\right)\\
W_{12}&=0\\
U_{12}&=Q_{12}\\
S_{12}&=mc_v\ln{\frac{p_2}{p_1}}
\end{align*}

\subsubsection*{Adiabat}
\index{Ideales Gas!Adiabat}
\begin{align*}
pV^\kappa&=\text{const}\\
Q&=\text{const}\\
\kappa&=\frac{c_p}{c_V}\\
\frac{T_2}{T_1}&=\left(\frac{V_2}{V_1}\right)^{1-\kappa}=\left(\frac{p_2}{p_1}\right)^{\frac{\kappa-1}{\kappa}}\\
Q_{12}&=0\\
W_{12}&=mc_v\left(T_2-T_1\right)\\
W_{12}&=\frac{RT_1}{\kappa-1}\left(\left(\frac{V_2}{V_1}\right)^{1-\kappa}-1\right)\\
U_{12}&=W_{12}\\
S_{12}&=0;
\end{align*}
\end{multicols}

\begin{multicols}{2}
\subsubsection*{Kreisprozeß}
\index{Kreisprozeß}
\begin{align*}
\oint \diff U&=0\\
\oint \diff U&= \oint \diff Q +\oint \diff W\\
&\text{Revesiebel:}
\oint \diff S&=0\\
&\text{Irrevesiebel}
\oint \diff S&>0\\
\end{align*}

\subsubsection*{Carnot-Prozeß}
\index{Carnot'prozeß}
\begin{align*}
\eta_C&=\frac{W_{ab}}{Q_{zu}}\\
\eta_C&=\frac{Q_{zu}-Q_{AB}}{Q_{zu}}\\
\eta_C&=\frac{T_h-T_n}{T_n}
\end{align*}
\vfill
\end{multicols}