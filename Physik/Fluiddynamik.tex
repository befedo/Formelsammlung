\section{Fluidmechanik}

\subsection{Ohne Reibung}

\begin{multicols}{2}{}
\subsubsection{Statischer Druck}
\begin{align*}
p&=\frac{\diff F_N}{\diff A}
\end{align*}


\subsubsection{Dynamischer Druck}
\begin{align*}
p&=\frac{1}{2}\rho v^2
\end{align*}


\subsubsection{Schweredruck}
\begin{align*}
p&=\frac{\rho V g}{A}\\
&=h\rho g
\end{align*}


\subsubsection{Volumenstrom}
\begin{align*}
\dot{V}&=v A\\
&=\iint_A \vec{v} \diff\vec{ A}\\
&=\frac{\diff V}{\diff t}\\
&=Q
\end{align*}


\subsubsection{Massenstrom}
\begin{align*}
\dot{m}&=jA\\
&=\iint_A \vec{j} \diff\vec{A}\\
&=\frac{\diff m}{\diff t}
\end{align*}

\subsubsection{Auftrieb}
\begin{align*}
\vec{F_A}&=-\rho_V \vec{g} V\\
&=-\frac{\rho_V}{\rho_M}\vec{F_G}
\end{align*}

\subsubsection{Kontinuitätsgleichung}
\begin{align*}
\left.\dot{m}\right|_1&=\left.\dot{m}\right|_2\\
\left.\dot{V}\right|_1&=\left.\dot{V}\right|_2&&\rho_1=\rho_2\\
v_1A_1&=v_2A_2&&\rho_1=\rho_2
\end{align*}
\hfill
\end{multicols}

\begin{multicols}{2}{}
\subsubsection{Kompressibilität}
\begin{align*}
\kappa&=\frac{\Delta V}{\Delta p V}
\end{align*}


\subsubsection{Volumenausdehnungskoeffezient}
\begin{align*}
\frac{\Delta V}{V}&= \gamma \Delta T
\end{align*}


\subsubsection{Barometrische Höhenformel}
\begin{align*}
p&=p_0 e^{-Ch}\\
C&=\frac{\rho_0 g}{p_0}
\end{align*}


\subsubsection{Bernoulli Gleichung}
\begin{align*}
p+\frac{1}{2}\rho v^2+ \rho g h= \text{const}
\end{align*}
\end{multicols}


\subsection{Laminare Reibung}

\begin{multicols}{2}{}
\subsubsection{Newtonsches Reibungsgesetz}
\begin{align*}
F_R&=\eta A \frac{\diff v}{\diff x}
\end{align*}


\subsubsection{Laminare Strömung (Rohr)}
\begin{align*}
v(r)&=\frac{p}{4\eta l}\left(R^2-r^2\right)\\
p&=\frac{4\eta l}{R^2}v(0)\\
\dot{V}&=\frac{\pi R^4}{8\eta l}p
\end{align*}


\subsubsection{Umströmung (Kugel)}
\begin{align*}
F_R=6\pi\eta r v
\end{align*}



\subsubsection{Bernoulligleichung mit Reibung}
\begin{align*}
&p_1+\frac{1}{2}\rho v_1^2+ \rho g h_1 \\
=&p_2+\frac{1}{2}\rho v_2^2+ \rho g h_2+\Delta p
\end{align*}


\subsubsection{Reynoldszahl}
\begin{align*}
Re&=\frac{L\rho v}{\eta}\\
Re&>Re_{krit}\\
&\text{Strömung wird Turbulent}
\end{align*}
\end{multicols}
