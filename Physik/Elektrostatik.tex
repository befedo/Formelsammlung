\begin{multicols}{2}{}
\subsubsection{Ladung}
\begin{align*}
Q&=n\cdot e_0\\
&=CU\\
&=\int i \diff t
\end{align*}

\subsubsection{Punktladungen}
\begin{align*}
\vec{E}(\vec{r})&=\sum_{i=1}^{N}\vec{E}_i{\vec{r}_i}
\end{align*}
\vspace{15mm}
\end{multicols}

\begin{multicols}{2}{}
\subsubsection{COULOMB Gesetz}
\begin{align*}
\vec{F}_{12}&=\frac{1}{4\pi\epsilon}\frac{Q_1Q_2}{r^2}\vec{r_12}\\
&=\vec{E}Q\\
\vec{E}&=\frac{1}{4\pi\epsilon}\frac{Q}{r^2}\vec{r}\\
&=-\grad\varphi\\
&=-\left(\frac{\partial \varphi}{\partial x}\vec{e}_x+\frac{\partial \varphi}{\partial y}\vec{e}_y+\frac{\partial \varphi}{\partial z}\vec{e}_z\right)
\end{align*}
\vspace{15mm}

\subsubsection{Spannung}
\begin{align*}
U_{AB}=&\frac{W_{AB}}{Q}\\
=&\int_A^B\vec{E}\circ\diff\vec{s}\\
=&\oint_s\vec{E}\circ\diff\vec{s}=0\\
=&\varphi_A-\varphi_B\\
=&-\int_\infty^A\vec{E}\circ\diff\vec{s}\\
&-\left(-\int_\infty^B\vec{E}\circ\diff\vec{s}\right)
\end{align*}
\end{multicols}

\newpage
\begin{multicols}{2}{}
\subsubsection{El- / Verschiebungsfluß}
\begin{align*}
\psi&=\int_A\vec{E}\circ\diff\vec{A}\\
\psi&=\oint_A\vec{E}\circ\diff\vec{A}=\frac{Q}{\epsilon}\\
\end{align*}

\subsubsection{Flußdichte}
\begin{align*}
\vec{D}&=\frac{\diff Q}{\diff A}\vec{e}_A\\
\vec{D}&=\epsilon\vec{E}\\
Q&=\oint_AD\diff A
\end{align*}
\end{multicols}

\subsubsection{Kapazität}
\begin{align*}
Q&=CU
\end{align*}

\begin{multicols}{2}{}
\subsubsection{OHMsches Gesetz}
\begin{align*}
I &=\oint_A\vec{j}\circ\diff\vec{A}\\
  &=\oint_A \kappa\vec{E}\circ\diff\vec{A}\\
  &=\underbrace{\kappa E\cdot 4\pi r^2}_{\text{Kugel}}
\end{align*}
\vspace{20mm}

\subsubsection{Arbeit im elektrischem Feld}
\begin{align*}
w&=\frac{1}{2}\vec{E}\circ\vec{D}\\
W&=\int_Vw\diff V\\
 &=-Q\int_A^B\vec{E}\circ\diff\vec{s}\\
 &=\int_U Q\diff U\\
 &= \int_U CU \diff U\\
 &=\frac{1}{2}CU^2
\end{align*}
\end{multicols}
