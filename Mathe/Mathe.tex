\begin{quote}
Why waste time learning\\when ignorance is instantaneous?\\- Hobbes
\end{quote}

	%1. Abschnitt
	\section{Rechenregeln fuer Potenzen}
	\index{Potenzen}							
			\begin{alignat*}{3}
				a^m \cdot a^n &= a^{m+n} &\quad\quad& \frac{a^m}{a^n} = a^{m-n} &\quad\quad& \left( a^m \right)^n = \left( a^n \right)^m = a^{m \cdot n} \\ 
				a^n \cdot b^n &= \left( a \cdot b \right)^n && \frac{a^n}{b^n} = \left( \frac{a}{b} \right)^n && \text{(fuer a} > \text{0) } a^b = e^{b \cdot \ln a}
			\end{alignat*}	
	%2.Abschnitt
	\vspace{10mm}
	\section{Zusammenhang zwischen Wurzeln und Potenzen}					
				\boxed{
				\text{Im Folgenden wird vorausgesetzt, dass alle Potenzen und Wurzeln existieren.}
				} 			
			\begin{align*}
				\sqrt[n]{a} &= a^{\frac{1}{n}} & \sqrt[n]{a^m} &= a^{\frac{m}{n}} & \left(\sqrt[n]{a}\right)^m &= a^{ \frac{m}{n}}
			\end{align*}	
	%3.Abschnitt
	\newpage
	\section{Potenzen und Logarithmen}
	\index{Potenzen}	
			\text{Schreibweise: }
			x\(=\log_a \left(b\right) \text{ mit } a > 0, a \neq 1 \text{ und } b > 0 \text{.}\)
			\newline
			\text{Es gillt: }
			\(\log_a \left(1\right) = 0, \text{ } \log_a \left( a \right) = 1\)
			\text{.}				
	%Unterpunkt
	\vspace{10mm}
	\subsection*{Der natuerliche Logarithmus}
	\index{Logarithmus}	
			%Hier muss der Limes noch angepasst werden, so dass der BEreich unter dem Ausdruck steht!
			\begin{flushleft}
			\text{Der Logarithmus zur Basis } \(e\) \text{ mit } \(e = \lim\limits_{n\to\infty} {\left(1+\frac{1}{n}\right)^n} = 2,71828...\)
			\begin{align*}
				\log_e \left(b\right) &= \ln \left(b\right) & \ln \left( \frac{1}{e} \right) = -1 ; \text{ da } e^{-1} = \frac{1}{e}
			\end{align*}
			
			\boxed{\text{Man beachte: } \text{x}^a = e^{\ln \left(\text{x}\right) \cdot a}}
			\end{flushleft}	
	%Unterpunkt
	\vspace{10mm}
	\subsection*{Rechnen mit Logarithmen}
	\index{Logarithmus}			
			%Tabelle anlegen
			\begin{table}[h]
						
			\begin{tabular}{|l|l|}
				
				\hline
					\text{Es gillt:}
				&	%Dient als Ueberschrifft
					\text{Weitere Beziehungen:}
				\\
				\hline
					%Beginnt in der ersten Spalte, erste Zeile
   				\(\log_a \left({u \cdot v}\right) = \log_a \left(u\right) + \log_a \left(v\right)\)
				&	%In die naechste Spalte springen
					\( \log_a \left( \sqrt[n]{u} \right) = \frac{1}{n} \log_a \left( u \right)\)
				\\%Zurueck in die erste Spalte, zweite Zeile
					\(\log_a \left( \frac{u}{v} \right) = \log_a \left(u\right) - \log_a \left(v\right)\)
				&	%%In die naechste Spalte springen						
 					\(a^{\log_a \left(u\right)} = \log_a \left(a^u\right) = u\)
				\\%Zurueck in die erste Spalte, dritte Zeile
					\(\log_a \left(u^p\right) = p \cdot \log_a \left(u\right)\)
				&	%%In die naechste Spalte springen
					\(\log_a\left(u\right) = \frac{\log_c \left( u \right)}{\log_c \left( a \right)}\)
				\\
				%Unterstreicht die Tabelle
					\hline
								
			\end{tabular}
			\end{table}
			
	%4.Abschnitt
	\vspace{10mm}
	\section{Der Binomische Lehrsatz}
	\index{Binomischer Lehrsatz}			
		Die Potenzen eines Binoms a+b lassen sich nach dem Binomischen Lehrsatz 
		\newline 
		wie folgt entwickeln \( \left(n \in \N^* \right)\):
		\vspace{5mm}
		\newline
		\( \left( a + b \right)^n = a^n + \binom{n}{1} a^{n-1} \cdot b^1 + \binom{n}{2} a^{n-2} \cdot b^2 + \binom{n}{3} a^{n-3} \cdot b^3 + 
		\ldots + \binom{n}{n-1} a^{1} \cdot b^{n-1} + b^n \)
		\vspace{5mm}
		\newline
		\text{Die Koeffizienten \( \binom{n}{k} \) heißen Binominalkoeffizienten, ihr Bildungsgesetz lautet:}
		\vspace{5mm}
		\newline
		\( \binom{n}{k} = \frac{n \left( n - 1 \right) \left( n - 2 \right) \ldots \left[ n - \left( k - 1 \right) \right]}{k!} = \frac{n!}{k! \left( n - k \right) !} \)	
	\vspace{10mm}
	\subsection*{Einige Eigenschaften der Binominalkoeffizienten}		
			\begin{align*}
				\binom{n}{0} &= \binom{n}{n} = 1 & \binom{n}{k} &= 0 \text{ fuer k} > \text{n} & \binom{n}{1} &= \binom{n}{n-1} = n
				\\
				\binom{n}{k} &= \binom{n}{n-k} & \binom{n}{k} &+ \binom{n}{k+1} = \binom{n+1}{k+1}
			\end{align*}			
	\vspace{10mm}
	\section{Sinus, Kosinus, Tangens und Kotangens}
	
	%Unterpunkt
	\vspace{10mm}
	\subsection{Beziehungen zwischen Sinus, Kosinus, Tangens und Kotangens}
	\index{Sinus}
	\index{Kosinus}
	\index{Tangens}
	\index{Kotangens}	
		\begin{align*}
			&\sin^2 \left( \alpha \right) + \cos^2 \left( \alpha \right) = 1 & &\tan \left( \alpha \right) \cdot \cot \left( \alpha \right) = 1	
			\\
			&\tan \left( \alpha \right) = \frac{\sin \left( \alpha \right)}{\cos \left( \alpha \right)} & &\cot \left( \alpha \right) = \frac{\cos \left( \alpha \right)}{\sin \left( \alpha \right)} 
			\\
			&1 + \tan^2 \left( \alpha \right) = \frac{1}{\cos^2 \left( \alpha \right)} & &1 + \cot^2 \left( \alpha \right) = \frac{1}{\sin^2 \left( \alpha \right)}
		\end{align*}	
	%Unterpunkt	
	\vspace{10mm}
	\subsection{Additionstheoreme}
	\index{Additionstheoreme}
		\begin{align*}
			\sin \left( \alpha \pm \beta \right) &= \sin \left( \alpha \right) \cos \left( \beta \right) \pm \cos \left( \alpha \right) \sin \left( \beta \right)
			\\ 
			\cos \left( \alpha \pm \beta \right) &= \cos \left( \alpha \right) \cos \left( \beta \right) \mp \sin \left( \alpha \right) \sin \left( \beta \right)
			\\
			\tan \left( \alpha \pm \beta \right) &= \frac{\tan \left( \alpha \right) \pm \tan \left( \beta \right)}{1 \mp \tan \left( \alpha \right) \tan \left( \beta \right)}
		\end{align*}
		
	%Unterpunkt
	\vspace{10mm}
	\subsection*{Funktionen des doppelten und halben Winkels}			
			\begin{align*}
				\sin \left( 2 \alpha \right) &= 2 \sin \left( \alpha \right) \cos \left( \alpha \right) 
				\\
				\cos \left( 2 \alpha \right) &= \cos^2 \left( \alpha \right) - \sin^2 \left( \alpha \right) = 2 \cos^2 \left( \alpha \right) -1 = 1 - 2 \sin^2 \left( \alpha \right)
				\\
				\tan \left( 2 \alpha \right) &= \frac{2 \tan \left( \alpha \right)}{1 - \tan^2 \left( \alpha \right)}
				\\
				\sin^2 \left( \frac{\alpha}{2} \right) &= \frac{1}{2} \left( 1 - \cos \left( \alpha \right) \right)
				\\
				\cos^2 \left( \frac{\alpha}{2} \right) &= \frac{1}{2} \left( 1 + \cos \left( \alpha \right) \right)
				\\
				\tan^2 \left( \frac{\alpha}{2} \right) &= \frac{1 - \cos \left( \alpha \right)}{1 + \cos \left( \alpha \right)}
			\end{align*}
			
	%Unterpunkt
	\vspace{10mm}
	\subsection*{Umformungen}
	
	%Unterunterpunkt :)
	\vspace{10mm}
	\subsubsection*{Summe oder Differenz in ein Produkt}		
			\begin{flushleft}
				\(\sin \left( \alpha \right) + \sin \left( \beta \right) = 2 \sin \left( \frac{\alpha + \beta}{2}\right) \cos \left( \frac{\alpha - \beta}{2} \right)\)
				\\
				\(\sin \left( \alpha \right) - \sin \left( \beta \right) = 2 \cos \left( \frac{\alpha + \beta}{2}\right) \sin \left( \frac{\alpha - \beta}{2} \right)\)
				\\
				\(\cos \left( \alpha \right) + \cos \left( \beta \right) = 2 \cos \left( \frac{\alpha + \beta}{2}\right) \cos \left( \frac{\alpha - \beta}{2} \right)\)
				\\
				\(\cos \left( \alpha \right) - \cos \left( \beta \right) = -2 \sin \left( \frac{\alpha + \beta}{2}\right) \sin \left( \frac{\alpha - \beta}{2} \right)\)
			\end{flushleft}
	
	%Unterunterpunkt :)
	\vspace{10mm}
	\subsubsection*{Produkt in eine Summe oder Differenz}	
			\begin{flushleft}
				\(2 \sin \left( \alpha \right) \sin \left( \beta \right) = \cos \left( \alpha - \beta \right) - \cos \left( \alpha + \beta \right)\)
				\\
				\(2 \cos \left( \alpha \right) \cos \left( \beta \right) = \cos \left( \alpha - \beta \right) + \cos \left( \alpha + \beta \right)\)
				\\
				\(2 \sin \left( \alpha \right) \cos \left( \beta \right) = \sin \left( \alpha - \beta \right) + \sin \left( \alpha + \beta \right)\)
			\end{flushleft}
	
	%5.Abschnitt		
	\vspace{10mm}	
	\section{Komplexe Zahlen}
	\index{Komplexe Zahlen}						
			\text{Für die Menge aller komplexen Zahlen schreibt man:}
			\vspace{5mm}
			\\
			\fbox{\( \C = \left\{ z | z = a + bj, a \in \R \wedge b \in \R \right\} \)}
			\vspace{5mm}
			\\
			\text{a-Realteil \ \  b-Imaginaerteil \ \  j-imaginaere Einheit}
						
			%Tabelle
			\begin{table}[h]
			
				\begin{tabular}{|l|l|l|}
				\hline
 				\text{kartesiche Form} & \text{trigonometrische Form}  & \text{exponentialform}	 \\ 
				\hline
 				\( z = a + bj \) & \( z = \left| z \right| \left( \cos \varphi + j \cdot \sin \varphi \right) \)  & \( z = \left| z \right| \cdot e^{j 	\varphi} \)  \\ 
				\hline
 				\( z^* = \left( a + bj \right)^* = a-bj \) & \( z^* = \left| z \right| \left( \cos \varphi - j \cdot \sin \varphi \right) \) & \( z^* = \left| z \right| \cdot e^{-j \varphi} \) \\ 
				\hline
				\end{tabular}
			
			\end{table}
			
			\begin{flushleft}
				\text{\( \left| z \right| \) = Betrag von z}
				\\
				\text{\( \varphi \) = Argument (Winkel) von z}
				\\
				\text{\( z^* \) = Konjugiert komplexe Zahl}
			\end{flushleft}
			
		%Unterpunkt
		\vspace{10mm}
		\subsection{Umrechnungen zwischen den Darstellungsformen}
			
			\vspace{10mm}
			\subsubsection*{Polarform \(\rightarrow\) Kartesiche Form}
			
			
				\( z = \left| z \right| \cdot e^{j \varphi} = \left| z \right| \left( \cos \varphi + j \cdot \sin \varphi \right) = 
				\underbrace{\left| z \right| \cdot \cos \varphi}_{a} + j \cdot \underbrace{\left| z \right| \cdot \sin \varphi}_{b} = a + bj \)	
			
			\vspace{10mm}
			\subsubsection*{Kartesische Form \(\rightarrow\) Polarform}
			
			
				\( \left| z \right| = \sqrt{a^2 + b^2}\), \ \ \(\tan \varphi = \frac{b}{a} \)
				
		%Unterpunkt
		\vspace{10mm}
		\subsection{Rechnen mit Komplexen Zahlen}
		\index{Komplexe Zahlen}
			\vspace{10mm}
			\subsubsection*{Multiplikation}
															
				\fbox{In kartesischer Form:}
								
				\begin{center}
				\(z_1 \cdot z_2 = \left( a_1 + j b_1 \right) \cdot \left( a_2 + j b_2 \right) 
												= \left( a_1 a_2 - b_1 b_2 \right) + j \cdot \left( a_1 b_2 + a_2 b_1 \right)\)
				\end{center}
								
				\fbox{In der Polarform:}
												
				\begin{align*}
					z_1 \cdot z_2 &= \left[ \left| z_1 \right| \left( \cos \varphi_1 + j \cdot \sin \varphi_1 \right) \right] \cdot 
													 \left[ \left| z_2 \right| \left( \cos \varphi_2 + j \cdot \sin \varphi_2 \right) \right]
					\\
					&= \left( \left| z_1 \right| \left| z_2 \right| \right) \cdot \left[ \cos \left( \varphi_1 + \varphi_2 \right) +  
					j \cdot \sin \left(	\varphi_1 + \varphi_2 \right) \right]
					\\
					&= \left( \left| z_1 \right| \cdot e^{j \varphi_1} \right) \cdot \left( \left| z_2 \right| \cdot e^{j \varphi_2} \right)
					= \left( \left| z_1 \right| \left| z_2 \right| \right) \cdot e^{j \left( \varphi_1 + \varphi_2 \right)}
				\end{align*}
			
			\vspace{10mm}	
			\subsubsection*{Division}
			
				\fbox{In kartesischer Form}
				
					\begin{center}
						
					\end{center}
				
				\fbox{In der Polarform}