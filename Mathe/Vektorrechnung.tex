\section{Vektorrechnung}
\index{Vektorrechnung}

\subsection{Grundlagen}

\begin{multicols}{2}
\subsubsection*{Darstellung}
\begin{align*}
\vec{a}	&=\vec{a}_x+\vec{a}_y+\vec{a}_z\\
	&=a_x\vec{e}_x+a_y\vec{e}_y+a_y\vec{e}_y\\
	&=\begin{pmatrix} a_x\\ a_y\\a_z\end{pmatrix}
\end{align*}
\vfill
\subsubsection*{2 Punkt Vektor}
\begin{align*} 
\vec{P_1P_2} &=\begin{pmatrix} x_2-x_1\\ y_2-y_1\\z_2-z_1\end{pmatrix}
\end{align*}
\vfill
\end{multicols}

\begin{multicols}{2}
\subsubsection*{Betrag}
\begin{align*} 
|\vec{a}|	&=a\\
		&=\sqrt{a_x^2+a_y^2+a_z^2}\\
		&=\sqrt{\vec{a}\circ\vec{a}}
\end{align*}
\vfill
\subsubsection*{Richtungswinkel}
\begin{align*} 
\cos \alpha &= \frac{a_x}{|\vec{a}|}\\
\cos \beta &= \frac{a_y}{|\vec{a}|}\\
\cos \gamma &= \frac{a_z}{|\vec{a}|}\\
1&=\cos^2\alpha+\cos^2\beta+\cos^2\gamma
\end{align*}
\vfill
\end{multicols}

\newpage
\subsection{Vektoroperationen}
\begin{multicols}{2}
\subsubsection*{Addition und Subtraktion}
\begin{align*} 
\vec{a}\pm\vec{b}&= \begin{pmatrix}a_x\pm b_x\\a_y\pm b_y\\a_z\pm b_z\end{pmatrix}
\end{align*}
\vfill
\subsubsection*{Multiplikation mit einem Skalar}
\begin{align*} 
a\cdot\vec{b}&= \begin{pmatrix}ab_x\\ ab_y\\ab_z\end{pmatrix}
\end{align*}
\vspace{5mm}
\vfill
\end{multicols}

\begin{multicols}{2}
\subsubsection*{Skalarprodukt}
\begin{align*} 
\vec{a}\circ\vec{b} &= \begin{pmatrix}a_x\\ a_y\\a_z\end{pmatrix}\circ\begin{pmatrix}b_x\\ b_y\\b_z\end{pmatrix}\\
&=a_xb_x+a_yb_y+a_zb_z\\
&=|\vec{a}|\cdot|\vec{b}|\cdot \cos\angle(\vec{a},\vec{b})
\end{align*}
\vfill
\subsubsection*{Einheitsvektor}
\begin{align*} 
\vec{e}_a&=\frac{\vec{a}}{|\vec{a}|}= \begin{pmatrix}a_x/|\vec{a}|\\ a_y/|\vec{a}|\\a_z/|\vec{a}|\end{pmatrix}
\end{align*}
\vfill
\end{multicols}

\begin{multicols}{2}
\subsubsection*{Kreuzprodukt}
\index{Kreuzprodukt}
\text{$|\vec{a}\times\vec{b}|$ Fläche des Parallelograms $\vec{a},\vec{b}$}\\
\text{$\vec{a}\times\vec{b} \perp \vec{a} \land \vec{a}\times\vec{b} \perp \vec{b}$}
\begin{align*} 
\vec{a}\times\vec{b} &= \begin{pmatrix}a_x\\ a_y\\a_z\end{pmatrix}\times\begin{pmatrix}b_x\\ b_y\\b_z\end{pmatrix}\\
&=\begin{pmatrix}a_yb_z-a_zb_y\\ a_zb_x-a_xb_z\\a_xb_y-a_yb_x\end{pmatrix}\\
&=\begin{vmatrix}\vec{e}_x&\vec{e}_y&\vec{e}_z\\a_x&a_y&a_z\\b_x&b_y&b_z\end{vmatrix}
\end{align*}
\vfill
\subsubsection*{Spatprodukt}
\index{Spatprodukt}
\text{$\vec{a}\circ(\vec{b}\times\vec{c})$}
\\
\text{Volumen des Parallelpiped $\vec{a},\vec{b},\vec{c}$}
\begin{align*} 
[\vec{a}\vec{b}\vec{c}]  &=\vec{a}\circ(\vec{b}\times\vec{c})\\
&=a_x(b_yc_z-b_zc_y)\\&+a_y(b_zc_x-b_xc_z)\\&+a_z(b_xc_y-b_yc_x)\\
&=\begin{vmatrix}a_x&a_y&a_z\\b_x&b_y&b_z\\c_x&c_y&c_z\end{vmatrix}
\end{align*}
\vfill
\end{multicols}

\begin{multicols}{2}
\subsubsection*{Schnittwinkel}
\begin{align*} 
\cos\angle(\vec{a},\vec{b})&=\frac{\vec{a}\circ\vec{b}}{|\vec{a}|\cdot|\vec{b}|}
\end{align*}
\vfill
\subsubsection*{Projektion}
\begin{align*} 
\vec{a}_b&=\left(\frac{\vec{a}\circ\vec{b}}{|\vec{a}|^2}\right)\vec{a}=(\vec{b}\circ\vec{e}_a)\vec{e}_a
\end{align*}
\vfill
\end{multicols}

\newpage
\subsection{Geraden}
\index{Geraden}
\begin{multicols}{2}
\subsubsection*{Geradegleichung}
\begin{align*} 
\vec{r}(t) &=\vec{r}_1+t\vec{a}\\
	  &=\vec{r}_1+t(\vec{r}_2-\vec{r}_1)
\end{align*}
\vfill
\subsubsection*{Abstand eines Punktes von einer Geraden}
\begin{align*} 
\vec{r}(t) &=\vec{r}_1+t\vec{a}\\
d&=\frac{|\vec{a}\times\left(\vec{OP}-\vec{r}_1\right)|}{\vec{a}}
\end{align*}
\vfill
\end{multicols}

\begin{multicols}{2}
\subsubsection*{Abstand zweier paralleler Geraden}
\begin{align*} 
\vec{r}(t) &=\vec{r}_1+t\vec{a}_1\\
\vec{g}(t) &=\vec{r}_2+t\vec{a}_1\\
d&=\frac{|\vec{a}_1\times\left(\vec{r}_2-\vec{r}_1\right)|}{\vec{a}_1}
\end{align*}
\vfill
\subsubsection*{Abstand zweier windschiefen Geraden}
\begin{align*} 
\vec{r}(t) &=\vec{r}_1+t\vec{a}_1\\
\vec{g}(t) &=\vec{r}_2+t\vec{a}_2\\
d&=\frac{|\vec{a}_1\circ\left(\vec{a}_2\times\left(\vec{r}_2-\vec{r}_1\right)\right)|}{\vec{a}_1\times\vec{a}_2}
\end{align*}
\vfill
\end{multicols}

\subsection{Ebenen}
\index{Ebenen}

\begin{multicols}{2}
\subsubsection*{Ebenengleichung}
\begin{align*} 
\vec{r}(t,s) =\vec{r}_1&+t\vec{a}_1+s\vec{a}_2\\
=\vec{r}_1 &+t(\vec{r}_2-\vec{r}_1)\\
	   &+s(\vec{r}_3-\vec{r}_1)
\end{align*}
\vfill
\subsubsection*{Parameterfreie Darstellung}
\begin{align*} 
\vec{r}(t,s) &=\vec{r}_1+t\vec{a}_1+s\vec{a}_2\\
\vec{r}\circ(\vec{a}_1\times\vec{a}_2)&=\vec{r}_1\circ(\vec{a}_1\times\vec{a}_2)\\
&+t\vec{a}_1\circ(\vec{a}_1\times\vec{a}_2)\\
&+s\vec{a}_2\circ(\vec{a}_1\times\vec{a}_2)\\
\vec{r}\circ\vec{n}&=\vec{r}_1\circ\vec{n}+0+0\\
\vec{n}\circ\left(\vec{r}-\vec{r}_1\right)&=0
\end{align*}
\vfill
\end{multicols}

\begin{multicols}{2}
\subsubsection*{Normalenvektor}
\begin{align*} 
\vec{n}&=\vec{a}_1\times\vec{a}_2
\end{align*}
\vfill
\subsubsection*{Normierter Normalenvektor}
\begin{align*} 
\vec{e}_n&=\frac{\vec{a}_1\times\vec{a}_2}{|\vec{a}_1\times\vec{a}_2|}
\end{align*}
\vfill
\end{multicols}

\begin{multicols}{2}
\subsubsection*{Hessesche Normalform}
\begin{align*} 
0&=\frac{Ax+By+Cz+D}{\sqrt{A^2+B^2+C^2}}
\end{align*}
\vfill
\subsubsection*{Abstand eines Punktes von einer Ebene}
\begin{align*} 
d&=\frac{|\vec{n}\times\left(\vec{OP}-\vec{r}_1\right)|}{\vec{n}}\\
d&=\frac{Ap_1+Bp_2+Cp_3+D}{\sqrt{A^2+B^2+C^2}}
\end{align*}
\vfill
\end{multicols}

\begin{multicols}{2}
\subsubsection*{Abstand eines Geraden von einer Ebene}
\begin{align*} 
\vec{r}(t) &=\vec{r}_G+t\vec{a}_1\\
d&=\frac{|\vec{n}\times\left(\vec{r}_G-\vec{r}_1\right)|}{\vec{n}}\\
d&=\frac{Ar_{G1}+Br_{G2}+Cr_{G3}+D}{\sqrt{A^2+B^2+C^2}}
\end{align*}
\vfill
\subsubsection*{Abstand zweier paralleler Ebenen}
\begin{align*} 
\vec{r}(t,s) &=\vec{r}_1+t\vec{a}_1+s\vec{a}_2\\
\vec{g}(t,s) &=\vec{r}_2+t\vec{a}_3+s\vec{a}_4\\
d&=\frac{|\vec{n}\times\left(\vec{r}_1-\vec{r}_2\right)|}{\vec{n}}
\end{align*}
\vfill
\end{multicols}

\begin{multicols}{2}
\subsubsection*{Schnittwinkel zweier Ebenen}
\begin{align*} 
\cos\angle(\vec{n}_1,\vec{n}_2)&=\frac{\vec{n}_1\circ\vec{n}_2}{|\vec{n}_1|\cdot|\vec{n}_2|}
\end{align*}
\vfill
\subsubsection*{Durchstoßpunkt}
\begin{align*} 
\vec{r}(t) &=\vec{r}_G+t\vec{a}\\
\vec{r}_s&=\vec{r}_G+\frac{\vec{n}\circ\left(\vec{r}_1-\vec{r}_G\right)}{\vec{n}\circ\vec{a}}\vec{a}\\
\varphi&=\arcsin\left(\frac{|\vec{n}\circ\vec{a}|}{|\vec{n}|\cdot|\vec{a}|}\right)
\end{align*}
\vfill
\end{multicols}