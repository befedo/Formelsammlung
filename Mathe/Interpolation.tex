\subsection{Interpolationspolynome}
Entwicklung einer Polynomefunktion anhand von $n+1$ Kurvenpunkten.
\begin{description*}
 \item[1. Möglichkeit] Aufstellen von $n+1$ Gleichungen und ermitteln der Kurvenfunktion mithilfe des Gaußen Algorithmus.
 \item[2. Möglichkeit] Interpolationspolynome von Newton
\end{description*}

\bla{Interpolationspolynome von Newton}

\begin{merkbox}Gegeben sind die Punkte $P_0=(x_0;y_0)$, $P_1=(x_1;y_1)$, $P_2=(x_2;y_2)$, $\ldots$, $P_n=(x_n;y_n)$, damit lautet die Funktion wie folgt:
\begin{align}
 f(x)=a_0&+a_1\cdot (x-x_0)+ a_2\cdot (x-x_0)\cdot(x-x_1)\\
	 &+a_3\cdot(x-x_0)\cdot(x-x_1)\cdot(x-x_2)\\
	 &+\ldots\\
	 &+a_n\cdot(x-x_0)\cdot\ldots\cdot\cdot(x-x_{n-1})
\end{align}
Die Koeffizienten$a_0, a_1, a_2,\ldots, a_n$ lassen sich mithilfe des Differentenshema berechnen. Dabei ist $y_0=a_0$, $[x_0,x_1]=a_1$, $[x_0,x_1,x_2]=a_2$
 usw. 
\end{merkbox}

\bla{Differentenshema}

\noindent\begin{tabularx}{\linewidth}{cccccccX}
\toprule
 k	&$x_k$	&$y_k$		&$1$		&$2$		&$3$			&$\ldots$	\\ \midrule
 $0$	&$x_0$	&\hebox{$y_0$}	&		&		&			&		\\
	&	&		&\hebox{$[x_0,x_1]$}	&		&			&		\\
 $1$	&$x_1$	&$y_1$		&		&\hebox{$[x_0,x_1,x_2]$}&			&		\\
 	&	&		&$[x_1,x_2]$	&		&\hebox{$[x_0,x_1,x_2,x_3]$}	&		\\
 $2$	&$x_2$	&$y_2$		&		&\heboxc{$[x_1,x_2,x_3]$}&			&$\ldots$	\\
 	&	&		&$[x_2,x_3]$	&		&\heboxc{$[x_1,x_2,x_3,x_4]$}	&		\\
 $3$	&$x_3$	&$y_3$		&		&\heboxc{$[x_2,x_3,x_4]$}&			&$\ldots$	\\
 	&	&		&$\ldots$	&		&$\ldots$		&		\\
 $\vdots$&$\vdots$&$\vdots$	&		&		&			&		\\
 $n$	&$x_n$	&$y_n$		&		&		&			&		\\ \bottomrule
\end{tabularx}

\bla{Rechenregel für dividierte Differenzen}
\begin{shaded}
\begin{minipage}{.5\textwidth}

 \begin{equation}
 \left.\begin{aligned}
  [x_0,x_1]&=\frac{y_0-y_1}{x_0-x_1}\\
  [x_1,x_2]&=\frac{y_1-y_2}{x_1-x_2} \\
\vdots &
 \end{aligned}\right.
\end{equation} 
\end{minipage}\begin{minipage}{.5\textwidth}
 \begin{equation}
 \left.\begin{aligned}
  [x_0,x_1,x_2]&=\frac{[x_0,x_1]-[x_1,x_2]}{x_0-x_2}\\
  [x_1,x_2,x_3]&=\frac{[x_1,x_2]-[x_2,x_3]}{x_1-x_3} \\
\vdots &
 \end{aligned}\right.
\end{equation} 

\end{minipage}\end{shaded}
\begin{shaded}
 \begin{equation}
 \left.\begin{aligned}
  [x_0,x_1,x_2,x_3]&=\frac{[x_0,x_1,x_2]-[x_1,x_2,x_3]}{x_0-x_2}\\
  [x_1,x_2,x_3,x_4]&=\frac{[x_1,x_2,x_3]-[x_2,x_3,x_4]}{x_1-x_3} \\
\vdots &
 \end{aligned}\right.
\end{equation} 
\end{shaded}
