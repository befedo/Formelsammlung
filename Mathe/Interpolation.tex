\section{Interpolationspolynome}
\index{Interpolation}
Entwicklung einer Polynomfunktion anhand von \(n+1\) Kurvenpunkten.
\begin{itemize}
 \item 1. Möglichkeit: Aufstellen von $n+1$ Gleichungen und ermitteln der Kurvenfunktion mithilfe des Gauß' Algorithmus.
 \item 2. Möglichkeit: Interpolationspolynome nach Newton.
\end{itemize}

\subsubsection{Interpolationspolynome nach Newton}
\index{Interpolation!nach Newton}
\(\text{Gegeben sind die Punkte: } \\ P_0=(x_0;y_0), P_1=(x_1;y_1), P_2=(x_2;y_2), \ldots, P_n=(x_n;y_n), \\ \text{damit lautet die Funktion wie folgt.}\)
\begin{align*}
 f(x)=a_0&+a_1\cdot (x-x_0)+ a_2\cdot (x-x_0)\cdot(x-x_1)\\
	 &+a_3\cdot(x-x_0)\cdot(x-x_1)\cdot(x-x_2)\\
	 &+\ldots\\
	 &+a_n\cdot(x-x_0)\cdot\ldots\cdot\cdot(x-x_{n-1})
\end{align*}
\( \text{Die Koeffizienten } a_0, a_1, a_2,\ldots, a_n \text{ lassen sich mithilfe des Differentenschema} \\ \text{berechnen. Dabei ist } y_0=a_0, [x_0,x_1]=a_1, [x_0,x_1,x_2]=a_2 \text{usw. .}\) 

\subsubsection{Differentenschema}
\index{Interpolation!Differentenschema}

\newcommand{\mcf}[3]{\multicolumn{#1}{#2}{#3}}
\definecolor{tcA}{rgb}{0.627451,0.627451,0.643137}
\begin{center}
\begin{tabular}{|c|c|c|c|c|c|c|}\hline
\rowcolor{tcA}
\(k\) & \(x_k\) & \(y_k\) & 1 & 2 & 3 & \(\ldots\)\\\hline
\mcf{1}{|>{\columncolor{tcA}}c|}{0} & \(x_0\) 	& \(y_0\) &   &   &   &  \\\hline
\mcf{1}{|>{\columncolor{tcA}}c|}{ } &   		& 	  & \(\left[x_0,x_1\right]\) &   &   &  	\\\hline
\mcf{1}{|>{\columncolor{tcA}}c|}{1} & \(x_1\) 	& \(y_1\) &   & \(\left[x_0,x_1,x_2\right]\) &   &  	\\\hline
\mcf{1}{|>{\columncolor{tcA}}c|}{ } &   		& 	  & \(\left[x_1,x_2\right]\) &  &\(\left[x_0,x_1,x_2,x_3\right]\)   &  \\\hline
\mcf{1}{|>{\columncolor{tcA}}c|}{2} & \(x_2\) 	& \(y_2\) &   & \(\left[x_1,x_2,x_3\right]\) &   &  \(\ldots\)\\\hline
\mcf{1}{|>{\columncolor{tcA}}c|}{ } &   		& 	  & \(\left[x_2,x_3\right]\) &  &\(\left[x_0,x_1,x_2,x_3\right]\)   &  \\\hline
\mcf{1}{|>{\columncolor{tcA}}c|}{3} & \(x_3\) 	& \(y_3\) &   & \(\left[x_2,x_3,x_4\right]\)  &   & \(\ldots\)\\\hline
\mcf{1}{|>{\columncolor{tcA}}c|}{ } &   		& 	  & \(\ldots\) & \(\ldots\) & \(\ldots\) &  \\\hline
\mcf{1}{|>{\columncolor{tcA}}c|}{\(\vdots\)} 	& \(\vdots\) & \(\vdots\) &   &   &   &  \\\hline
\mcf{1}{|>{\columncolor{tcA}}c|}{n} & \(x_n\) 	& \(y_n\) &   &   &   &  \\\hline
\end{tabular}
\end{center}

\subsubsection{Rechenregeln für dividierte Differenzen}
\index{Interpolation!Rechenregeln}

\begin{alignat*}{2}
&\left[x_0,x_1\right]=\frac{y_0-y_1}{x_0-x_1} && \left[x_1,x_2\right]=\frac{y_1-y_2}{x_1-x_2} \\
&\left[x_0,\hdots,x_2\right]=\frac{\left[x_0,x_1\right]-\left[x_1,x_2\right]}{x_0-x_2} && \left[x_1,\hdots,x_3\right]=\frac{\left[x_1,x_2\right]-\left[x_2,x_3\right]}{x_1-x_3} \\
&\left[x_0,\hdots,x_3\right]=\frac{\left[x_0,x_1,x_2\right]-\left[x_1,x_2,x_3\right]}{x_0-x_2} && \left[x_1,\hdots,x_4\right]=\frac{\left[x_1,x_2,x_3\right]-\left[x_2,x_3,x_4\right]}{x_1-x_3}
\end{alignat*}
