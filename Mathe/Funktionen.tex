\section{Gleichungen}
\subsection{Gleichungen \emph{n}-ten Grades}

 \begin{equation*}
  a_n\cdot x^n+a_{n-1}\cdot x^{n-1}+\ldots+a_1\cdot x+a_0=0\quad (a_n\neq0,a_k\in\mathbb{R})
 \end{equation*}

\subsubsection*{Eigenschafften}
\begin{itemize}
 \item Die Gleichung besitzen maximal $n$ reelle Lösungen.
\item Es gibt genau $n$ komplexe Lösungen.
\item Für ungerades $n$ gibt es mindestens eine reelle Lösung.
\item Komplexe Lösungen treten immer Paarweise auf.
\item Es existieren nur Lösungsformeln bis $n\leq 4$. Für $n>4$ gibt es nur noch grafische oder numerische Lösungswege.
\item Wenn eine Nullstelle bekannt ist kann man die Gleichung um einen Grad verringern, indem man denn zugehörigen Linearfaktor $x -x_1$ abspaltet(Polynome Division).
\end{itemize}

\subsection{Lineare Gleichungen}
 \begin{equation*}
  a_1\cdot x+a_0=0 \Rightarrow x_1=-\frac{a_0}{a_1}\quad (a_1\neq 0)
 \end{equation*}

\subsection{Quadratische Gleichungen}
 \begin{equation*}
  a_2\cdot x^2+a_1\cdot x+a_0=0\quad (a_2\neq0)
 \end{equation*}

 Normalform mit Lösung
 \begin{equation*}
  x^2+p\cdot x+q=0\Rightarrow x_{1/2}=-\frac{p}{2}\pm\sqrt{\left(\frac{p}{2}\right)^2-q}
 \end{equation*}

\emph{Überprüfung (Vietascher Wurzelsatz)}
 \begin{align*}
  x_1+x_2&=-p& x_1\cdot x_2&=q 
 \end{align*}
$x_1, x_2:$ Lösung der quadratischen Gleichung.

\subsection{Biquadratische Gleichungen}
Diese Gleichungen lassen sich mithilfe der Substitution lösen.
 \begin{align*}
  a\cdot x^4+b\cdot x^2 +c&=0&u&=x^2 \\
  a\cdot u^2+b\cdot u +c&=0&x&=\pm\sqrt{u}
 \end{align*}
Das $u$ kann mithilfe der Lösungsformel einer quadratischen Gleichung gelöst werden.

\subsection{Gleichungen höheren Grades} 
Gleichungen höheren Grades kann man durch graphische oder numerische Ansätze lösen. Hilfreich ist das finden einer Lösung und das abspalten eines Linearfaktor
, mithilfe der Polynomdivision oder dem Hornor Schema,von der ursprünglichen Gleichung.

\emph{Polynomdivision}
 \begin{equation*}
  \frac{f(x)}{x-x_0}=\frac{a_3\cdot x^3+a_2\cdot x^2+a_1\cdot x +a_0}{x-x_0}=b_2\cdot x^2+b_1 \cdot x+b_0+r(x)
 \end{equation*}
$x_0$ ist dabei die erste gefunden Nullstelle. r(x) verschwindet wenn $x_0$ ein Nullstellen oder eine Lösung von f(x) ist.

 \begin{equation*}
  r(x)=\frac{a_3\cdot x_0^3+a_2\cdot x_0^2+a_1\cdot x_0 +a_0}{x-x_0}=\frac{f(x_0)}{x-x_0}
 \end{equation*}

\subsection{Wurzelgleichung}
Wurzelgleichungen löst man durch quadrieren oder mit hilfe von Substitution.
Bei Wurzelgleichung ist zu beachten das quadrieren keine Aquivalente Umformung ist und das 
Ergebniss überprüft werden muss.

\subsection{Ungleichungen}
\begin{itemize}
\item Beidseitiges Subtrahieren oder Addieren ist möglich
\item Die Ungleichung darf mit einer beliebige positiven Zahl multipliziert oder dividiert werden
\item Die Ungleichung darf mit einer beliebige negativen Zahl multipliziert oder dividiert werden, wenn man gleichzeitig das Relationszeichen umdreht.
\end{itemize}

\subsection{Betragsgleichungen}
Betragsgleichungen löst man mithilfe der Fallunterscheidung. Dabei wird einmal davon ausgegangen das der Term inerhalb des Betrags einmal positiv und einmal negativen
sein kann.
\begin{align*}
y&=|x|=
\begin{cases}
x \text{ für } x \geq 0 \\
-x \text{ für } x < 0
\end{cases}
\end{align*}