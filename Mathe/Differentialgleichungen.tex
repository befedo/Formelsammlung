\newpage
\section{Differentialgleichungen}

\begin{align*}
\text{Anfangswertproblem:}\quad&\text{ Werte nur an einer Stelle vorgegeben}\\
\text{Randwertproblem:}\quad&\text{ Werte an mehreren Stellen vorgegeben}
\end{align*}


\subsubsection*{Lineare DG}
\begin{align*}
y_{all}&=y_h+y_p
\end{align*}

\subsection{DG 1. Ordnung}
\begin{multicols}{2}
\subsubsection*{Trennung der variablen}
\begin{align*}
y'(x)&=f(x)\cdot g(y)\\
\int\frac{\diff y}{g(y)}&=\int f(x)\diff x
\end{align*}
\vfill
\subsubsection*{Lineare DG}
\[y'+f(x)\cdot g(y)=g(x) \text{$g(x)=0\Rightarrow$homogen}\]
\[y_{all}=e^{-F(x)}\cdot\left(\int g(x)\cdot e^{F(x)}\diff x+C\right)\]
\vfill
\end{multicols}

\subsection{Lineare DG 2. Ordnung}
\begin{multicols}{2}
\subsubsection*{Darstellung}
\[a(x)\cdot y''+b(x)\cdot y'+c(x)\cdot y=g(x)\]
\[\text{$g(x)=0\Rightarrow$homogen}\]
\vfill
\subsubsection*{Fundamental Lösungen}
\begin{align*}
&a\lambda^2+b\lambda+c=0\\
&\text{$\lambda_{1/2}=\alpha\pm \beta\cdot j$}\\ {} \\
&y_h=C_1 e^{\lambda_1 x}+C_2 e^{\lambda_2 x} \quad \text{$\lambda_1\neq\lambda_2$}\\
&y_h=C_1 e^{\lambda_1 x}+C_2 x e^{\lambda_2 x} \quad \text{$\lambda_1=\lambda_2$}\\
&y_h=C_1 e^{\alpha x}\cdot\cos{\left(\beta x\right)}\\&+C_2  e^{\alpha x}\cdot\sin{\left(\beta x\right)}
\end{align*}
\vfill
\end{multicols}

\subsubsection*{In Folgenden Aufzählungen gillt:}
\begin{center}
 \begin{itemize}
 \item \(G(x)\) Ansatz
 \item \(g(x)\) Störglied
 \item \(r\) Anzahl der Resonanzfälle
\end{itemize}
\end{center}


\subsubsection*{Partikuläre Lösungen(Polynom)}
\begin{align*}
&a\lambda^2+b\lambda+c=0\\
&g(x)=b_0+b_1x+b_2x^2+\dots+b_nx^n\\
&G(x)=B_0+B_1x+B_2x^2+\dots+B_nx^n&&\text{$\lambda\neq 0$}\\
&G(x)=\left(B_0+B_1x+B_2x^2+\dots+B_nx^n\right)\cdot x^r&&\text{$\lambda = 0$}
\end{align*}

\subsubsection*{Partikuläre Lösungen(Polynom und e-Funktion)}
\begin{align*}
&a\lambda^2+b\lambda+c=0\\
&g(x)=\left(b_0+b_1x+b_2x^2+\dots+b_nx^n\right)e^{mx}\\
&G(x)=\left(B_0+B_1x+B_2x^2+\dots+B_nx^n\right)e^{mx}&&\text{$\lambda\neq m$}\\
&G(x)=\left(B_0+B_1x+B_2x^2+\dots+B_nx^n\right)e^{mx}\cdot x^r&&\text{$\lambda = m$}
\end{align*}

\newpage
\subsubsection*{Partikuläre Lösungen(sin- und cos Funktion)}
\begin{align*}
&a\lambda^2+b\lambda+c=0\\
&g(x)=a\cos\left(kx\right)+b\sin\left(kx\right)\\
&G(x)=A\cos\left(kx\right)+B\sin\left(kx\right)&&\text{$\lambda\neq \pm kj$}\\
&G(x)=A\cos\left(kx\right)+B\sin\left(kx\right)\cdot x^r&&\text{$\lambda = \pm kj$}
\end{align*}

\subsubsection*{Partikuläre Lösungen(e-, sin- und cos Funktion)}
\begin{align*}
0&=a\lambda^2+b\lambda+c\\
g(x)&=\left(b_0+b_1x+b_2x^2+\dots+b_nx^n\right)e^{mx}\cdot\left(c\cos\left(kx\right)+d\sin\left(kx\right)\right)\\
G(x)&=\left(B_0+B_1x+B_2x^2+\dots+B_nx^n\right)e^{mx}\cdot\left(C\cos\left(kx\right)+D\sin\left(kx\right)\right)\\
\lambda&\neq m \pm kj\\
G(x)&=\left(B_0+B_1x+B_2x^2+\dots+B_nx^n\right)e^{mx}\cdot\left(C\cos\left(kx\right)+D\sin\left(kx\right)\right)\cdot x^r\\
\lambda&= m \pm kj
\end{align*}