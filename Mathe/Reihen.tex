\section{Reihen}
\subsection{Geometrische Folge}

\subsubsection{Darstellung}

\begin{alignat*}{2}
&a_n = a \cdot q^n &\quad& \sum_{n=0}^\infty a \cdot q^n = \frac{a}{1-q} \\
&\text{Konvergent für} \left| q \right| < 1 
\end{alignat*}

\subsection{Harmonische Reihe}

\subsubsection{Darstellung}

\begin{align*}
&\sum_{n=1}^\infty \frac{1}{n^s} \\
&\text{Konvergent für } s > 1
\end{align*}

\newpage
\subsection{Konvergenz}

\begin{multicols}{2}
\subsubsection{Majorantenkriterium}
\begin{align*}
&\sum_{n=0}^\infty a_n\leq\sum_{n=0}^\infty b_n \\
& b_n \text{ bekannte konvergente Reihe}
\end{align*}

\subsubsection{Minorantenkriterium}
\begin{align*}
&\sum_{n=0}^\infty a_n\geq\sum_{n=0}^\infty b_n \\
& b_n \text{ bekannte divergente Reihe}
\end{align*}
\end{multicols}


\subsubsection{Wurzelkriterium}
\[
\lim_{n \to \infty} \sqrt[n]{a_n} = q
\begin{cases}
 q > 1 \text{ ist die Reihe divergent}\\
 q < 1 \text{ ist die Reihe konvergent}\\
 q = 1 \text{ ist keine Aussage möglich}
\end{cases}
\]

\subsubsection{Quotientenkriterium}
\[
\lim_{n\to\infty} \frac{a_{n+1}}{a_n} = q
\begin{cases}
q > 1 \text{ ist die Reihe divergent}\\
q < 1 \text{ ist die Reihe konvergent}\\
q = 1 \text{ ist keine Aussage möglich}
\end{cases}
\]

\subsubsection{Leibnizkriterium}
\text{Nur bei alternierenden Reihen}

\begin{align*}
&\lim_{n\to\infty}\left(-1\right)^n a_n\\
&\lim_{n\to\infty}a_n=q&&\text{$q=0$ ist die Reihe divergent}\\
&\lim_{n\to\infty}\left(-1\right)^n a_n=\lim_{n\to\infty} a_n&&\text{Absolut Konvergent}\\
\end{align*}

\subsection{Bekannte konvergente Reihen}

\begin{alignat*}{3}
&\sum_{n=0}^\infty\frac{1}{n!}=e &\quad\quad& \sum_{n=0}^\infty\frac{\left(-1\right)^n}{n!} = \frac{1}{e} &\quad\quad& \sum_{n=0}^\infty\frac{1}{2^n} = 2 \\
&\sum_{n=0}^\infty\frac{\left(-1\right)^n}{2^n} = \frac{2}{3} &\quad\quad& \sum_{n=0}^\infty\frac{\left(-1\right)^{n+1}}{n} =\ln2 &\quad\quad& \sum_{n=0}^\infty\frac{\left(-1\right)^{n+1}}{2n-1}=\frac{\pi}{4}
\end{alignat*}


\section{Funktionenreihen}

\subsubsection{Darstellung}

\begin{align*}
\sum_{n=0}^\infty f_n(x)
\end{align*}


\subsection{Potenzreihen}

\begin{multicols}{2}
\subsubsection{Darstellung}
\begin{align*}
&\sum_{n=0}^\infty a_n x^n\\
&\sum_{n=0}^\infty a_n \left(x-x_0\right)^n\\
&x_0: \text{ Verschiebung des} \\ &\text{Entwicklungspunktes.}
\end{align*}
\vfill
\subsubsection{Konvergenz}
\begin{align*}
r&=\lim_{n\to\infty}\left|\frac{a_n}{a_{n+1}}\right|\\
r&=\frac{1}{\lim_{n\to\infty}\sqrt[n]{\left|a_n\right|}}\\
&\text{Ränder müssen} \\ &\text{untersucht werden.}
\end{align*}
\vfill
\end{multicols}


\subsection{Bekannte Potenzreihen}
\begin{align*}
e^x&=\sum_{n=0}^\infty\frac{x^n}{n!}&x&\in\mathbb{R}\\
\ln x&=\sum_{n=1}^\infty\frac{\left(-1\right)^{n-1}}{n}\left(x-1\right)^n&x&\in(0,2]\\
\ln\left(1+x\right)&=\sum_{n=1}^\infty\frac{\left(-1\right)^{n-1}}{n}x^n&x&\in(-1,1]\\
\ln\left(1-x\right)&=-\sum_{n=1}^\infty\frac{x^n}{n}&x&\in[-1,1)\\
\left(1+x\right)^\alpha&=\sum_{n=0}^\infty\binom{\alpha}{n}x^n&x&\in[-1,1]
\end{align*}

\newpage
\subsection{spezielle Reihen}

\begin{align*}
\sin x&=\sum_{n=0}^\infty\frac{\left(-1\right)^n}{\left(2n+1\right)!}x^{2n+1}&x&\in\mathbb{R}\\
\cos x&=\sum_{n=0}^\infty\frac{\left(-1\right)^n}{\left(2n\right)!}x^{2n}&x&\in\mathbb{R}\\
\sinh x&=\sum_{n=0}^\infty\frac{1}{\left(2n+1\right)!}x^{2n+1}&x&\in\mathbb{R}\\
\cosh x&=\sum_{n=0}^\infty\frac{1}{\left(2n\right)!}x^{2n}&x&\in\mathbb{R}\\
\arcsin x &=\sum_{n=0}^\infty\frac{\left(2n\right)!}{2^{2n}\left(n!\right)^2\left(2n+1\right)}x^{2n+1}&x&\in[-1,1]\\
\arctan x &=\sum_{n=0}^\infty\frac{\left(-1\right)^n}{\left(2n+1\right)}x^{2n+1}&x&\in\mathbb{R}\\
\operatorname{ar sinh} x &=\sum_{n=0}^\infty\frac{\left(-1\right)^n\left(2n\right)!}{2^{2n}\left(n!\right)^2\left(2n+1\right)}x^{2n+1}&x&\in[-1,1]\\
\operatorname{ar tanh}  x &=\sum_{n=0}^\infty\frac{1}{\left(2n+1\right)}x^{2n+1}&x&\in\mathbb{R}\\
\end{align*}


\subsection{Fourier Reihen}

\subsubsection{Allgemein}
\begin{align*}
y(t)&=\frac{a_0}{2}+\sum_{n=1}^\infty\left(a_n\cdot\cos\left(n\omega_0 t\right)+a_n\cdot\sin\left(n\omega_0 t\right)\right)\\
a_0&=\frac{2}{T}\int_{(T)}y(t)\diff t\\
a_n&=\frac{2}{T}\int_{(T)}y(t)\cdot\cos\left(n\omega_0 t\right)\diff t\\
b_n&=\frac{2}{T}\int_{(T)}y(t)\cdot\sin\left(n\omega_0 t\right)\diff t
\end{align*}


\newpage
\subsubsection{Symetrie}
\begin{multicols}{2}
\begin{align*}
 y(t)&=\frac{a_0}{2}+\sum_{n=1}^\infty\left(a_n\cdot\cos\left(n\omega_0 t\right)\right) \\ &\text{gerade Funktion } b_n = 0
\end{align*}

\begin{align*}
 y(t)&=\sum_{n=1}^\infty\left(b_n\cdot\sin\left(n\omega_0 t\right)\right) \\ &\text{ungerade Funktion } a_n = 0 
\end{align*}
\end{multicols}

\subsubsection{Komplex}
\begin{alignat*}{2}
y(x)&=\sum_{n=-\infty}^\infty c_n\cdot e^{jnx} &\quad\quad\quad&
c_n=\frac{1}{T}\int_{(T)}y(x)\cdot e^{-jnx}\diff x
\end{alignat*}


\subsubsection{Umrechnung}

\begin{align*}
c_0&=\frac{1}{2}a_0\\
c_n&=\frac{1}{2}\left(a_n-jb_n\right)\\
c_{-n}&=\frac{1}{2}\left(a_n+jb_n\right)\\
a_0&=2c_0\\
a_n&=c_n+c_{-n}\\
b_n&=j\left(c_n-c_{-n}\right)
\end{align*}

