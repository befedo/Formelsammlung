\section{Differntialrechnung}

\subsection{Erste Ableitungen der elementaren Funktionen}

\begin{multicols}{2}
\subsubsection*{Potenzfunktion}
 \begin{align*}
  &x^n& &\Longleftrightarrow& &n\cdot x^{n-1}
 \end{align*}
\vfill
\subsubsection*{Exponentialfunktionen}
 \begin{align*}
  &e^x& &\Longleftrightarrow& &e^x\\
  &a^x& &\Longleftrightarrow& &\ln a\cdot a^x
 \end{align*}
\vfill
\end{multicols}

\begin{multicols}{2}
\subsubsection*{Logarithmusfunktionen}
 \begin{align*} 
  &\ln x& &\Longleftrightarrow& &\frac{1}{x}\\
  &\log_a x& &\Longleftrightarrow& &\frac{1}{(\ln a)\cdot x}
 \end{align*}
\vfill
\subsubsection*{Trigonometrische Funktionen}
 \begin{align*} 
  &\sin x& &\Longleftrightarrow& &\cos x \\
  &\cos x& &\Longleftrightarrow& &-\sin x\\
  &\tan x& &\Longleftrightarrow& &\frac{1}{\cos^2 x}\\
  &\tan x& &\Longleftrightarrow& &1+\tan^2 x
\end{align*}
\vfill
\end{multicols}

\newpage
\begin{multicols}{2}
\subsubsection*{Arcusfunktionen}
 \begin{align*} 
  &\arcsin x& &\Longleftrightarrow& &\frac{1}{\sqrt{1-x^2}}\\
  &\arccos x& &\Longleftrightarrow& &\frac{-1}{\sqrt{1-x^2}}\\
  &\arctan x& &\Longleftrightarrow& &\frac{1}{1-x^2}
\end{align*}
\vfill
\subsubsection*{Hyperbolische Funktionen}
 \begin{align*} 
  &\sinh x& &\Longleftrightarrow& &\cosh x\\
  &\cosh x& &\Longleftrightarrow& &\sinh x\\
  &\tanh x& &\Longleftrightarrow& &\frac{1}{\cosh^2 x}\\
  &\tanh x& &\Longleftrightarrow& &1+\tanh^2 x
\end{align*}
\vfill
\end{multicols}

\subsection{Rechenregeln}
\begin{multicols}{2}
\subsubsection*{Faktorregel}
 \begin{align*} 
\frac{\diff }{\diff x}\left(C\cdot f(x)\right)=C\cdot f'(x)
 \end{align*}
\vfill
\subsubsection*{Summenregel}
 \begin{align*} 
\frac{\diff }{\diff x}\left(g(x) + f(x)\right)=g'(x)+f'(x)
 \end{align*}
\vfill
\end{multicols}


\subsubsection*{Produktregel}
 \begin{align*} 
\frac{\diff }{\diff x}&\left(g(x)\cdot f(x)\right)=g'(x)\cdot f(x)+g(x)\cdot f'(x)\\
\frac{\diff }{\diff x}&\left(h(x)\cdot g(x)\cdot f(x)\right)=h'\cdot g\cdot f+h\cdot g'\cdot f+h\cdot g\cdot f'
 \end{align*}

\subsubsection*{Quotientenregel}
 \begin{align*} 
\frac{\diff }{\diff x}\left(\frac{g(x)}{f(x)}\right)&=\frac{g'(x)\cdot f(x)-g(x)\cdot f'(x)}{f(x)^2}
 \end{align*}

\begin{multicols}{2}
\subsubsection*{Kettenregel}
 \begin{align*} 
\frac{\diff }{\diff x}\left(g\left( f(x\right)\right)&=g'(f)\cdot f'(x)
 \end{align*}
\vfill            
\subsubsection*{Logarithmische Ableitungen}
 \begin{align*} 
\frac{\diff }{\diff x}y&=f(x)\\
\frac{1}{y}y'&=\frac{\diff }{\diff x}\ln{f(x)}
 \end{align*}
\vfill
\end{multicols}
          
\subsection{Fehlerrechnung}

\subsubsection*{Absoluter Fehler}
\text{\(\Delta x\)}{ Absoluter Fehler der Eingangsgröße}\\
\text{\(\Delta y\)}{ Absoluter Fehler der Ausgangsgröße}
\begin{align*} 
\Delta y=f(x+\Delta x)-f(x)
\end{align*}
         
\subsubsection*{Relativer Fehler}
\text{\(\delta x\)}{ Relativer Fehler der Eingangsgröße in $\%$}\\
\text{\(\delta y\)}{ Relativer Fehler der Ausgangsgröße in $\%$}

\begin{align*} 
\delta x&=\frac{\Delta x}{x}\\
\delta y&=\frac{\Delta y}{y}\\
\Delta y&=f'(x)\cdot \Delta x\\
\delta y&=\frac{x\cdot f'(x)}{f(x)}\delta x
\end{align*}
         
\subsection{Linearisierung und Taylor-Polynom}

\subsubsection*{Tangentengleichung}
\text{\(x_0\) Punkt an dem das Polynom entwickelt wird}

\begin{align*} 
y_T(x)&=f(x_0)+f'(x_0)(x-x_0)
\end{align*}

\subsubsection*{Taylor Polynom}
\text{\(x_0\) Punkt an dem das Polynom entwickelt wird}\\
\text{\(R_n\) Restglied}
 
\[y(x)=f(x_0)+f'(x_0)(x-x_0)+\frac{f''(x_0)}{2!}(x-x_0)^2+\dots+\frac{f^{(n)}(x_0)}{n!}(x-x_0)^n+R_n(x)\]
\[y(x)=\sum_{i=0}^n\frac{f^{(i)}}{i!}(x-x_0)^i+R_n(x)\]

\subsubsection*{Restglied}
\text{\(x_0\) Punkt an dem das Polynom entwickelt wird}\\
\text{$x_0<c<x$, wenn $x_0<x$}\\
\text{$x_0>c>x$, wenn $x_0>x$}

\begin{align*} 
R_n(x)=\frac{f^{(n+1)}(c)}{(n+1)!}(x-x_0)^{n+1}
\end{align*}

\subsection{Grenzwertregel von Bernoulli und de l`Hospital}
\subsubsection*{de l`Hospital}
\text{Gilt nur wenn $\lim\limits_{x \to x_0} f(x)$ gleich $\frac{0}{0}$ oder $\frac{\infty}{\infty}$ ist}

\begin{align*} 
 \lim_{x \to x_0}\frac{f(x)}{g(x)}=\lim_{x \to x_0}\frac{f'(x)}{g'(x)}
\end{align*}

\subsection{Differentielle Kurvenuntersuchung}


\subsubsection*{Normale der Kurve}

\begin{align*} 
y_N(x)&=f(x_0)-\frac{1}{f'(x)}\left(x-x_0\right)
\end{align*}

\begin{multicols}{2}
\subsubsection*{Monotonie-Verhalten}
\[f'(x)=
\begin{cases}
>0\text{ Monoton wachsend}\\
<0\text{ Monoton fallend}
\end{cases}\]
\vfill
\subsubsection*{Krümmungs-Verhalten}
\[ 
f''(x)=
\begin{cases}
>0\text{ Linkskr.(konvex)}\\ 
<0\text{ Rechtskr.(konkav)}
\end{cases}\]
\end{multicols}

\begin{multicols}{2}
\subsubsection*{Ableitung in Polarkordinaten}
\text{\(\dot{r}\)}{ Ableitung nach $\varphi$}\\
\text{\(\ddot{r}\)}{ Zweite Ableitung nach $\varphi$}
\begin{align*} 
y(\varphi)&=r(\varphi)\sin\varphi\\
x(\varphi)&=r(\varphi)\cos\varphi\\
y'&=\frac{\diff y}{\diff x}=\frac{r'\sin\varphi+r\cos\varphi}{r'\cos\varphi-r\sin\varphi}\\
y''&=\frac{\diff^2 y}{\diff x^2}=\frac{2(r')^2-r\cdot r''+r^2}{\left(r'\cos\varphi-r\sin\varphi\right)^3}
\end{align*}
\vfill
\subsubsection*{Ableitung in Parameterform}
\text{\(\dot{x}\)}{ Ableitung nach t}\\
\text{\(\dot{y}\)}{ Ableitung nach t}
\begin{align*} 
y&=y(t)\\
x&=x(t)\\
y'&=\frac{\diff y}{\diff x}=\frac{\dot{y}}{\dot{x}}\\
y''&=\frac{\diff^2 y}{\diff x^2}=\frac{\dot{x}\ddot{y}-\dot{y}\ddot{x}}{\dot{x}^3}
\end{align*}
\vfill
\end{multicols}

\begin{multicols}{2}
\subsubsection*{Bogendifferential}
\text{''Wegelement'' einer Funktion}
\begin{align*} 
\diff s&=\sqrt{1+\left(f'(x)\right)^2}\cdot \diff x\\
\diff s&=\sqrt{(\dot{x})^2+(\dot{y})^2}\cdot \diff t\\
\diff s&=\sqrt{r^2+(r')^2}\cdot \diff \varphi
\end{align*}
\subsubsection*{Winkeländerung}
\begin{align*} 
\tau&=\arctan y'\\
\diff \tau&=\frac{y''}{1+(y')^2}\cdot \diff x
\end{align*}
\vspace{5mm}
\vfill
\end{multicols}


\begin{multicols}{2}
\subsubsection*{Krümmungskreis}
\begin{align*} 
\rho&=\frac{1}{|\kappa|}\\
x_K&=x_P-y'\frac{1+(y')^2}{|y''|}\\
y_K&=y_P+\frac{1+(y')^2}{|y''|}\\
\rho&: \text{Radius}\\
\left(x_K, y_K \right)&: \text{Kreismittelpunkt} \\
\left( x_P, y_P \right)&: \text{Kurvenpunkt}
\end{align*}

\subsubsection*{Kurvenkrümmung}
\begin{align*} 
\kappa&=\frac{\diff \tau}{\diff s}\\
&=\frac{y''}{\sqrt{\left(1+(y')^2\right)^3}}\\
&=\frac{\dot{x}\ddot{y}-\dot{y}\ddot{x}}{\sqrt{\left(\dot{x}^2+\dot{y}^2\right)^3}}\\
&=\frac{2(r')^2-r\cdot r''+r^2}{\sqrt{\left(r^2+(r')^2\right)^3}}
\end{align*}
\vfill
\end{multicols}