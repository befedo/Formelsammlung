\section{Komplexe Zahlen}
\index{Komplexe Zahlen}
\begin{boxleft}\bla{Grundlagen}
\end{boxleft}\begin{boxrightshaded}
 \begin{align} 
j&=\sqrt{-1}\\
j^2&=-1
\end{align}\end{boxrightshaded}

\subsection{Darstellungsformen}
\index{Komplexe Zahlen!Darstellungsformen}

\begin{boxleft}\bla{Kartesische Form}
  \des[]{x}{Realanteil}\\
  \des[]{y}{Imaginäranteil}
\end{boxleft}\begin{boxrightshaded}
 \begin{align} 
z&=x+jy
\end{align}\end{boxrightshaded}

\begin{boxleft}
  \bla{Trigometrische Form}
\des[]{r}{Betrag}\\
\des[]{\varphi}{Argument}
\end{boxleft}\begin{boxrightshaded}
 \begin{align} 
z&=r\left(\cos{\varphi}+j\sin{\varphi}\right)
\end{align}\end{boxrightshaded}

\begin{boxleft}
  \bla{Exponentialform}
\end{boxleft}\begin{boxrightshaded}
 \begin{align} 
z&=re^{j\varphi}
\end{align}\end{boxrightshaded}

\begin{boxleft}
  \bla{Umrechnung}
\end{boxleft}\begin{boxrightshaded}
 \begin{align} 
x&=r\cos{\varphi}\\
y&=r\sin{\varphi}\\
r&=\left|z\right|=\sqrt{x^2+y^2}
\end{align}\end{boxrightshaded}

\begin{boxleft}
  \bla{Umrechnung Winkel}
\end{boxleft}\begin{boxrightshaded}
 \begin{align} 
\tan{\varphi}&=\frac{y}{x}\\
\varphi&=\begin{dcases*}
  \arctan{\frac{y}{x}}& Quadrant I\\
\arctan{\frac{y}{x}}+\pi& Quadrant II,III\\
\arctan{\frac{y}{x}}+2\pi& Quadrant IV
\end{dcases*}
\end{align}\end{boxrightshaded}

\subsection{Rechenregeln}
\index{Komplexe Zahlen!Rechenregeln}

\begin{boxleft}
  \bla{Konjugiert komplexe Zahl}
\des[]{\overline{z}}{konjugierte Komplexe}
\end{boxleft}\begin{boxrightshaded}
 \begin{align} 
\overline{z}&=z^*\\
\overline{z}&=\overline{x+jy}\\
&=x-jy\\
\overline{z}&=\overline{r\left(\cos{\varphi}+j\sin{\varphi}\right)}\\
&=r\left(\cos{\varphi}-j\sin{\varphi}\right)\\
\overline{z}&=\overline{re^{j\varphi}}\\
&=re^{-j\varphi}
\end{align}\end{boxrightshaded}

\begin{boxleft}\bla{Addition und Subtraktion}
\end{boxleft}\begin{boxrightshaded}
\begin{align} 
  z_1\pm z_2&=(x_1+jy_1)\pm(x_2+jy_2)\\
	    &=(x_1\pm x_2)+ j(y_1 \pm y_2)
\end{align}\end{boxrightshaded}

\begin{boxleft}\bla{Multiplikation}
\end{boxleft}\begin{boxrightshaded}
\begin{align} 
 z_1 \cdot z_2&=(x_1+jy_1)\cdot(x_2+jy_2)\\
	      &=(x_1x_2-y_1y_2)+ j(x_1y_2+x_2y_1)\\
 z_1 \cdot z_2&=r_1\left(\cos{\varphi_1}+j\sin{\varphi_1}\right)\cdot r_2\left(\cos{\varphi_2}+j\sin{\varphi_2}\right)\\
	      &=r_1r_2\left(\cos(\varphi_1+\varphi_2)+j\sin(\varphi_1+\varphi_2)\right)\\
 z_1 \cdot z_2&=r_1e^{j\varphi_1}\cdot r_2e^{j\varphi_2}\\
	      &=r_1r_2e^{j(\varphi_1+\varphi_2)}
\end{align}\end{boxrightshaded}

\begin{boxleft}\bla{Division}
\end{boxleft}\begin{boxrightshaded}
\begin{align} 
\frac{z_1}{z_2} &=\frac{x_1+jy_1}{x_2+jy_2}\\
		&=\frac{x_1x_2+y_1y_2}{x_2^2+y_2^2}+j\frac{x_2y_1-x_1y_2}{x_2^2+y_2^2}\\
\frac{z_1}{z_2} &=\frac{r_1\left(\cos{\varphi_1}+j\sin{\varphi_1}\right)}{r_2\left(\cos{\varphi_2}+j\sin{\varphi_2}\right)}\\
	        &=\frac{r_1}{r_2}\left(\cos(\varphi_1-\varphi_2)+j\sin(\varphi_1-\varphi_2)\right)\\
\frac{z_1}{z_2} &=\frac{r_1e^{j\varphi_1}}{r_2e^{j\varphi_2}}\\
	        &=\frac{r_1}{r_2}e^{j(\varphi_1-\varphi_2)}
\end{align}\end{boxrightshaded}

\begin{boxleft}\bla{Potenzieren}
\end{boxleft}\begin{boxrightshaded}
\begin{align} 
z^n		&=\left(r_1\left(\cos{\varphi_1}+j\sin{\varphi_1}\right)\right)^n\\
		&=r_1^n\left(\cos(n\varphi_1)+j\sin(n\varphi_1)\right)\\
z^n		&=\left(r_1e^{j\varphi_1}\right)^n\\
		&=r_1^ne^{jn\varphi_1}
\end{align}\end{boxrightshaded}

\begin{boxleft}\bla{Wurzelziehen}
\destext{Es entsthen n Lösungen}\\
\destext{Für k muss nacheinander $0,1,\dots,n-1$ eingesetzt werden}
\end{boxleft}\begin{boxrightshaded}
\begin{align} 
\sqrt[n]{z}	&=\sqrt[n]{r_1\left(\cos\varphi_1+j\sin{\varphi_1}\right)}\\
\omega_k	&=\sqrt[n]{r_1}\left(\cos\left(\frac{\varphi_1+2k\pi}{n}\right)+j\sin\left(\frac{\varphi_1+2k\pi}{n}\right)\right)\\
\sqrt[n]{z}	&=\sqrt[n]{r_1e^{j\varphi_1}}\\
\omega_k	&=\sqrt[n]{r_1}e^{j\left(\frac{\varphi_1+2k\pi}{n}\right)}
\end{align}\end{boxrightshaded}