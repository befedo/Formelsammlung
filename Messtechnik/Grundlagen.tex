\section{Begriffe}
\begin{itemize}
  \item Messwert \(x_i\): gemessener Wert der Messgröße
  \item Wahrer Wert \(x_w\): existierender Wert der Messgröße
  \item Richtiger Wert \(x_r\): bekannter Wert mit vernachläßigbarer Differenz
  zum wahren Wert
  \item Messabweichung \(e\): Differenz zwischen gemessenem und wahrem Wert
  \item Systematische Messabweichung \(e_{sys}\): Bekannte systematische
  Messabweichung (korrigierbar)
  \item Messunsicherheit \(u\): Intervall um den Messwert in dem der wahre Wert
  mit einer bestimmten Wahrscheinlichkeit zu finden ist
\end{itemize}

\section{Messabweichung \texorpdfstring{$e$}{}}
\[e=x-x_w\]

\subsection{relative Messabweichung}
\[e_{rel} = \frac{e}{x_w} = \frac{x-x_w}{x_w} = \frac{x}{x_w} - 1\]

\begin{multicols}{2}
\subsubsection*{Korrekturfaktor \texorpdfstring{$K$}{}}
Bei bekannter systematischer Messabweichung.
\[K = -e_{sys}\]

\subsubsection*{Korrigierter Messwert \texorpdfstring{$x_{korr}$}{}}
\[x_{korr} = x + K\]
\end{multicols}

\subsection{Messabweichung \texorpdfstring{$e_y$}{}}
\[e_y = y-y_w = f\left( x_1 + e_{x_1}, x_2 + e_{x_2}, \cdots ,x_n +
e_{x_n}\right)\]

\begin{multicols}{2}
\[ e_y = \sum_{i = 1}^{n} \frac{\partial f}{\partial x_i} e_{x_i}\]
\vfill
\[ \Delta y = \sum_{i = 1}^{n} \frac{\partial f}{\partial x_i}
\Delta{x_i}\]
\vfill
\end{multicols}

\subsection{Fortpflanzung systematischer Messabweichungen}
\subsubsection*{Addition / Subtraktion}
\begin{multicols}{3}
\[y = x_1 \pm x_2\]
\vfill
\[\longrightarrow\]
\vfill
\[e_y = e_{x_1} \pm e_{x_2}\]
\vfill
\end{multicols}

\subsubsection*{Multiplikation}
\begin{multicols}{3}
\[y = x_1 \cdot x_2\]
\vfill
\[\longrightarrow\]
\vfill
\[e_y = x_2 \cdot e_{x_1} + x_1 \cdot
e_{x_2}\]
\end{multicols}

\[ 
e_{rel} = \frac{e_y}{y} = \frac{x_2 \cdot e_{x_1} + x_1 \cdot e_{x_2}}{x_1
\cdot x_2} = e_{rel, x_1} + e_{rel, x_2} 
\]

\subsubsection*{Division}
\begin{multicols}{3}
\[y = \frac{x_1}{x_2}\]
\vfill
\[\longrightarrow\]
\vfill
\[e_y = \frac{1}{x_2}e_{x_1} -
\frac{x_1}{{x_2}^2}e_{x_2}\]
\end{multicols}
\[
e_{rel} = \frac{e_y}{y} = \frac{\frac{1}{x_2}e_{x_1} -
\frac{x_1}{{x_2}^2}e_{x_2}}{x_1 \cdot {x_2}^{-1}} = e_{rel, x_1} - e_{rel, x_2}
\]

\section{Statistische Größen}
\begin{multicols}{2}
	\subsection*{Verteilungsfunktion}
	\[F\left(x\right) = prob\left(X \leq x \right) \]
		\vfill
	\subsection*{Verteilungsdichtefunktion}
	\[f\left(x\right) = \frac{d}{dx}F\left(x\right)\]
		\vfill
\end{multicols}

Es gillt:
\begin{align*}
	&F\left(x\right) = \int_{-\infty}^{x} f\left(t\right)dt\\
	&F\left(x \rightarrow \infty \right) = \int_{-\infty}^{\infty}
	f\left(t\right)dt = 1\\
	&prob\left( a < x \leq b \right) = F\left(b\right) - F\left(a\right) =
	\int_{a}^{b}f\left(x\right)dx
\end{align*}

\section{Erwartungswert, Varianz und Standardabweichung}
\begin{multicols}{2}
	\subsection{Erwartungswert \texorpdfstring{$\mu$}{}}
		\begin{align*}
			\mu &= \frac{1}{N} \sum_{i = 1}^{N} x_i\\
				&= \int_{-\infty}^{\infty} x \cdot f\left(x\right)dx\\ 
				&\text{nur für stetige Zufallsgrößen}
		\end{align*}
		\vfill
	\subsection{wahrer Wert \texorpdfstring{$X$}{}}
		\begin{align*}
			x_w &= \mu\\
				&\text{nach Korrektur} \\ 
				&\text{der systematischen Abweichung}
		\end{align*}
		\vspace{5mm}
		\vfill
\end{multicols}

\begin{multicols}{2}
	\subsection{Varianz \texorpdfstring{$\sigma^2$}{}}
		\begin{align*}
			\sigma^2 &= \frac{1}{N} \sum_{i = 1}^{N}\left( x_i - \mu \right)^2\\
					 &= \int_{-\infty}^{\infty}\left( x_i - \mu \right)^2 \cdot f\left( 
					x \right)dx
		\end{align*}
		\vfill
	\subsection{Standardabweichung}
		\[\sigma = \sqrt{\sigma^2}\]
		\vspace{5mm}
		\vfill
\end{multicols}

\section{Verteilungsfunktionen}
	\subsection{Normalverteilung}
	\begin{itemize}
	  \item Normal oder Gaußverteilung
	  \item gute Näherung bei unbekannter statistischer Verteilung
	  \item Werteverteilung: \begin{itemize}
	    \item 68,3\% aller Werte liegen in \( \mu \pm \sigma \)
	    \item 95,5\% aller Werte liegen in \( \mu \pm 2\sigma \)
	    \item 99,7\% aller Werte liegen in \( \mu \pm 3\sigma \) 
	  \end{itemize}
	\end{itemize}
	
	\begin{align*}
		&f\left( x \right) = \frac{1}{\sqrt{2 \pi} \sigma} e^{-\frac{1}{2}\left(
		\frac{x - \mu}{\sigma} \right)^2}\\
		\int_{-\infty}^{\infty} &f\left( x \right) = 1
	\end{align*}

	\subsection{Gleichverteilung}
	\begin{itemize}
	  \item auch Rechteckverteilung
	  \item alle vorkommenden Werte besitzen gleiche Wahrscheinlichkeit im
	  Intervall
	\end{itemize}
	
	\begin{align*}
		f\left( x \right) &= 
			\begin{cases}
				\frac{1}{2a} \quad \mu - a < x < \mu + a\\
				0 \quad \text{ sonst}
			\end{cases}\\
		\int_{-\infty}^{\infty}f\left(x\right) &= 1\\
		\sigma^2 &= \frac{1}{3}a^2
	\end{align*}
	
\newpage
\section{Stichprobe}
\begin{multicols}{2}
	\subsection{Mittelwert \texorpdfstring{$\overline{x}$}{}}
	Der Mittelwert ist ein Schätzwert für den Erwartungswert \(\mu\) und damit für
	den wahren Wert.

	\[\overline{x} = \frac{1}{n} \sum_{i = 1}^{n} x_i\]
	
	\subsection{empirische Varianz \texorpdfstring{$s^2$}{}}
	Die empirische Varianz ist ein Schätzwert für die eigentliche Varianz der
	Messreihe.

	\[s^2 = \frac{1}{n-1} \sum_{i = 1}^{n} \left(x_i - \overline{x}\right)^2\]
	
	\vfill	
\end{multicols}

\section{Vertrauensbereich für den Erwartungswert}
Endlich große Stichprobe liefert zufällige Differenz zwischem Schätzwert
\(\overline{x}\) und wahrem Wert \(\mu=x_w\).

\begin{alignat*}{3}
	\overline{x_g} = \frac{1}{m}\sum_{i = 1}^{m}\overline{x_i} && s_{g}^2 =
	\frac{1}{m}s_{i}^2 && s_g = \frac{1}{\sqrt{m}}s_i
\end{alignat*}

Vertrauensbereich:
\[
\overline{x} - \frac{t}{\sqrt{n}} s < \mu < \overline{x} + \frac{t}{\sqrt{n}} \quad
\text{mit} \quad t = t\left( n, \alpha \right)
\]

\subsection{Studentverteilung}
Gibt den \(t\) Faktor für Normalverteilungen an
\begin{itemize}
  \item [\(\alpha\)] Überschreitungswhrscheinlichkeit
  \item [\(1 - \alpha\)] Vertrauensniveau
\end{itemize} 

\begin{center}
\begin{tabular}{c|c|c|c}
\(1 - \alpha\) & 68,3\% & 95\% & 99,73\%\\
\hline
\(n = 2\) & 1,84 & 12,70 & 235,80\\
\(n = 3\) & 1,32 & 4,30 & 19,21\\
\(n = 4\) & 1,20 & 3,18 & 9,22\\
\(n = 5\) & 1,15 & 2,78 & 6,62\\
\(n = 6\) & 1,11 & 2,57 & 5,51\\
\(n = 10\) & 1,06 & 2,26 & 4,09\\
\(n = 20\) & 1,03 & 2,09 & 3,45\\
\(n = 50\) & 1,01 & 2,01 & 3,16\\
\(n \rightarrow \infty\) & 1,00 & 2,00 & 3,00\\
\end{tabular}
\end{center}

\section{Fortpflanzung zufälliger Abweichungen}
Bedingung: Messergebnis setzt sich aus mehreren Messgrößen \(x_i\) zusammen

\begin{multicols}{2}
	\subsection*{Erwartungswerte}
	\[
	\mu_n = \frac{1}{N}\sum_{i = 1}^{N}x_{n_i}
	\]
	\subsection*{Varianzen}
	\[
	\sigma_{n}^{2} = \frac{1}{N}\sum_{i = 1}^{N} \left( x_{n_i} - \mu_n
	\right)^{2}
	\]
	\vfill
\end{multicols}

\subsection*{Worst-Case-Kombination}
Maximale Abweichung des Ergebnisses vom Mittelwert.
\begin{align*}
	y &= f\left( x_1, x_2, \cdots, x_n  \right)\\
	\left| \Delta y \right| &= \sum_{i = 1}^{n} \left| \frac{\partial f}{\partial
	x_i} \Delta x_i \right|
\end{align*}

\subsection*{statistische Kombination der Varianzen}
Gaußsches Fehlerfortpflanzungsgesetz\ldots

\begin{align*}
	y &= f\left( x_1, x_2, \cdots, x_n  \right)\\
	\sigma_{y}^{2} &= \sum_{k = 1}^{n} \left( \left( \frac{\partial f}{\partial
	x_k} \biggr|_{\left( \mu_1, \mu_2, \cdots, \mu_n \right)} \right)^2
	\sigma_{k}^{2} \right)
	\\
	\sigma_{y}^{2} &= 
	\left( \frac{\partial f}{\partial x_1} \biggr|_{\mu_1} \sigma_{1} \right)^2 + 
	\left( \frac{\partial f}{\partial x_2} \biggr|_{\mu_2} \sigma_{2} \right)^2 + 
	\left( \frac{\partial f}{\partial x_3} \biggr|_{\mu_3} \sigma_{3} \right)^2 +
	\quad \cdots
\end{align*}

\ldots kann auf empirische Varianz übertragen werden.

\begin{align*}
	y &= f\left( \overline{x_1}, \overline{x_2}, \cdots, \overline{x_n} \right)\\
	s_{y}^{2} &= \sum_{k = 1}^{n} \left( \left( \frac{\partial f}{\partial
	\overline{x_k}} \biggr|_{\left( \mu_1, \mu_2, \cdots, \mu_n \right)} \right)^2
	s_{k}^{2} \right)
\end{align*}

\newpage
\section{Fortpflanzung von Messunsicherheiten}
Worst Case Abschätzung und Gaußsches Fortpflanzungsgesetz lassen sich auf die
Messunsicherheiten übertragen.

\begin{multicols}{2}
	Worst Case Abschätzung der Unsicherheit
	\[
	u_y = \sum_{i = 1}^{n} \left| \frac{\partial f}{\partial x_i} \right| u_{x_i}
	\]
	Statistische Fortpflanzung der Unsicherheit
	\[
	u_{y}^{2} = \sum_{i = 1}^{n} \left( \frac{\partial f}{\partial x_i} \right)^2
	u_{x_i}^{2}
	\]
\end{multicols}