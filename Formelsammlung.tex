\documentclass[a5paper]{report}
%\documentclass[a5paper,10pt]{scrartcl}

\usepackage[utf8]{inputenc}
\usepackage{geometry}
\usepackage{german}
\usepackage[pdftex]{graphicx}
\usepackage[fleqn]{amsmath}
\usepackage{amssymb}
\usepackage{amstext}
\usepackage{amsfonts}
\usepackage{mathrsfs}
\usepackage{hyperref}
\usepackage{color}
\usepackage{tabularx}
\usepackage{multicol}

\title{Formelsammlung - ET/TI}
\author{Marc Ludwig}
\date{\today}

\pdfinfo{%
  /Title    (Formelsammlung - ET/TI)
  /Author   (Marc Ludwig)
  /Creator  (Marc Ludwig)
  /Producer (Marc Ludwig)
  /Keywords (Hilfe; Formeln; Hirn leer;...)
}

\hypersetup{linktocpage=true, colorlinks=false}	%Ganzseitiganzeigen %pdfpagemode=FullScreen,
\geometry{left=15mm,right=5mm, top=15mm, bottom=15mm}
\pagestyle{headings}
\hyphenation{words}
\allowdisplaybreaks

%Definitionen
\newcommand{\N}{\mathbb{N}}	%Natürliche Zahlen
\newcommand{\R}{\mathbb{R}}	%Reelle Zahlen
\newcommand{\C}{\mathbb{C}}	%Komplexe Zahlen


%%%%%%%%%%%%%%%Matzes Anpassungen

%Abstand zwischen Formeln
\setlength{\abovedisplayskip}{3mm}
\setlength{\belowdisplayskip}{3mm}

\newenvironment{merkbox}{\begin{minipage}{\linewidth}}{\end{minipage}}

%Einheitenbeschreibung
\newcommand{\des}[3][1]{$\left[#2\right]=\si{#1}$: {\small\textcolor{gray}{ #3}}}
\newcommand{\destext}[1]{{\small\textcolor{gray}{ #1}}}

\newcommand{\hebox}[1]{#1}
\newcommand{\heboxc}[1]{#1}

%Differential
\newcommand*\diff{\mathop{}\!\mathrm{d}}
\newcommand*\grad{\mathop{}\!\mathrm{grad}}
%%%%%%%%%%%%%%%%%%%%%%%%%%%%%%%%%%


\begin{document}
\maketitle

%INHALTSVERZEICHNIS
	\tableofcontents

		%TEIL
	\part{Mathematik}

	
		%KAPITEL
		\chapter{Algebra}
		
				%1. Abschnitt
	\section{Rechenregeln fuer Potenzen}
				
			
			\begin{align*}
				a^m \cdot a^n &= a^{m+n}	& \frac{a^m}{a^n} &= a^{m-n}	\\ \\
				\left( a^m \right)^n = \left( a^n \right)^m &= a^{m \cdot n} & a^n \cdot b^n &= \left( a \cdot b \right)^n	\\ \\
				\frac{a^n}{b^n} &= \left( \frac{a}{b} \right)^n	\\ \\
				\text{(fuer a} > \text{0) } a^b &= e^{b \cdot \ln a}
			\end{align*}
	
	%2.Abschnitt
	\vspace{10mm}
	\section{Zusammenhang zwischen Wurzeln und Potenzen}
		
	
			
				\boxed{
				\text{Im Folgenden wird vorausgesetzt, dass alle Potenzen und Wurzeln existieren.}
				} 
			
			
			\begin{align*}
				\sqrt[n]{a} &= a^{\frac{1}{n}} & \sqrt[n]{a^m} &= a^{\frac{m}{n}} & \left(\sqrt[n]{a}\right)^m &= a^{ \frac{m}{n}}
			\end{align*}
	
	%3.Abschnitt
	\newpage
	\section{Potenzen und Logarithmen}
	
	
			\text{Schreibweise: }
			x\(=\log_a \left(b\right) \text{ mit } a > 0, a \neq 1 \text{ und } b > 0 \text{.}\)
			\newline
			\text{Es gillt: }
			\(\log_a \left(1\right) = 0, \text{ } \log_a \left( a \right) = 1\)
			\text{.}
				
	%Unterpunkt
	\vspace{10mm}
	\subsection{Der natuerliche Logarithmus}
	
	
			%Hier muss der Limes noch angepasst werden, so dass der BEreich unter dem Ausdruck steht!
			\begin{flushleft}
			\text{Der Logarithmus zur Basis } \(e\) \text{ mit } \(e = \lim\limits_{n\to\infty} {\left(1+\frac{1}{n}\right)^n} = 2,71828...\)
			\begin{align*}
				\log_e \left(b\right) &= \ln \left(b\right) & \ln \left( \frac{1}{e} \right) = -1 ; \text{ da } e^{-1} = \frac{1}{e}
			\end{align*}
			
			\boxed{\text{Man beachte: } \text{x}^a = e^{\ln \left(\text{x}\right) \cdot a}}
			\end{flushleft}
	
	%Unterpunkt
	\vspace{10mm}
	\subsection{Rechnen mit Logarithmen}
	
	
			
			
			%Tabelle anlegen
			\begin{table}[h]
						
			\begin{tabular}{|l|l|}
				
				\hline
					\text{Es gillt:}
				&	%Dient als Ueberschrifft
					\text{Weitere Beziehungen:}
				\\
				\hline
					%Beginnt in der ersten Spalte, erste Zeile
   				\(\log_a \left({u \cdot v}\right) = \log_a \left(u\right) + \log_a \left(v\right)\)
				&	%In die naechste Spalte springen
					\( \log_a \left( \sqrt[n]{u} \right) = \frac{1}{n} \log_a \left( u \right)\)
				\\%Zurueck in die erste Spalte, zweite Zeile
					\(\log_a \left( \frac{u}{v} \right) = \log_a \left(u\right) - \log_a \left(v\right)\)
				&	%%In die naechste Spalte springen						
 					\(a^{\log_a \left(u\right)} = \log_a \left(a^u\right) = u\)
				\\%Zurueck in die erste Spalte, dritte Zeile
					\(\log_a \left(u^p\right) = p \cdot \log_a \left(u\right)\)
				&	%%In die naechste Spalte springen
					\(\log_a\left(u\right) = \frac{\log_c \left( u \right)}{\log_c \left( a \right)}\)
				\\
				%Unterstreicht die Tabelle
					\hline
								
			\end{tabular}
			\end{table}
			
	%4.Abschnitt
	\vspace{10mm}
	\section{Der Binomische Lehrsatz}
	
			
		Die Potenzen eines Binoms a+b lassen sich nach dem Binomischen Lehrsatz 
		\newline 
		wie folgt entwickeln \( \left(n \in \N^* \right)\):
		\vspace{5mm}
		\newline
		\( \left( a + b \right)^n = a^n + \binom{n}{1} a^{n-1} \cdot b^1 + \binom{n}{2} a^{n-2} \cdot b^2 + \binom{n}{3} a^{n-3} \cdot b^3 + 
		\ldots + \binom{n}{n-1} a^{1} \cdot b^{n-1} + b^n \)
		\vspace{5mm}
		\newline
		\text{Die Koeffizienten \( \binom{n}{k} \) heißen Binominalkoeffizienten, ihr Bildungsgesetz lautet:}
		\vspace{5mm}
		\newline
		\( \binom{n}{k} = \frac{n \left( n - 1 \right) \left( n - 2 \right) \ldots \left[ n - \left( k - 1 \right) \right]}{k!} = \frac{n!}{k! \left( n - k \right) !} \)
	
	\vspace{10mm}
	\subsubsection{Einige Eigenschaften der Binominalkoeffizienten}
		
		
			\begin{align*}
				\binom{n}{0} &= \binom{n}{n} = 1 & \binom{n}{k} &= 0 \text{ fuer k} > \text{n} & \binom{n}{1} &= \binom{n}{n-1} = n
				\\
				\binom{n}{k} &= \binom{n}{n-k} & \binom{n}{k} &+ \binom{n}{k+1} = \binom{n+1}{k+1}
			\end{align*}
			
	\vspace{10mm}
	\section{Sinus, Kosinus, Tangens und Kotangens}
	
	%Unterpunkt
	\vspace{10mm}
	\subsection{Beziehungen zwischen Sinus, Kosinus, Tangens und Kotangens}
	
	
		\begin{align*}
			&\sin^2 \left( \alpha \right) + \cos^2 \left( \alpha \right) = 1 & &\tan \left( \alpha \right) \cdot \cot \left( \alpha \right) = 1	
			\\
			&\tan \left( \alpha \right) = \frac{\sin \left( \alpha \right)}{\cos \left( \alpha \right)} & &\cot \left( \alpha \right) = \frac{\cos \left( \alpha \right)}{\sin \left( \alpha \right)} 
			\\
			&1 + \tan^2 \left( \alpha \right) = \frac{1}{\cos^2 \left( \alpha \right)} & &1 + \cot^2 \left( \alpha \right) = \frac{1}{\sin^2 \left( \alpha \right)}
		\end{align*}
	
	%Unterpunkt	
	\vspace{10mm}
	\subsection{Additionstheoreme}
	
	
		\begin{align*}
			\sin \left( \alpha \pm \beta \right) &= \sin \left( \alpha \right) \cos \left( \beta \right) \pm \cos \left( \alpha \right) \sin \left( \beta \right)
			\\ 
			\cos \left( \alpha \pm \beta \right) &= \cos \left( \alpha \right) \cos \left( \beta \right) \mp \sin \left( \alpha \right) \sin \left( \beta \right)
			\\
			\tan \left( \alpha \pm \beta \right) &= \frac{\tan \left( \alpha \right) \pm \tan \left( \beta \right)}{1 \mp \tan \left( \alpha \right) \tan \left( \beta \right)}
		\end{align*}
		
	%Unterpunkt
	\vspace{10mm}
	\subsection{Funktionen des doppelten und halben Winkels}
			
			
			\begin{align*}
				\sin \left( 2 \alpha \right) &= 2 \sin \left( \alpha \right) \cos \left( \alpha \right) 
				\\
				\cos \left( 2 \alpha \right) &= \cos^2 \left( \alpha \right) - \sin^2 \left( \alpha \right) = 2 \cos^2 \left( \alpha \right) -1 = 1 - 2 \sin^2 \left( \alpha \right)
				\\
				\tan \left( 2 \alpha \right) &= \frac{2 \tan \left( \alpha \right)}{1 - \tan^2 \left( \alpha \right)}
				\\
				\sin^2 \left( \frac{\alpha}{2} \right) &= \frac{1}{2} \left( 1 - \cos \left( \alpha \right) \right)
				\\
				\cos^2 \left( \frac{\alpha}{2} \right) &= \frac{1}{2} \left( 1 + \cos \left( \alpha \right) \right)
				\\
				\tan^2 \left( \frac{\alpha}{2} \right) &= \frac{1 - \cos \left( \alpha \right)}{1 + \cos \left( \alpha \right)}
			\end{align*}
			
	%Unterpunkt
	\vspace{10mm}
	\subsection{Umformungen}
	
	%Unterunterpunkt :)
	\vspace{10mm}
	\subsubsection{Summe oder Differenz in ein Produkt}
	
	
			\begin{flushleft}
				\(\sin \left( \alpha \right) + \sin \left( \beta \right) = 2 \sin \left( \frac{\alpha + \beta}{2}\right) \cos \left( \frac{\alpha - \beta}{2} \right)\)
				\\
				\(\sin \left( \alpha \right) - \sin \left( \beta \right) = 2 \cos \left( \frac{\alpha + \beta}{2}\right) \sin \left( \frac{\alpha - \beta}{2} \right)\)
				\\
				\(\cos \left( \alpha \right) + \cos \left( \beta \right) = 2 \cos \left( \frac{\alpha + \beta}{2}\right) \cos \left( \frac{\alpha - \beta}{2} \right)\)
				\\
				\(\cos \left( \alpha \right) - \cos \left( \beta \right) = -2 \sin \left( \frac{\alpha + \beta}{2}\right) \sin \left( \frac{\alpha - \beta}{2} \right)\)
			\end{flushleft}
	
	%Unterunterpunkt :)
	\vspace{10mm}
	\subsubsection{Produkt in eine Summe oder Differenz}
	
	
			\begin{flushleft}
				\(2 \sin \left( \alpha \right) \sin \left( \beta \right) = \cos \left( \alpha - \beta \right) - \cos \left( \alpha + \beta \right)\)
				\\
				\(2 \cos \left( \alpha \right) \cos \left( \beta \right) = \cos \left( \alpha - \beta \right) + \cos \left( \alpha + \beta \right)\)
				\\
				\(2 \sin \left( \alpha \right) \cos \left( \beta \right) = \sin \left( \alpha - \beta \right) + \sin \left( \alpha + \beta \right)\)
			\end{flushleft}
	
	%5.Abschnitt		
	\vspace{10mm}	
	\section{Komplexe Zahlen}
			
			
			\text{Für die Menge aller komplexen Zahlen schreibt man:}
			\vspace{5mm}
			\\
			\fbox{\( \C = \left\{ z | z = a + bj, a \in \R \wedge b \in \R \right\} \)}
			\vspace{5mm}
			\\
			\text{a-Realteil \ \  b-Imaginaerteil \ \  j-imaginaere Einheit}
			
			
			%Tabelle
			\begin{table}[h]
			
				\begin{tabular}{|l|l|l|}
				\hline
 				\text{kartesiche Form} & \text{trigonometrische Form}  & \text{exponentialform}	 \\ 
				\hline
 				\( z = a + bj \) & \( z = \left| z \right| \left( \cos \varphi + j \cdot \sin \varphi \right) \)  & \( z = \left| z \right| \cdot e^{j 	\varphi} \)  \\ 
				\hline
 				\( z^* = \left( a + bj \right)^* = a-bj \) & \( z^* = \left| z \right| \left( \cos \varphi - j \cdot \sin \varphi \right) \) & \( z^* = \left| z \right| \cdot e^{-j \varphi} \) \\ 
				\hline
				\end{tabular}
			
			\end{table}
			
			\begin{flushleft}
				\text{\( \left| z \right| \) = Betrag von z}
				\\
				\text{\( \varphi \) = Argument (Winkel) von z}
				\\
				\text{\( z^* \) = Konjugiert komplexe Zahl}
			\end{flushleft}
			
		%Unterpunkt
		\vspace{10mm}
		\subsection{Umrechnungen zwischen den Darstellungsformen}
			
			\vspace{10mm}
			\subsubsection{Polarform \(\rightarrow\) Kartesiche Form}
			
			
				\( z = \left| z \right| \cdot e^{j \varphi} = \left| z \right| \left( \cos \varphi + j \cdot \sin \varphi \right) = 
				\underbrace{\left| z \right| \cdot \cos \varphi}_{a} + j \cdot \underbrace{\left| z \right| \cdot \sin \varphi}_{b} = a + bj \)	
			
			\vspace{10mm}
			\subsubsection{Kartesische Form \(\rightarrow\) Polarform}
			
			
				\( \left| z \right| = \sqrt{a^2 + b^2}\), \ \ \(\tan \varphi = \frac{b}{a} \)
				
		%Unterpunkt
		\vspace{10mm}
		\subsection{Rechnen mit Komplexen Zahlen}
		
			\vspace{10mm}
			\subsubsection{Multiplikation}
								
				
				\fbox{In kartesischer Form:}
								
				\begin{center}
				\(z_1 \cdot z_2 = \left( a_1 + j b_1 \right) \cdot \left( a_2 + j b_2 \right) 
												= \left( a_1 a_2 - b_1 b_2 \right) + j \cdot \left( a_1 b_2 + a_2 b_1 \right)\)
				\end{center}
								
				\fbox{In der Polarform:}
												
				\begin{align*}
					z_1 \cdot z_2 &= \left[ \left| z_1 \right| \left( \cos \varphi_1 + j \cdot \sin \varphi_1 \right) \right] \cdot 
													 \left[ \left| z_2 \right| \left( \cos \varphi_2 + j \cdot \sin \varphi_2 \right) \right]
					\\
					&= \left( \left| z_1 \right| \left| z_2 \right| \right) \cdot \left[ \cos \left( \varphi_1 + \varphi_2 \right) +  
					j \cdot \sin \left(	\varphi_1 + \varphi_2 \right) \right]
					\\
					&= \left( \left| z_1 \right| \cdot e^{j \varphi_1} \right) \cdot \left( \left| z_2 \right| \cdot e^{j \varphi_2} \right)
					= \left( \left| z_1 \right| \left| z_2 \right| \right) \cdot e^{j \left( \varphi_1 + \varphi_2 \right)}
				\end{align*}
			
			\vspace{10mm}	
			\subsubsection{Division}
			
				\fbox{In kartesischer Form}
				
					\begin{center}
						
					\end{center}
				
				\fbox{In der Polarform}
		
		\chapter{Lineare Algebra}

			Ein Test um das Skript auszuprobieren.

	%Physik als eigenes Dokument, mit eigenen *.tex Files
	\part{Physik}
	
		\chapter{Kinematik}
		\input{Physik/Algemein.tex}

		\chapter{Fluiddynamik}
		\begin{multicols}{2}
\begin{quote}
  Premature optimization\\is the root of all evil.\\- D. Knuth
\end{quote}
\vfill
\begin{quote}
 On the other hand,\\we cannot ignore efficiency.\\- Jon Bentley
\end{quote}
\vfill
\end{multicols}

\section{Ohne Reibung}

\begin{multicols}{3}
\subsubsection*{Statischer Druck}
\begin{align*}
p&=\frac{\diff F_N}{\diff A}
\end{align*}

\subsubsection*{Dynamischer Druck}
\begin{align*}
p&=\frac{1}{2}\rho v^2
\end{align*}

\subsubsection*{Schweredruck}
\begin{align*}
p&=\frac{\rho V g}{A}\\
&=h\rho g
\end{align*}
\end{multicols}

\begin{multicols}{2}
\subsubsection*{Volumenstrom}
\begin{align*}
\dot{V}&=v A\\
&=\iint_A \vec{v} \diff\vec{ A}\\
&=\frac{\diff V}{\diff t}\\
&=Q
\end{align*}

\subsubsection*{Massenstrom}
\begin{align*}
\dot{m}&=jA\\
&=\iint_A \vec{j} \diff\vec{A}\\
&=\frac{\diff m}{\diff t}
\end{align*}
\end{multicols}

\begin{multicols}{2}
\subsubsection*{Auftrieb}
\begin{align*}
\vec{F_A}&=-\rho_V \vec{g} V\\
&=-\frac{\rho_V}{\rho_M}\vec{F_G}
\end{align*}

\subsubsection*{Kontinuitätsgleichung}
\begin{alignat*}{2}
\left.\dot{m}\right|_1&=\left.\dot{m}\right|_2 & \quad \left.\dot{V}\right|_1&=\left.\dot{V}\right|_2\\
v_1A_1&=v_2A_2 & \rho_1&=\rho_2
\end{alignat*}
\end{multicols}

\begin{multicols}{2}{}
\subsubsection*{Kompressibilität}
\begin{align*}
\kappa&=\frac{\Delta V}{\Delta p V}
\end{align*}


\subsubsection*{Volumenausdehnungskoeffezient}
\begin{align*}
\frac{\Delta V}{V}&= \gamma \Delta T
\end{align*}


\subsubsection*{Barometrische Höhenformel}
\begin{align*}
p&=p_0 e^{-Ch}\\
C&=\frac{\rho_0 g}{p_0}
\end{align*}


\subsubsection*{Bernoulli Gleichung}
\begin{align*}
p+\frac{1}{2}\rho v^2+ \rho g h= \text{const}
\end{align*}
\end{multicols}


\section{Laminare Reibung}

\begin{multicols}{2}{}
\subsubsection*{Newtonsches Reibungsgesetz}
\begin{align*}
F_R&=\eta A \frac{\diff v}{\diff x}
\end{align*}


\subsubsection*{Laminare Strömung (Rohr)}
\begin{align*}
v(r)&=\frac{p}{4\eta l}\left(R^2-r^2\right)\\
p&=\frac{4\eta l}{R^2}v(0)\\
\dot{V}&=\frac{\pi R^4}{8\eta l}p
\end{align*}


\subsubsection*{Umströmung (Kugel)}
\begin{align*}
F_R=6\pi\eta r v
\end{align*}



\subsubsection*{Bernoulligleichung mit Reibung}
\begin{align*}
&p_1+\frac{1}{2}\rho v_1^2+ \rho g h_1 \\
=&p_2+\frac{1}{2}\rho v_2^2+ \rho g h_2+\Delta p
\end{align*}


\subsubsection*{Reynoldszahl}
\begin{align*}
Re&=\frac{L\rho v}{\eta}\\
Re&>Re_{krit}\\
&\text{Strömung wird Turbulent}
\end{align*}
\end{multicols}


		\chapter{Gravitation}
		 \begin{quote}
  The year is 787!\\A.D.?\\- Monty Python
 \end{quote}

\begin{multicols}{2}{}
\subsubsection{Gravitationskraft}
\begin{align*}
\vec{F}_{g,2}&=-G\frac{m_1m_2}{r_{12}^2}\vec{e}_r\\
\vec{F}_g&=\vec{E}_g\cdot m=\vec{g}m
\end{align*}
\hfill

\subsubsection{Gravitationspotential}
\begin{align*}
\phi&=-G\frac{M}{r}\\
\vec{E}_g&=\grad\phi
\end{align*}
\hfill
\end{multicols}

\begin{multicols}{2}{}
\subsubsection{Arbeit}
\begin{align*}
W_{12}&=-\int_{\vec{r}_1}^{\vec{r}_2}\vec{F}_g\circ\diff\vec{r}\\
&=GmM\left(\frac{1}{r_1}-\frac{1}{r_2}\right)
\end{align*}

\subsubsection{Planetenbahnen}
\begin{align*}
\left(\frac{a}{a_E}\right)^3=\left(\frac{T}{T_E}\right)^2
\end{align*}
\hfill
\end{multicols}


		\chapter{Elektrostatik}
		 \begin{quote}
  Don't interrupt me\\while I'm interrupting.\\- Winston S. Churchill
 \end{quote}

\begin{multicols}{2}{}
\subsubsection*{Ladung}
\index{Elektrostatik!Ladung}
\begin{align*}
Q&=n\cdot e_0\\
&=CU\\
&=\int i \diff t
\end{align*}

\subsubsection*{Punktladungen}
\index{Elektrostatik!Punktladungen}
\begin{align*}
\vec{E}(\vec{r})&=\sum_{i=1}^{N}\vec{E}_i{\vec{r}_i}
\end{align*}
\vspace{15mm}
\end{multicols}

\begin{multicols}{2}{}
\subsubsection*{COULOMB Gesetz}
\index{Elektrostatik!Coulomb Gesetz}
\begin{align*}
\vec{F}_{12}&=\frac{1}{4\pi\epsilon}\frac{Q_1Q_2}{r^2}\vec{r_12}\\
&=\vec{E}Q\\
\vec{E}&=\frac{1}{4\pi\epsilon}\frac{Q}{r^2}\vec{r}\\
&=-\grad\varphi\\
&=-\left(\frac{\partial \varphi}{\partial x}\vec{e}_x+\frac{\partial \varphi}{\partial y}\vec{e}_y+\frac{\partial \varphi}{\partial z}\vec{e}_z\right)
\end{align*}

\subsubsection*{Spannung}
\index{Elektrostatik!Spannung}
\begin{align*}
U_{AB}=&\frac{W_{AB}}{Q}\\
=&\int_A^B\vec{E}\circ\diff\vec{s}\\
=&\oint_s\vec{E}\circ\diff\vec{s}=0\\
=&\varphi_A-\varphi_B\\
=&-\int_\infty^A\vec{E}\circ\diff\vec{s}\\
&-\left(-\int_\infty^B\vec{E}\circ\diff\vec{s}\right)
\end{align*}
\vfill
\end{multicols}

\newpage
\begin{multicols}{2}{}
\subsubsection*{El- / Verschiebungsfluß}
\index{Elektrostatik!Fluß}
\begin{align*}
\psi&=\int_A\vec{E}\circ\diff\vec{A}\\
\psi&=\oint_A\vec{E}\circ\diff\vec{A}=\frac{Q}{\epsilon}\\
\end{align*}

\subsubsection*{Flußdichte}
\index{Elektrostatik!Flußdichte}
\begin{align*}
\vec{D}&=\frac{\diff Q}{\diff A}\vec{e}_A\\
\vec{D}&=\epsilon\vec{E}\\
Q&=\oint_AD\diff A
\end{align*}
\end{multicols}

\subsubsection*{Kapazität}
\index{Elektrostatik!Kapazität}
\begin{align*}
Q&=CU
\end{align*}

\begin{multicols}{2}{}
\subsubsection*{OHMsches Gesetz}
\index{Elektrostatik!Ohm'sches Gesetz}
\begin{align*}
I &=\oint_A\vec{j}\circ\diff\vec{A}\\
  &=\oint_A \kappa\vec{E}\circ\diff\vec{A}\\
  &=\underbrace{\kappa E\cdot 4\pi r^2}_{\text{Kugel}}
\end{align*}
\vspace{20mm}

\subsubsection*{Arbeit im elektrischen Feld}
\index{Elektrostatik!Arbeit im elektrischen Feld}
\begin{align*}
w&=\frac{1}{2}\vec{E}\circ\vec{D}\\
W&=\int_Vw\diff V\\
 &=-Q\int_A^B\vec{E}\circ\diff\vec{s}\\
 &=\int_U Q\diff U\\
 &= \int_U CU \diff U\\
 &=\frac{1}{2}CU^2
\end{align*}
\end{multicols}


	\part{Elektrotechnik}

		\chapter{Gleichstromtechnik}
		\section{Grundgrößen}

\textbf{Elementarladung}
\index{Elementarladung}
\begin{multicols}{2}{}
\begin{align*}
e\approx 1,6\cdot 10^{-19}C
\end{align*}
\hfill

\begin{align*}
\left[Q\right]&=1C=1As\\
Q&=n\cdot e
\end{align*}
\end{multicols}


\begin{multicols}{2}{}
\textbf{Strom}
\index{Strom}
\begin{align*}
\left[I\right]&=1A\\
i(t)&=\frac{\diff Q}{\diff t}
\end{align*}

\textbf{Stromdichte}
\index{Strom!-dichte}
\begin{align*}
\left[J\right]&=1\frac{A}{mm^2}\\
\vec{J}&=\frac{I}{\vec{A}}
\end{align*}
\end{multicols}


\begin{multicols}{2}{}
\textbf{Potential}
\index{Potential}
\begin{align*}
\left[\varphi\right]&=1V=1\frac{Nm}{As}=1\frac{kgm^2}{As^3}\\
\varphi&=\frac{W}{Q}
\end{align*}

\textbf{Spannung}
\index{Spannung}
\begin{align*}
\left[U\right]&=1V\\
U_{AB}&=\varphi_a-\varphi_b
\end{align*}
\hfill
\end{multicols}

\newpage
\textbf{Widerstand und Leitwert}
\index{Widerstand}
\index{Leitwert}
\begin{multicols}{2}{}
\begin{align*}
\left[R\right]&=1\Omega=1\frac{V}{A}\\
R&=\frac{U}{I}\\
&=\rho\frac{l}{A}=\frac{1}{\kappa}\frac{l}{A}
\end{align*}
\hfill

\begin{align*}
\left[G\right]&=1S=1\frac{A}{V}\\
G&=\frac{I}{U}\\
&=\frac{1}{R}\\
&=\kappa\frac{A}{l}=\frac{1}{\rho}\frac{A}{l}
\end{align*}
\end{multicols}


\textbf{Temperaturabhängigkeit}
\index{Widerstand!Temperaturabhängigkeit}
\begin{align*}
R_2=R_1\cdot\left(1+\alpha\left(\vartheta_2-\vartheta_1\right)+\beta\left(\vartheta_2-\vartheta_1\right)^2\right)
\end{align*}


\begin{multicols}{2}{}
\textbf{Leistung}
\begin{align*}
\left[P\right]&=1W=1VA\\
P&=u(t)\cdot i(t)
\end{align*}

\textbf{Leistung im Mittel}
\begin{align*}
P&=\frac{1}{T}\int_0^T u(t)\cdot i(t)\diff t 
\end{align*}
\hfill
\end{multicols}


\section{Lineare Quellen}


\begin{multicols}{2}{}
\textbf{Spannungsquelle}
\index{Spannungsquelle}
\begin{align*}
U&=U_q-R_i\cdot I\\
I_K&=\frac{U_q}{R_i}
\end{align*}

\textbf{Stromquelle}
\index{Stromquelle}
\begin{align*}
I&=I_q-\frac{U}{R_i}\\
U_l&=I_q\cdot R_i
\end{align*}
\end{multicols}


\section{Kirchhoffsche Gesetze}
\begin{multicols}{2}{}
\textbf{Knotenpunktsatz}
\index{Knotenpunktsatz}
\begin{align*}
\sum_{i=1}^n I_i=0
\end{align*}

\textbf{Maschensatz}
\index{Maschensatz}
\begin{align*}
\sum_{i=1}^n U_i=0
\end{align*}
\end{multicols}


		\chapter{Wechselstromtechnik}
		 \begin{quote}
  No rule is so general,\\which admits not some exception.\\- Robert Burton
 \end{quote}
 
\section{Definitionen}

\subsection{Periodische zeitabhängige Größen}
Allgemein \(x\left(t\right) \xrightarrow{}\) speziell \(u\left(t\right); i\left(t\right); q\left(t\right); \dots\) \\
es gillt \(x\left(t\right) = x\left(t + n \cdot T\right) ; \left(n \in \N^* \right) \)

\subsection{Wechselgrößen}
\index{Wechselgrößen}
Allgemein \(x_{\sim} \left(t\right)\); periodisch sich ändernde Größe, deren Gleichanteil bzw. 
zeitlich linearer Mittelwert gleich Null ist. \\ \vspace{0mm} \\
Nachweis: \[\int\limits_{t1}^{t1 + n \cdot T} x_{\sim} \left(t\right)dt = 0 \; ; \; \left(n \in \N^* \right) \;
; \; t1 \; \text{beliebiger Zeitwert}\]

\subsection{Mischgrößen}
\index{Mischgrößen}
Sind periodisch, Ihr Gleichanteil \(\overline{x}\) bzw. zeitlich linearer Mittelwert \\
jedoch ist ungleich Null. 
\begin{align*}
\text{Mischgröße} &= \text{Wechselgröße + Gleichanteil} \\
x\left(t\right) &= x_{\sim}\left(t\right) + \overline{x} \\
&= \text{gleichanteilbehaftete Wechselgröße}
\end{align*}

\section{Anteile und Formfaktoren}

\begin{multicols}{2}{}
\textbf{Gleichanteil}
\index{Anteile!Gleichanteil}
\index{Formfaktoren!Gleichanteil}
\[ \overline{x} = \frac{1}{n \cdot T} \cdot \int_{t_{1}}^{t_{1} + n \cdot T} x \left( t \right) dt \]

\textbf{Gleichrichtwert}
\index{Anteile!Gleichrichtwert}
\index{Formfaktoren!Gleichrichtwert}
\[ \left| \overline{x} \right| = \frac{1}{n \cdot T} \cdot 
\int_{t_{1}}^{t_{1} + n \cdot T} \left| x \right| \left( t \right) dt\]

\textbf{Effektivwert}
\index{Anteile!Effektivwert}
\index{Formfaktoren!Effektivwert}
\[ x_{eff} = X = \sqrt{ \frac{1}{n \cdot T} \cdot \int_{t_{1}}^{t_{1} + n \cdot T} x^2 \left( t \right) dt} \]

\textbf{Formfaktor}
\index{Anteile!Formfaktor}
\index{Formfaktoren!Formfaktoren}
\[F = \frac{x_{eff}}{\left|\overline{x}\right|} \\ 
x_{eff} = \left|\overline{x}\right| \cdot F \]

\textbf{crest - Faktor}
\index{Anteile!crest - Faktor}
\index{Formfaktoren!crest - Faktor}
\[ \sigma = \frac{\hat{x}}{x_{eff}} \]
\hfill
\end{multicols}

\begin{center}
\(n \in \N^* \rightarrow
t1 \; \text{beliebiger Zeitwert} \rightarrow
\left[|\overline{x} \right|] = \left[ x\left( t \right) \right] \) 
\end{center}

\section{Leistung und Leistungsfaktoren}
\begin{multicols}{2}{}
 
\textbf{Wirkleistung}
\index{Leistung!Wirkleistung}
\begin{align*}
P &= \frac{1}{n \cdot T} \int_{t_{1}}^{t_{1} + n \cdot T} P \left( t \right) dt \\
  &= \frac{1}{n \cdot T} \int_{t_{1}}^{t_{1} + n \cdot T} u \left( t \right) \cdot i \left( t \right) dt
\end{align*}

\textbf{Mittlere Leistung}
\index{Leistung!Mittlere Leistung}
\[\bar{p} \left(t\right) = P = \frac{1}{n \cdot T} \int_{t_{1}}^{t_{1} + n \cdot T} P \left( t \right) dt\]

\textbf{Scheinleistung}
\index{Leistung!Scheinleistung}
\[ S = u_{eff} \cdot i_{eff} = U \cdot I\]
\end{multicols}

\textbf{Leistungsfaktor}
\index{Leistung!Leistungssfaktor}
\begin{align*}
\lambda &= \frac{P}{S} \\
	&= \frac{\frac{1}{n \cdot T} \int_{t_{1}}^{t_{1} + n \cdot T} p\left( t \right) dt}
	   { u_{eff} \cdot i_{eff}} \\
	&=  \frac{ \int_{t_{1}}^{t_{1} + n \cdot T} u \left( t \right) \cdot i \left( t \right) dt}
	   {\sqrt{ \int_{t_{1}}^{t_{1} + n \cdot T} u^2 \left( t \right) dt} \cdot
	    \sqrt{ \int_{t_{1}}^{t_{1} + n \cdot T} i^2 \left( t \right) dt}}
\end{align*}

\newpage
\section{Sinusförmige Größen}
\begin{multicols}{2}{}
 \textbf{Sinusschwingung}
 \index{Sinus!Schwingung}
  \begin{align*}
   x\left(t\right) &= \hat{x} \sin\left( 2 \pi f + \varphi_x\right) \\
   x\left( \omega t \right) &= \hat{x} \sin\left( \omega t + \varphi_x\right)
  \end{align*}
  \begin{itemize}
   \item \(\hat{x} :\) Amplitude
   \item \(\varphi_x :\) Nullphasenwinkel
   \item \(\varphi_{x}>0 :\) Linksverschiebung der Kurve
  \end{itemize}

 \textbf{Kosinusschwingung}
 \index{Kosinus!Schwingung}
  \begin{align*}
   x\left(t\right) &= \hat{x} \cos\left( 2 \pi f + \varphi_x\right) \\
   x\left( \omega t \right) &= \hat{x} \cos\left( \omega t + \varphi_x\right)
  \end{align*}
  \begin{itemize}
   \item \(\hat{x} :\) Amplitude
   \item \(\varphi_x :\) Nullphasenwinkel
   \item \(\varphi_{x}>0 :\) Rechtssverschiebung der Kurve
  \end{itemize}
\end{multicols}

\textbf{Nullphasenzeit}
\index{Nullphasenzeit}
\[ t_{x} = -\frac{\varphi_x}{\omega} = -\varphi_x \cdot \frac{T}{2 \pi} \]

\textbf{Addition zweier Sinusgrößen gleicher Frequenz}
\index{Sinus!Addition}
\[\text{mit: } a = \hat{a} \sin \left( \omega t + \alpha \right) \wedge b = \hat{b} \sin \left( \omega t + \beta \right)\]

Resultierende Funktion:
\begin{align*}
 x &= a + b \\
   &= \hat{a} \sin \left( \omega t  + \alpha \right) + \hat{b} \sin \left( \omega t  + \beta \right) \\
   &= \hat{x} \sin \left( \omega t + \varphi \right)
\end{align*}

\begin{itemize}
 \item \(\hat{x} :\) resultierende Amplitude
 \item \(\varphi :\) Nullphasenwinkel
\end{itemize}

\begin{align*}
 \text{Wobei: } \hat{x} &= + \sqrt{\hat{a}^2 + \hat{b}^2 +2 \hat{a}\hat{b} \cos\left( \alpha - \beta \right)} \\
		\varphi &= \arctan \frac{\hat{a} \sin \alpha + \hat{b} \sin \beta}{\hat{a} \cos \alpha + \hat{b} \cos \beta } 
\end{align*}

\subsubsection{Vierquadrantenarkustangens}
\newcommand{\mc}[3]{\multicolumn{#1}{#2}{#3}}
\begin{center}
\begin{tabular}{|c|c|}
\mc{2}{c}{\( \varphi = \arctan\frac{ZP}{NP}\)}\\\hline
\text{2. Quadrant} \(ZP > 0 , NP < 0\) & \text{1. Quadrant} \(ZP > 0 , NP > 0\)\\\hline
\text{3. Quadrant} \(ZP < 0 , NP < 0\) & \text{4. Quadrant} \(ZP < 0 , NP > 0\)\\\hline
\end{tabular}
\end{center}

\subsubsection{Der rotierende Zeiger als rotierender Vektor}
\index{rotierender Zeiger}
\begin{align*}
 \text{Allgemein gillt: } \sin \left( \omega t + \varphi_{x} \right) &= \frac{GK}{HT} = \frac{b}{\hat{x}} \\
			  \cos \left( \omega t + \varphi_{x} \right) &= \frac{AK}{HT} = \frac{a}{\hat{x}} \\
			  b &= \hat{x}\sin\left(\omega t + \varphi_{x}\right) \\
			  a &= \hat{x}\cos\left(\omega t + \varphi_{x}\right) \\
\text{Als Einheitsvektor: } \vec{x} &= a \cdot \vec{i} + b \cdot \vec{j}
\end{align*}

\subsubsection{Zeigerspitzenendpunkt}
\begin{align*}
\underline{x} &= \text{ Zeigerspitzenendpunkt}\\
\underline{x} &= \underbrace{\hat{x}\cos\left(\omega t + \varphi_{x}\right)}_{Re \rightarrow Abszisse} + j \cdot
\underbrace{\hat{x}\sin\left(\omega t + \varphi_{x}\right)}_{Im \rightarrow Ordinate} \\
\underline{x} &= \hat{x} \cdot e^{j \left( \omega t + \varphi_x \right)} \\
\underline{x}_{eff} &= \text{ rotierender Effektivwertzeiger} \\
\underline{x}_{eff} &= \hat{x}_{eff} \cdot e^{j \left( \omega t + \varphi_x \right)} 
\end{align*}

\textbf{Wechsel zwischen Sinus und Kosinus}
\index{Sinus!zu Kosinus}
\index{Kosinus!zu Sinus}
\begin{align*}
\hat{x}\left(t\right)\cos\left(\omega t + \varphi_x\right) \equiv \hat{x}\left(t\right)\sin\left(\omega t + \varphi_x + \frac{\pi}{2}\right) \\
\hat{x}\left(t\right)\sin\left(\omega t + \varphi_x\right) \equiv \hat{x}\left(t\right)\cos\left(\omega t + \varphi_x - \frac{\pi}{2}\right)
\end{align*}

\newpage
\begin{sidewaysfigure}
\index{Transformation!Zeitbereich}
\index{Transformation!Bildbereich}
\vspace*{\fill}
\begin{large}
\newcommand{\mcb}[3]{\multicolumn{#1}{#2}{#3}}
\definecolor{tcA}{rgb}{0.627451,0.627451,0.643137}
\definecolor{tcB}{rgb}{0.764706,0.764706,0.764706}
\begin{center}
\begin{tabular}{|l|l|l|}\hline
% use packages: color,colortbl
Zeitbereich &  & komplexer Zeitbereich\\\hline
\(x = \hat{x}\sin\left(\omega t + \varphi_{x}\right)\) & \mcb{1}{>{\columncolor{tcA}}l}{\( \xrightarrow{Hintransformation 1} \)} & \( \underline{x} = \hat{x}\cos\left(\omega t + \varphi_{x}\right) + j \hat{x}\sin\left(\omega t + \varphi_{x}\right) \)\\\hline
\(x = \hat{x}\cos\left(\omega t + \varphi_{x}\right)\) & \mcb{1}{>{\columncolor{tcB}}l}{\( \xrightarrow{Hintransformation 2} \)} & \( \underline{x} = \hat{x}e^{j \left( \omega t + \varphi_x \right)} \)\\\hline
 &  & Berechnungen im komplexen Bereich\\\hline
\( y = Im\left\{y\right\} = \hat{y} \sin \left( \omega t + \varphi_y \right) \) & \mcb{1}{>{\columncolor{tcA}}l}{\( \xleftarrow{Ruecktransformation 1} \)} & \( \underline{y} = \hat{y} e^{j \left( \omega t + \varphi_y\right)} \)\\\hline
\( y = Re\left\{y\right\} = \hat{y} \cos \left( \omega t + \varphi_y \right) \) & \mcb{1}{>{\columncolor{tcB}}l}{\( \xleftarrow{Ruecktransformation 2} \)} & \( \underline{y} = \hat{y} \cos \left( \omega t + \varphi_{y}\right) + j \hat{y} \sin \left( \omega t + \varphi_{y} \right) \)\\\hline
\end{tabular}
\end{center}

\begin{itemize}
 \item[HT1] erfordert die Ergänzung eines gleichwertigen reellen Kosinusterms mit dem ursprünglichen Sinusterm als Imaginärteil
 \item[HT2] erfordert die Ergänzung eines gleichwertigen imaginären Sinusterms mit dem ursprünglichen Kosinusterm als Realteil
 \item[RT1] entnahme des Imaginärteils
 \item[RT2] entnahme des Realteils
\end{itemize}
\end{large}
\vspace*{\fill}
\end{sidewaysfigure}


\begin{alignat*}{3}
&\text{Merke:} &\quad\quad& \frac{1}{j}=-j &\quad\quad& j=e^{j\frac{\pi}{2}}
\end{alignat*}

\textbf{Differentiation und Integration von Sinusgrößen}
\index{Sinus!Differentiation}
\index{Kosinus!Differentiation}
\definecolor{tcA}{rgb}{0.627451,0.627451,0.643137}
\begin{center}
\begin{tabular}{|r|l|}\hline
% use packages: color,colortbl
\rowcolor{tcA}
Zeitbereich & Zeigerbereich\\\hline
\(\begin{array}{c c}
x\left(t\right) = \hat{x} \sin\left( \omega t + \varphi_{x} \right) \xrightarrow{HT_{1}} \\
x\left(t\right) = \hat{x} \cos\left( \omega t + \varphi_{x} \right) \xrightarrow{HT_{2}}
\end{array}\)
 & 
\( \underline{x} = \hat{x} e^{j \left( \omega t + \varphi_{x} \right)}\)
\\\hline
\(\frac{d^{n} x \left( t \right)}{{dt}^n} \xrightarrow{HT_{1/2}}\) 
& \(\frac{d^{n} \underline{x} \left( t \right)}{{dt}^{n}} = {\left( j \omega \right)}^{n} \underline{x}\)
\\\hline
\end{tabular}
\end{center}

\definecolor{tcA}{rgb}{0.627451,0.627451,0.643137}
\begin{center}
\begin{tabular}{|r|l|}\hline
% use packages: color,colortbl
\rowcolor{tcA}
Zeitbereich & Zeigerbereich\\\hline
\(\begin{array}{c c}
x\left(t\right) = \hat{x} \sin\left( \omega t + \varphi_{x} \right) \xrightarrow{HT_{1}} \\
x\left(t\right) = \hat{x} \cos\left( \omega t + \varphi_{x} \right) \xrightarrow{HT_{2}}
\end{array}\)
 & 
\( \underline{x} = \hat{x} e^{j \left( \omega t + \varphi_{x} \right)}\)
\\\hline
\(\idotsint x \left( t \right) dt^n \xrightarrow{HT_{1/2}}\) 
& \(\idotsint \underline{x} \left( t \right) dt = \frac{1}{\left( j \omega \right)^n} \underline{x}\)
\\\hline
\end{tabular}
\end{center}

\textbf{R, L und C im kompl. Zeigerbereich}

\newcommand{\mcc}[3]{\multicolumn{#1}{#2}{#3}}
\definecolor{tcA}{rgb}{0.627451,0.627451,0.643137}
\begin{center}
\begin{tabular}{|l|l|}\hline
% use packages: color,colortbl
\mcc{1}{>{\columncolor{tcA}}l}{Ohmscher Widerstand}
\index{Zeigerbereich!R}
& 
\(\begin{array}{ll}
\hat{U} = R \hat{I}&
\hat{I} = \frac{\hat{U}}{R}
\end{array}\)
\\\hline
\mcc{1}{>{\columncolor{tcA}}l}{Induktivität}
\index{Zeigerbereich!L}
&
\(\begin{array}{ll}
\hat{U} = \omega L \hat{I}&
\hat{I} = \frac{\hat{U}}{\omega L}
\end{array}\)
\\\hline
\mcc{1}{>{\columncolor{tcA}}l}{Kapazität}
\index{Zeigerbereich!C}
& 
\(\begin{array}{ll}
\hat{U} = \frac{\hat{I}}{\omega C}&
\hat{I} = \omega C \hat{U}
\end{array}\)
\\\hline
\end{tabular}
\end{center}

\textbf{Widerstands und Leitwertoperator}

\definecolor{tcA}{rgb}{0.627451,0.627451,0.643137}
\begin{center}
\index{Operatoren!Widerstandsop.}
\index{Operatoren!Leitwertsop.}
\begin{tabular}{|l|l|}\hline
\rowcolor{tcA}
\(\underline{Z}\) komplexer Widerstand / Impedanz & 
\(\underline{Y}\) komplexer Leitwert / Admitanz
\\\hline
\(\underline{Z}=\frac{\underline{u}}{\underline{i}}=\frac{{\hat{U}}}{{\hat{I}}}\cdot e^{j\left(\varphi_u - \varphi_i\right)}\) & 
\(\underline{Y}=\frac{1}{\underline{Z}}=\frac{\hat{I}}{\hat{U}}\cdot e^{j\left(\varphi_i - \varphi_u\right)}\)
\\\hline
\(\left|\underline{Z}\right|=Z=\frac{\hat{U}}{\hat{I}}=\frac{U}{I}\) & 
\(\left|\underline{Y}\right|=Y=\frac{1}{\underline{Z}}=\frac{I}{U}\)
\\\hline
mit \(\varphi_u - \varphi_i = \varphi_Z\) &
mit \(\varphi_i - \varphi_u = -\varphi_Z = \gamma_Y\)
\\\hline
\end{tabular}
\end{center}

\emph{Widerstand}
\[\underline{Z} = R \wedge \underline{Y} = 1/R\]
\emph{Kapazität}
\[\underline{Z} = \frac{1}{j \omega C} = \frac{1}{\omega C} e^{-j \frac{\pi}{2}} \wedge \underline{Y} = j \omega C = \omega C e^{j \frac{\pi}{2}}\]
\emph{Induktivität}
\[\underline{Z} = j \omega L = \omega L e^{j \frac{\pi}{2}} \wedge \underline{Y} = \frac{1}{j \omega L} = \frac{1}{\omega L} e^{-j \frac{\pi}{2}}\]

\newpage
\textbf{Resultierende Operatoren}

\begin{multicols}{2}{}
\subsubsection*{Reihenschaltung}
\index{Operatoren!Reihenschaltung}
\[\underline{Z}_{ges} = \sum \limits_{i=1}^{n} \underline{Z}_i\]
\subsubsection*{Parallelschaltung}
\index{Operatoren!Parallelschaltung}
\[\underline{Y}_{ges} = \sum \limits_{i=1}^{n} \underline{Y}_i\]
\end{multicols}

\begin{multicols}{2}{}
\subsubsection*{Spannungsteiler}
\index{Operatoren!Spannungsteiler}
\[\frac{\underline{u}_1}{\underline{u}_2} = \frac{\underline{Z}_1 + \underline{Z}_2}{\underline{Z}_2}\]
\subsubsection*{Stromteiler}
\index{Operatoren!Stromteiler}
\[\frac{\underline{i}_1}{\underline{i}_2} = \frac{\underline{Y}_1}{\underline{Y}_2}\]
\end{multicols}

\textbf{Anteile am komplexen Widerstand (Impedanz)}
\[\underline{Z} = \operatorname{Re}\{\underline{Z}\} + j \cdot \operatorname{Im}\{\underline{Z}\} = R + jX = \left|\underline{Z}\right| \cdot e^{j\varphi}\]
\begin{align*}
\text{mit }\varphi &= \varphi_u - \varphi_i \text{ Phasenwinkel; } R = \text{Wirkwiderstand; } \\ X &= \text{Blindwiderstand; } \left|\underline{Z}\right| = \text{Scheinwiderstand }
\end{align*}
\begin{alignat*}{3}
R&=R &\quad\quad& L=\frac{X}{\omega} \text{ mit } X>0 &\quad\quad& C=-\frac{1}{\omega X} \text{ mit } X<0
\end{alignat*}

\textbf{Anteile am komplexen Leiwert (Admitanz)}
\[\underline{Y} = \operatorname{Re}\{\underline{Y}\} + j \cdot \operatorname{Im}\{\underline{Y}\} = G + jB = \left|\underline{Y}\right| \cdot e^{j\gamma}\]
\begin{align*}
\text{mit }\gamma &= \varphi_i - \varphi_u \text{ Phasenwinkel; } G = \text{Wirkleitwert; } \\ B &= \text{Blindleitwert; } \left|\underline{Y}\right| = \text{Scheinleitwert }
\end{align*}
\begin{alignat*}{3}
R&=\frac{1}{G} &\quad\quad& C=\frac{B}{\omega} \text{ mit } B>0 &\quad\quad& L=-\frac{1}{\omega B} \text{ mit } B<0
\end{alignat*}

\textbf{komplexer Widerstand / komplexer Leitwert}
\begin{align*}
\underline{Y} = G + jB &= \frac{1}{\underline{Z}} = \frac{1}{Z} \cdot e^{-j\varphi} \\
	      &= \frac{1}{\sqrt{R^2 + X^2}} \cdot e^{-j\arctan\frac{X}{R}} \\
	      &= \frac{1}{R + jX} = \frac{R-jX}{R^2 + X^2} = \underbrace{\frac{R}{R^2 + X^2}}_G \underbrace{-j\frac{X}{R^2 + X^2}}_B
\end{align*}
\begin{align*}
\underline{Z} = R + jX &= \frac{1}{\underline{Y}} = \frac{1}{Y} \cdot e^{-j\gamma} \\
	      &= \frac{1}{\sqrt{G^2 + B^2}} \cdot e^{-j\arctan\frac{B}{G}} \\
	      &= \frac{1}{G + jB} = \frac{G-jB}{G^2 + B^2} = \underbrace{\frac{G}{G^2 + B^2}}_R \underbrace{-j\frac{B}{G^2 + B^2}}_X
\end{align*}

\textbf{Momentanleistung / Augenblicksleistung}
\index{Leistung!Momentanleistung}
\begin{align*}
P\left(t\right) &= \underbrace{UI \cos \varphi}_{\text{zeitlich konstant}} - \underbrace{UI \cos \left(2 \omega t + \varphi_u + \varphi_i\right)}_{\text{mit doppelter Frequenz schwingend}} \\
		&= UI \cos \varphi - UI \cos \left(2 \omega t + 2\varphi_u - \varphi\right) \\ \\
		&\text{mit } \varphi = \varphi_u - \varphi_i \rightarrow \varphi_i = \varphi_u - \varphi
\end{align*}

\textbf{Blindleistung}
\index{Leistung!Blindleistung}
\emph{Ermittlung des Blindleistungsanteils aus der Momentanleistung}
\begin{align*}
P\left(t\right) &= \underbrace{UI\cos\varphi}_{\text{Wirkleistung}}\underbrace{-UI\sin\varphi\cdot\sin\left(2\omega t + 2\varphi_u\right)}_{\text{Blindleistung}} \\
P_{ges}\left(t\right) &= P_{wirk}\left(t\right) + P_{blind}\left(t\right)
\end{align*}

\vspace{0mm}

\[u\left(t\right)\cdot i\left(t\right)
\begin{cases}
>0\text{ Energie zum Verbraucher} \\
<0\text{ Energie zum Erzeuger}
\end{cases}\]

\textbf{Mittlere Leistung / Wirkleistung}
\index{Leistung!Wirkleistung}
\[P = \overline{P}\left(t\right) = \frac{1}{n \cdot T} \int_{t_1}^{t_1 + n \cdot T} u\left(t\right) \cdot i\left(t\right) dt = UI\cos\varphi\]

\textbf{Definition von Blind- und Scheinleistung}
\index{Leistung!Blindleistung!Definition}
\index{Leistung!Scheinleistung!Definition}
\[Q = UI\sin\varphi \quad \left[Q\right] = \text{var} \quad \text{mit}
\begin{cases}
Q>0\text{ induktive Blindleistung } Q_{ind} \\ 
Q<0\text{ kapazitive Blindleistung } Q_{kap}
\end{cases}
\]
\[S = u_{eff} \cdot i_{eff} = U \cdot I \quad \left[S\right] = VA\]

\textbf{Beziehungen zwischen Wirk- Blind- und Scheinleistung}

\begin{center}
\boxed{P=UI\cdot\cos\varphi \quad\quad Q=UI\cdot\sin\varphi \quad\quad S=UI} 
\end{center}

\begin{multicols}{2}
\[\tan\varphi=\frac{Q}{P}=\frac{\sin\varphi}{\cos\varphi}\]

\begin{align*}
P&=\sqrt{S^2-Q^2} \\
 &=S\cdot\cos\varphi \\
 &=\frac{Q}{\tan\varphi}
\end{align*}

\begin{align*}
S&=\sqrt{P^2+Q^2} \\
 &=\frac{Q}{\sin\varphi} \\
 &=\frac{P}{\cos\varphi}
\end{align*}

\[P^2+Q^2=U^2\cdot I^2=S^2\]

\begin{align*}
\text{Leistungsfaktor} \\
\lambda = \frac{P}{S} = \cos\varphi
\end{align*}

\[Q =
\begin{cases}
>0 \rightarrow Q_{ind} = \sqrt{S^2-P^2} \\
<0 \rightarrow Q_{kap} = -\sqrt{S^2-P^2}
\end{cases}\]

\[Q=S\cdot\sin\varphi=P\cdot\tan\varphi\]

\begin{align*}
\varphi&=\arctan\frac{Q}{P} \\ 
       &=\arcsin\frac{Q}{S} \\
       &=\arccos\frac{P}{S}
\end{align*}

\end{multicols}

\textbf{Die komplexe Leistung}
\index{Leistung!komplexe L.}
\begin{alignat*}{2}
\underline{S}&=\underline{U}\cdot\underline{I}^* &\quad\quad\quad& ^*\text{ - konjugiert Komplex} \\
	     &=U \cdot I \cdot e^{j\left(\varphi_u - \varphi_i\right)} \\
	     &=S \cdot e^{j\varphi} \\
	     &=\underbrace{S \cdot \cos\varphi}_P + j \cdot \underbrace{S \cdot \sin\varphi}_Q \\
	     &=P + j Q &\quad\quad& \left[\underline{S}\right] = VA \quad \left[P\right] = W \quad \left[Q\right] = var
\end{alignat*}

\subsubsection*{Zusammenhang mit dem komplexen Leitwert / Widerstand}
\begin{alignat*}{3}
\underline{S} &= I^2 \cdot \underline{Z} &\quad\quad\quad  P &= I^2 \cdot R = U^2 \cdot G &\quad\quad\quad Q &= I^2 \cdot X = -U^2 \cdot B
\end{alignat*}

\end{document}
