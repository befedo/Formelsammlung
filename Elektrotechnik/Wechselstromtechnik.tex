\section{Definitionen}

\subsection{periodische zeitabhängige Größen}
Allgemein \(x\left(t\right) \xrightarrow{}\) speziell \(u\left(t\right); i\left(t\right); q\left(t\right); \dots\) \\
es gillt \(x\left(t\right) = x\left(t + n \cdot T\right) ; \left(n \in \N^* \right) \)

\subsection{Wechselgrößen}
Allgemein \(x_{\sim} \left(t\right)\); periodisch sich ändernde Größe, deren Gleichanteil bzw. 
zeitlich linearer Mittelwert gleich Null ist. \\ \vspace{0mm} \\
Nachweis: \[\int\limits_{t1}^{t1 + n \cdot T} x_{\sim} \left(t\right)dt = 0 \; ; \; \left(n \in \N^* \right) \;
; \; t1 \; \text{beliebiger Zeitwert}\]

\subsection{Mischgrößen}
Sind periodisch, Ihr Gleichanteil \(\overline{x}\) bzw. zeitlich linearer Mittelwert \\
jedoch ist ungleich Null. 
\begin{align*}
\text{Mischgröße} &= \text{Wechselgröße + Gleichanteil} \\
x\left(t\right) &= x_{\sim}\left(t\right) + \overline{x} \\
&= \text{gleichanteilbehaftete Wechselgröße}
\end{align*}

\newpage
\section{Anteile und Formfaktoren}

\begin{multicols}{2}{}
\subsection{Gleichanteil}
\[ \overline{x} = \frac{1}{n \cdot T} \cdot \int_{t_{1}}^{t_{1} + n \cdot T} x \left( t \right) dt \]

\subsection{Gleichrichtwert}
\[ \left| \overline{x} \right| = \frac{1}{n \cdot T} \cdot 
\int_{t_{1}}^{t_{1} + n \cdot T} \left| x \right| \left( t \right) dt\]

\subsection{Effektivwert}
\[ x_{eff} = X = \sqrt{ \frac{1}{n \cdot T} \cdot \int_{t_{1}}^{t_{1} + n \cdot T} x^2 \left( t \right) dt} \]

\subsection{Formfaktor}
\[F = \frac{x_{eff}}{\left|\overline{x}\right|} \\ 
x_{eff} = \left|\overline{x}\right| \cdot F \]

\subsection{crest - Faktor}
\[ \sigma = \frac{\hat{x}}{x_{eff}} \]
\hfill
\end{multicols}

\(n \in \N^* \rightarrow
t1 \; \text{beliebiger Zeitwert} \rightarrow
\left[|\overline{x} \right|] = \left[ x\left( t \right) \right] \)


\section{Leistung und Leistungsfaktoren}
\begin{multicols}{2}{}
 
\subsection{Wirkleistung}
\begin{align*}
P &= \frac{1}{n \cdot T} \int_{t_{1}}^{t_{1} + n \cdot T} P \left( t \right) dt \\
  &= \frac{1}{n \cdot T} \int_{t_{1}}^{t_{1} + n \cdot T} u \left( t \right) \cdot i \left( t \right) dt
\end{align*}

\subsection{Mittlere Leistung}
\[\bar{p} \left(t\right) = P = \frac{1}{n \cdot T} \int_{t_{1}}^{t_{1} + n \cdot T} P \left( t \right) dt\]

\subsection{Scheinleistung}
\[ S = u_{eff} \cdot i_{eff} = U \cdot I\]
\end{multicols}

\subsection{Leistungsfaktor}
\begin{align*}
\lambda &= \frac{P}{S} \\
	&= \frac{\frac{1}{n \cdot T} \int_{t_{1}}^{t_{1} + n \cdot T} p\left( t \right) dt}
	   { u_{eff} \cdot i_{eff}} \\
	&=  \frac{ \int_{t_{1}}^{t_{1} + n \cdot T} u \left( t \right) \cdot i \left( t \right) dt}
	   {\sqrt{ \int_{t_{1}}^{t_{1} + n \cdot T} u^2 \left( t \right) dt} \cdot
	    \sqrt{ \int_{t_{1}}^{t_{1} + n \cdot T} i^2 \left( t \right) dt}}
\end{align*}

\newpage
\section{Sinusförmige Größen}
\begin{multicols}{2}{}
 \subsection{Sinusschwingung}
  \begin{align*}
   x\left(t\right) &= \hat{x} \sin\left( 2 \pi f + \varphi_x\right) \\
   x\left( \omega t \right) &= \hat{x} \sin\left( \omega t + \varphi_x\right)
  \end{align*}
  \begin{itemize}
   \item \(\hat{x} :\) Amplitude
   \item \(\varphi_x :\) Nullphasenwinkel
   \item \(\varphi_{x}>0 :\) Linksverschiebung der Kurve
  \end{itemize}

 \subsection{Kosinusschwingung}
  \begin{align*}
   x\left(t\right) &= \hat{x} \cos\left( 2 \pi f + \varphi_x\right) \\
   x\left( \omega t \right) &= \hat{x} \cos\left( \omega t + \varphi_x\right)
  \end{align*}
  \begin{itemize}
   \item \(\hat{x} :\) Amplitude
   \item \(\varphi_x :\) Nullphasenwinkel
   \item \(\varphi_{x}>0 :\) Rechtssverschiebung der Kurve
  \end{itemize}
\end{multicols}

\subsection{Nullphasenzeit}
\[ t_{x} = -\frac{\varphi_x}{\omega} = -\varphi_x \cdot \frac{T}{2 \pi} \]

\subsection{Addition zweier Sinusgrößen gleicher Frequenz}
\[\text{mit: } a = \hat{a} \sin \left( \omega t + \alpha \right) \wedge b = \hat{b} \sin \left( \omega t + \beta \right)\]

Resultierende Funktion:
\begin{align*}
 x &= a + b \\
   &= \hat{a} \sin \left( \omega t  + \alpha \right) + \hat{b} \sin \left( \omega t  + \beta \right) \\
   &= \hat{x} \sin \left( \omega t + \varphi \right)
\end{align*}

\begin{itemize}
 \item \(\hat{x} :\) resultierende Amplitude
 \item \(\varphi :\) Nullphasenwinkel
\end{itemize}

\begin{align*}
 \text{Wobei: } \hat{x} &= + \sqrt{\hat{a}^2 + \hat{b}^2 +2 \hat{a}\hat{b} \cos\left( \alpha - \beta \right)} \\
		\varphi &= \arctan \frac{\hat{a} \sin \alpha + \hat{b} \sin \beta}{\hat{a} \cos \alpha + \hat{b} \cos \beta } 
\end{align*}

\subsubsection*{Vierquadrantenarkustangens}
\newcommand{\mc}[3]{\multicolumn{#1}{#2}{#3}}
\begin{center}
\begin{tabular}{|c|c|}
\mc{2}{c}{\( \varphi = \arctan\frac{ZP}{NP}\)}\\\hline
\text{2. Quadrant} \(ZP > 0 , NP < 0\) & \text{1. Quadrant} \(ZP > 0 , NP > 0\)\\\hline
\text{3. Quadrant} \(ZP < 0 , NP < 0\) & \text{4. Quadrant} \(ZP < 0 , NP > 0\)\\\hline
\end{tabular}
\end{center}

\subsubsection*{Der rotierende Zeiger als rotierender Vektor}
\begin{align*}
 \text{Allgemein gillt: } \sin \left( \omega t + \varphi_{x} \right) &= \frac{GK}{HT} = \frac{b}{\hat{x}} \\
			  \cos \left( \omega t + \varphi_{x} \right) &= \frac{AK}{HT} = \frac{a}{\hat{x}} \\
			  b &= \hat{x}\sin\left(\omega t + \varphi_{x}\right) \\
			  a &= \hat{x}\cos\left(\omega t + \varphi_{x}\right) \\
\text{Als Einheitsvektor: } \vec{x} &= a \cdot \vec{i} + b \cdot \vec{j}
\end{align*}

\subsubsection*{Zeigerspitzenendpunkt}
\begin{align*}
\underline{x} &= \text{ Zeigerspitzenendpunkt}\\
\underline{x} &= \underbrace{\hat{x}\cos\left(\omega t + \varphi_{x}\right)}_{Re \rightarrow Abszisse} + j \cdot
\underbrace{\hat{x}\sin\left(\omega t + \varphi_{x}\right)}_{Im \rightarrow Ordinate} \\
\underline{x} &= \hat{x} \cdot e^{j \left( \omega t + \varphi_x \right)} \\
\underline{x}_{eff} &= \text{ rotierender Effektivwertzeiger} \\
\underline{x}_{eff} &= \hat{x}_{eff} \cdot e^{j \left( \omega t + \varphi_x \right)} 
\end{align*}

\subsection{Wechsel zwischen Sinus und Kosinus}
\begin{align*}
\hat{x}\left(t\right)\cos\left(\omega t + \varphi_x\right) \equiv \hat{x}\left(t\right)\sin\left(\omega t + \varphi_x + \frac{\pi}{2}\right) \\
\hat{x}\left(t\right)\sin\left(\omega t + \varphi_x\right) \equiv \hat{x}\left(t\right)\cos\left(\omega t + \varphi_x - \frac{\pi}{2}\right)
\end{align*}

\newpage
\begin{landscape}
\vspace*{\fill}
\begin{large}
\newcommand{\mcb}[3]{\multicolumn{#1}{#2}{#3}}
\definecolor{tcA}{rgb}{0.627451,0.627451,0.643137}
\definecolor{tcB}{rgb}{0.764706,0.764706,0.764706}
\begin{center}
\begin{tabular}{|l|l|l|}\hline
% use packages: color,colortbl
Zeitbereich &  & komplexer Zeitbereich\\\hline
\(x = \hat{x}\sin\left(\omega t + \varphi_{x}\right)\) & \mcb{1}{>{\columncolor{tcA}}l}{\( \xrightarrow{Hintransformation 1} \)} & \( \underline{x} = \hat{x}\cos\left(\omega t + \varphi_{x}\right) + j \hat{x}\sin\left(\omega t + \varphi_{x}\right) \)\\\hline
\(x = \hat{x}\cos\left(\omega t + \varphi_{x}\right)\) & \mcb{1}{>{\columncolor{tcB}}l}{\( \xrightarrow{Hintransformation 2} \)} & \( \underline{x} = \hat{x}e^{j \left( \omega t + \varphi_x \right)} \)\\\hline
 &  & Berechnungen im komplexen Bereich\\\hline
\( y = Im\left\{y\right\} = \hat{y} \sin \left( \omega t + \varphi_y \right) \) & \mcb{1}{>{\columncolor{tcA}}l}{\( \xleftarrow{Ruecktransformation 1} \)} & \( \underline{y} = \hat{y} e^{j \left( \omega t + \varphi_y\right)} \)\\\hline
\( y = Re\left\{y\right\} = \hat{y} \cos \left( \omega t + \varphi_y \right) \) & \mcb{1}{>{\columncolor{tcB}}l}{\( \xleftarrow{Ruecktransformation 2} \)} & \( \underline{y} = \hat{y} \cos \left( \omega t + \varphi_{y}\right) + j \hat{y} \sin \left( \omega t + \varphi_{y} \right) \)\\\hline
\end{tabular}
\end{center}

\begin{itemize}
 \item[HT1] erfordert die Ergänzung eines gleichwertigen reellen Kosinusterms mit dem ursprünglichen Sinusterm als Imaginärteil
 \item[HT2] erfordert die Ergänzung eines gleichwertigen imaginären Sinusterms mit dem ursprünglichen Kosinusterm als Realteil
 \item[RT1] entnahme des Imaginärteils
 \item[RT2] entnahme des Realteils
\end{itemize}
\end{large}
\vspace*{\fill}
\end{landscape}

\subsection{Differentiation von Sinusgrößen}

\emph{Merke:}
\begin{itemize}
 \item \( \frac{1}{j} = -j \)
 \item \( j = e^{j\frac{\pi}{2}} \)
\end{itemize}





