 \begin{quote}
  No rule is so general,\\which admits not some exception.\\- Robert Burton
 \end{quote}

\subsection*{Periodische zeitabhängige Größen}
Allgemein \(x\left(t\right) \xrightarrow{}\) speziell \(u\left(t\right); i\left(t\right); q\left(t\right); \dots\) \\
es gillt \(x\left(t\right) = x\left(t + n \cdot T\right) ; \left(n \in \N^* \right) \)

\subsection*{Wechselgrößen}
Allgemein \(x_{\sim} \left(t\right)\); periodisch sich ändernde Größe, deren Gleichanteil bzw. 
zeitlich linearer Mittelwert gleich Null ist. \\ \vspace{0mm} \\
Nachweis: \[\int\limits_{t1}^{t1 + n \cdot T} x_{\sim} \left(t\right)dt = 0 \; ; \; \left(n \in \N^* \right) \;
; \; t1 \; \text{beliebiger Zeitwert}\]

\subsection*{Mischgrößen}
Sind periodisch, Ihr Gleichanteil \(\overline{x}\) bzw. zeitlich linearer Mittelwert \\
jedoch ist ungleich Null. 
\begin{align*}
\text{Mischgröße} &= \text{Wechselgröße + Gleichanteil} \\
x\left(t\right) &= x_{\sim}\left(t\right) + \overline{x} \\
&= \text{gleichanteilbehaftete Wechselgröße}
\end{align*}

\section{Anteile und Formfaktoren}

\begin{multicols}{2}{}
\subsection*{Gleichanteil}
\[ \overline{x} = \frac{1}{n \cdot T} \cdot \int_{t_{1}}^{t_{1} + n \cdot T} x \left( t \right) dt \]

\subsection*{Gleichrichtwert}
\[ \left| \overline{x} \right| = \frac{1}{n \cdot T} \cdot 
\int_{t_{1}}^{t_{1} + n \cdot T} \left| x \right| \left( t \right) dt\]

\subsection*{Effektivwert}
\[ x_{eff} = X = \sqrt{ \frac{1}{n \cdot T} \cdot \int_{t_{1}}^{t_{1} + n \cdot T} x^2 \left( t \right) dt} \]

\subsection*{Formfaktor}
\[F = \frac{x_{eff}}{\left|\overline{x}\right|} \\ 
x_{eff} = \left|\overline{x}\right| \cdot F \]

\subsection*{crest - Faktor}
\[ \sigma = \frac{\hat{x}}{x_{eff}} \]
\hfill
\end{multicols}

\begin{center}
\(n \in \N^* \rightarrow
t1 \; \text{beliebiger Zeitwert} \rightarrow
\left[|\overline{x} \right|] = \left[ x\left( t \right) \right] \) 
\end{center}

\section{Leistung und Leistungsfaktoren}
\begin{multicols}{2}{}
 
\subsection*{Wirkleistung}
\begin{align*}
P &= \frac{1}{n \cdot T} \int_{t_{1}}^{t_{1} + n \cdot T} P \left( t \right) dt \\
  &= \frac{1}{n \cdot T} \int_{t_{1}}^{t_{1} + n \cdot T} u \left( t \right) \cdot i \left( t \right) dt
\end{align*}

\subsection*{Mittlere Leistung}
\[\bar{p} \left(t\right) = P = \frac{1}{n \cdot T} \int_{t_{1}}^{t_{1} + n \cdot T} P \left( t \right) dt\]

\subsection*{Scheinleistung}
\[ S = u_{eff} \cdot i_{eff} = U \cdot I\]
\end{multicols}

\subsection*{Leistungsfaktor}
\begin{align*}
\lambda &= \frac{P}{S} \\
	&= \frac{\frac{1}{n \cdot T} \int_{t_{1}}^{t_{1} + n \cdot T} p\left( t \right) dt}
	   { u_{eff} \cdot i_{eff}} \\
	&=  \frac{ \int_{t_{1}}^{t_{1} + n \cdot T} u \left( t \right) \cdot i \left( t \right) dt}
	   {\sqrt{ \int_{t_{1}}^{t_{1} + n \cdot T} u^2 \left( t \right) dt} \cdot
	    \sqrt{ \int_{t_{1}}^{t_{1} + n \cdot T} i^2 \left( t \right) dt}}
\end{align*}

\newpage
\section{Sinusförmige Größen}
\begin{multicols}{2}{}
 \subsection*{Sinusschwingung}
  \begin{align*}
   x\left(t\right) &= \hat{x} \sin\left( 2 \pi f + \varphi_x\right) \\
   x\left( \omega t \right) &= \hat{x} \sin\left( \omega t + \varphi_x\right)
  \end{align*}
  \begin{itemize}
   \item \(\hat{x} :\) Amplitude
   \item \(\varphi_x :\) Nullphasenwinkel
   \item \(\varphi_{x}>0 :\) Linksverschiebung der Kurve
  \end{itemize}

 \subsection*{Kosinusschwingung}
  \begin{align*}
   x\left(t\right) &= \hat{x} \cos\left( 2 \pi f + \varphi_x\right) \\
   x\left( \omega t \right) &= \hat{x} \cos\left( \omega t + \varphi_x\right)
  \end{align*}
  \begin{itemize}
   \item \(\hat{x} :\) Amplitude
   \item \(\varphi_x :\) Nullphasenwinkel
   \item \(\varphi_{x}>0 :\) Rechtssverschiebung der Kurve
  \end{itemize}
\end{multicols}

\subsection*{Nullphasenzeit}
\[ t_{x} = -\frac{\varphi_x}{\omega} = -\varphi_x \cdot \frac{T}{2 \pi} \]

\subsection*{Addition zweier Sinusgrößen gleicher Frequenz}
\[\text{mit: } a = \hat{a} \sin \left( \omega t + \alpha \right) \wedge b = \hat{b} \sin \left( \omega t + \beta \right)\]

Resultierende Funktion:
\begin{align*}
 x &= a + b \\
   &= \hat{a} \sin \left( \omega t  + \alpha \right) + \hat{b} \sin \left( \omega t  + \beta \right) \\
   &= \hat{x} \sin \left( \omega t + \varphi \right)
\end{align*}

\begin{itemize}
 \item \(\hat{x} :\) resultierende Amplitude
 \item \(\varphi :\) Nullphasenwinkel
\end{itemize}

\begin{align*}
 \text{Wobei: } \hat{x} &= + \sqrt{\hat{a}^2 + \hat{b}^2 +2 \hat{a}\hat{b} \cos\left( \alpha - \beta \right)} \\
		\varphi &= \arctan \frac{\hat{a} \sin \alpha + \hat{b} \sin \beta}{\hat{a} \cos \alpha + \hat{b} \cos \beta } 
\end{align*}

\subsubsection*{Vierquadrantenarkustangens}
\newcommand{\mc}[3]{\multicolumn{#1}{#2}{#3}}
\begin{center}
\begin{tabular}{|c|c|}
\mc{2}{c}{\( \varphi = \arctan\frac{ZP}{NP}\)}\\\hline
\text{2. Quadrant} \(ZP > 0 , NP < 0\) & \text{1. Quadrant} \(ZP > 0 , NP > 0\)\\\hline
\text{3. Quadrant} \(ZP < 0 , NP < 0\) & \text{4. Quadrant} \(ZP < 0 , NP > 0\)\\\hline
\end{tabular}
\end{center}

\subsubsection*{Der rotierende Zeiger als rotierender Vektor}
\begin{align*}
 \text{Allgemein gillt: } \sin \left( \omega t + \varphi_{x} \right) &= \frac{GK}{HT} = \frac{b}{\hat{x}} \\
			  \cos \left( \omega t + \varphi_{x} \right) &= \frac{AK}{HT} = \frac{a}{\hat{x}} \\
			  b &= \hat{x}\sin\left(\omega t + \varphi_{x}\right) \\
			  a &= \hat{x}\cos\left(\omega t + \varphi_{x}\right) \\
\text{Als Einheitsvektor: } \vec{x} &= a \cdot \vec{i} + b \cdot \vec{j}
\end{align*}

\subsubsection*{Zeigerspitzenendpunkt}
\begin{align*}
\underline{x} &= \text{ Zeigerspitzenendpunkt}\\
\underline{x} &= \underbrace{\hat{x}\cos\left(\omega t + \varphi_{x}\right)}_{Re \rightarrow Abszisse} + j \cdot
\underbrace{\hat{x}\sin\left(\omega t + \varphi_{x}\right)}_{Im \rightarrow Ordinate} \\
\underline{x} &= \hat{x} \cdot e^{j \left( \omega t + \varphi_x \right)} \\
\underline{x}_{eff} &= \text{ rotierender Effektivwertzeiger} \\
\underline{x}_{eff} &= \hat{x}_{eff} \cdot e^{j \left( \omega t + \varphi_x \right)} 
\end{align*}

\subsection*{Wechsel zwischen Sinus und Kosinus}
\begin{align*}
\hat{x}\left(t\right)\cos\left(\omega t + \varphi_x\right) \equiv \hat{x}\left(t\right)\sin\left(\omega t + \varphi_x + \frac{\pi}{2}\right) \\
\hat{x}\left(t\right)\sin\left(\omega t + \varphi_x\right) \equiv \hat{x}\left(t\right)\cos\left(\omega t + \varphi_x - \frac{\pi}{2}\right)
\end{align*}

\newpage
\begin{landscape}
\vspace*{\fill}
\begin{large}
\newcommand{\mcb}[3]{\multicolumn{#1}{#2}{#3}}
\definecolor{tcA}{rgb}{0.627451,0.627451,0.643137}
\definecolor{tcB}{rgb}{0.764706,0.764706,0.764706}
\begin{center}
\begin{tabular}{|l|l|l|}\hline
% use packages: color,colortbl
Zeitbereich &  & komplexer Zeitbereich\\\hline
\(x = \hat{x}\sin\left(\omega t + \varphi_{x}\right)\) & \mcb{1}{>{\columncolor{tcA}}l}{\( \xrightarrow{Hintransformation 1} \)} & \( \underline{x} = \hat{x}\cos\left(\omega t + \varphi_{x}\right) + j \hat{x}\sin\left(\omega t + \varphi_{x}\right) \)\\\hline
\(x = \hat{x}\cos\left(\omega t + \varphi_{x}\right)\) & \mcb{1}{>{\columncolor{tcB}}l}{\( \xrightarrow{Hintransformation 2} \)} & \( \underline{x} = \hat{x}e^{j \left( \omega t + \varphi_x \right)} \)\\\hline
 &  & Berechnungen im komplexen Bereich\\\hline
\( y = Im\left\{y\right\} = \hat{y} \sin \left( \omega t + \varphi_y \right) \) & \mcb{1}{>{\columncolor{tcA}}l}{\( \xleftarrow{Ruecktransformation 1} \)} & \( \underline{y} = \hat{y} e^{j \left( \omega t + \varphi_y\right)} \)\\\hline
\( y = Re\left\{y\right\} = \hat{y} \cos \left( \omega t + \varphi_y \right) \) & \mcb{1}{>{\columncolor{tcB}}l}{\( \xleftarrow{Ruecktransformation 2} \)} & \( \underline{y} = \hat{y} \cos \left( \omega t + \varphi_{y}\right) + j \hat{y} \sin \left( \omega t + \varphi_{y} \right) \)\\\hline
\end{tabular}
\end{center}

\begin{itemize}
 \item[HT1] erfordert die Ergänzung eines gleichwertigen reellen Kosinusterms mit dem ursprünglichen Sinusterm als Imaginärteil
 \item[HT2] erfordert die Ergänzung eines gleichwertigen imaginären Sinusterms mit dem ursprünglichen Kosinusterm als Realteil
 \item[RT1] entnahme des Imaginärteils
 \item[RT2] entnahme des Realteils
\end{itemize}
\end{large}
\vspace*{\fill}
\end{landscape}


\begin{alignat*}{3}
&\text{Merke:} &\quad\quad& \frac{1}{j}=-j &\quad\quad& j=e^{j\frac{\pi}{2}}
\end{alignat*}

\subsection*{Differentiation und Integration von Sinusgrößen}

\definecolor{tcA}{rgb}{0.627451,0.627451,0.643137}
\begin{center}
\begin{tabular}{|r|l|}\hline
% use packages: color,colortbl
\rowcolor{tcA}
Zeitbereich & Zeigerbereich\\\hline
\(\begin{array}{c c}
x\left(t\right) = \hat{x} \sin\left( \omega t + \varphi_{x} \right) \xrightarrow{HT_{1}} \\
x\left(t\right) = \hat{x} \cos\left( \omega t + \varphi_{x} \right) \xrightarrow{HT_{2}}
\end{array}\)
 & 
\( \underline{x} = \hat{x} e^{j \left( \omega t + \varphi_{x} \right)}\)
\\\hline
\(\frac{d^{n} x \left( t \right)}{{dt}^n} \xrightarrow{HT_{1/2}}\) 
& \(\frac{d^{n} \underline{x} \left( t \right)}{{dt}^{n}} = {\left( j \omega \right)}^{n} \underline{x}\)
\\\hline
\end{tabular}
\end{center}

\definecolor{tcA}{rgb}{0.627451,0.627451,0.643137}
\begin{center}
\begin{tabular}{|r|l|}\hline
% use packages: color,colortbl
\rowcolor{tcA}
Zeitbereich & Zeigerbereich\\\hline
\(\begin{array}{c c}
x\left(t\right) = \hat{x} \sin\left( \omega t + \varphi_{x} \right) \xrightarrow{HT_{1}} \\
x\left(t\right) = \hat{x} \cos\left( \omega t + \varphi_{x} \right) \xrightarrow{HT_{2}}
\end{array}\)
 & 
\( \underline{x} = \hat{x} e^{j \left( \omega t + \varphi_{x} \right)}\)
\\\hline
\(\idotsint x \left( t \right) dt^n \xrightarrow{HT_{1/2}}\) 
& \(\idotsint \underline{x} \left( t \right) dt = \frac{1}{\left( j \omega \right)^n} \underline{x}\)
\\\hline
\end{tabular}
\end{center}

\subsection*{R, L und C im kompl. Zeigerbereich}

\newcommand{\mcc}[3]{\multicolumn{#1}{#2}{#3}}
\definecolor{tcA}{rgb}{0.627451,0.627451,0.643137}
\begin{center}
\begin{tabular}{|l|l|}\hline
% use packages: color,colortbl
\mcc{1}{>{\columncolor{tcA}}l}{Ohmscher Widerstand} 
& 
\(\begin{array}{ll}
\hat{U} = R \hat{I}&
\hat{I} = \frac{\hat{U}}{R}
\end{array}\)
\\\hline
\mcc{1}{>{\columncolor{tcA}}l}{Induktivität} 
&
\(\begin{array}{ll}
\hat{U} = \omega L \hat{I}&
\hat{I} = \frac{\hat{U}}{\omega L}
\end{array}\)
\\\hline
\mcc{1}{>{\columncolor{tcA}}l}{Kapazität} 
& 
\(\begin{array}{ll}
\hat{U} = \frac{\hat{I}}{\omega C}&
\hat{I} = \omega C \hat{U}
\end{array}\)
\\\hline
\end{tabular}
\end{center}

\subsection*{Widerstands und Leitwertoperator}

\definecolor{tcA}{rgb}{0.627451,0.627451,0.643137}
\begin{center}
\begin{tabular}{|l|l|}\hline
\rowcolor{tcA}
\(\underline{Z}\) komplexer Widerstand / Impedanz & 
\(\underline{Y}\) komplexer Leitwert / Admitanz
\\\hline
\(\underline{Z}=\frac{\underline{u}}{\underline{i}}=\frac{{\hat{U}}}{{\hat{I}}}\cdot e^{j\left(\varphi_u - \varphi_i\right)}\) & 
\(\underline{Y}=\frac{1}{\underline{Z}}=\frac{\hat{I}}{\hat{U}}\cdot e^{j\left(\varphi_i - \varphi_u\right)}\)
\\\hline
\(\left|\underline{Z}\right|=Z=\frac{\hat{U}}{\hat{I}}=\frac{U}{I}\) & 
\(\left|\underline{Y}\right|=Y=\frac{1}{\underline{Z}}=\frac{I}{U}\)
\\\hline
mit \(\varphi_u - \varphi_i = \varphi_Z\) &
mit \(\varphi_i - \varphi_u = -\varphi_Z = \gamma_Y\)
\\\hline
\end{tabular}
\end{center}

\emph{Widerstand}
\[\underline{Z} = R \wedge \underline{Y} = 1/R\]
\emph{Kapazität}
\[\underline{Z} = \frac{1}{j \omega C} = \frac{1}{\omega C} e^{-j \frac{\pi}{2}} \wedge \underline{Y} = j \omega C = \omega C e^{j \frac{\pi}{2}}\]
\emph{Induktivität}
\[\underline{Z} = j \omega L = \omega L e^{j \frac{\pi}{2}} \wedge \underline{Y} = \frac{1}{j \omega L} = \frac{1}{\omega L} e^{-j \frac{\pi}{2}}\]

\newpage
\subsection*{Resultierende Operatoren}

\begin{multicols}{2}{}
\subsubsection*{Reihenschaltung}
\[\underline{Z}_{ges} = \sum \limits_{i=1}^{n} \underline{Z}_i\]
\subsubsection*{Parallelschaltung}
\[\underline{Y}_{ges} = \sum \limits_{i=1}^{n} \underline{Y}_i\]
\end{multicols}

\begin{multicols}{2}{}
\subsubsection*{Spannungsteiler}
\[\frac{\underline{u}_1}{\underline{u}_2} = \frac{\underline{Z}_1 + \underline{Z}_2}{\underline{Z}_2}\]
\subsubsection*{Stromteiler}
\[\frac{\underline{i}_1}{\underline{i}_2} = \frac{\underline{Y}_1}{\underline{Y}_2}\]
\end{multicols}

\subsection*{Anteile am komplexen Widerstand (Impedanz)}
\[\underline{Z} = \operatorname{Re}\{\underline{Z}\} + j \cdot \operatorname{Im}\{\underline{Z}\} = R + jX = \left|\underline{Z}\right| \cdot e^{j\varphi}\]
\begin{align*}
\text{mit }\varphi &= \varphi_u - \varphi_i \text{ Phasenwinkel; } R = \text{Wirkwiderstand; } \\ X &= \text{Blindwiderstand; } \left|\underline{Z}\right| = \text{Scheinwiderstand }
\end{align*}
\begin{alignat*}{3}
R&=R &\quad\quad& L=\frac{X}{\omega} \text{ mit } X>0 &\quad\quad& C=-\frac{1}{\omega X} \text{ mit } X<0
\end{alignat*}

\subsection*{Anteile am komplexen Leiwert (Admitanz)}
\[\underline{Y} = \operatorname{Re}\{\underline{Y}\} + j \cdot \operatorname{Im}\{\underline{Y}\} = G + jB = \left|\underline{Y}\right| \cdot e^{j\gamma}\]
\begin{align*}
\text{mit }\gamma &= \varphi_i - \varphi_u \text{ Phasenwinkel; } G = \text{Wirkleitwert; } \\ B &= \text{Blindleitwert; } \left|\underline{Y}\right| = \text{Scheinleitwert }
\end{align*}
\begin{alignat*}{3}
R&=\frac{1}{G} &\quad\quad& C=\frac{B}{\omega} \text{ mit } B>0 &\quad\quad& L=-\frac{1}{\omega B} \text{ mit } B<0
\end{alignat*}

\subsection*{komplexer Widerstand / komplexer Leitwert}
\begin{align*}
\underline{Y} = G + jB &= \frac{1}{\underline{Z}} = \frac{1}{Z} \cdot e^{-j\varphi} \\
	      &= \frac{1}{\sqrt{R^2 + X^2}} \cdot e^{-j\arctan\frac{X}{R}} \\
	      &= \frac{1}{R + jX} = \frac{R-jX}{R^2 + X^2} = \underbrace{\frac{R}{R^2 + X^2}}_G \underbrace{-j\frac{X}{R^2 + X^2}}_B
\end{align*}
\begin{align*}
\underline{Z} = R + jX &= \frac{1}{\underline{Y}} = \frac{1}{Y} \cdot e^{-j\gamma} \\
	      &= \frac{1}{\sqrt{G^2 + B^2}} \cdot e^{-j\arctan\frac{B}{G}} \\
	      &= \frac{1}{G + jB} = \frac{G-jB}{G^2 + B^2} = \underbrace{\frac{G}{G^2 + B^2}}_R \underbrace{-j\frac{B}{G^2 + B^2}}_X
\end{align*}

\subsection*{Momentanleistung / Augenblicksleistung}
\begin{align*}
P\left(t\right) &= \underbrace{UI \cos \varphi}_{\text{zeitlich konstant}} - \underbrace{UI \cos \left(2 \omega t + \varphi_u + \varphi_i\right)}_{\text{mit doppelter Frequenz schwingend}} \\
		&= UI \cos \varphi - UI \cos \left(2 \omega t + 2\varphi_u - \varphi\right) \\ \\
		&\text{mit } \varphi = \varphi_u - \varphi_i \rightarrow \varphi_i = \varphi_u - \varphi
\end{align*}

\subsection*{Blindleistung}
\emph{Ermittlung des Blindleistungsanteils aus der Momentanleistung}
\begin{align*}
P\left(t\right) &= \underbrace{UI\cos\varphi}_{\text{Wirkleistung}}\underbrace{-UI\sin\varphi\cdot\sin\left(2\omega t + 2\varphi_u\right)}_{\text{Blindleistung}} \\
P_{ges}\left(t\right) &= P_{wirk}\left(t\right) + P_{blind}\left(t\right)
\end{align*}

\vspace{0mm}

\[u\left(t\right)\cdot i\left(t\right)
\begin{cases}
>0\text{ Energie zum Verbraucher} \\
<0\text{ Energie zum Erzeuger}
\end{cases}\]

\subsection*{Mittlere Leistung / Wirkleistung}
\[P = \overline{P}\left(t\right) = \frac{1}{n \cdot T} \int_{t_1}^{t_1 + n \cdot T} u\left(t\right) \cdot i\left(t\right) dt = UI\cos\varphi\]

\subsection*{Definition von Blind- und Scheinleistung}
\[Q = UI\sin\varphi \quad \left[Q\right] = \text{var} \quad \text{mit}
\begin{cases}
Q>0\text{ induktive Blindleistung } Q_{ind} \\ 
Q<0\text{ kapazitive Blindleistung } Q_{kap}
\end{cases}
\]
\[S = u_{eff} \cdot i_{eff} = U \cdot I \quad \left[S\right] = VA\]

\subsection*{Beziehungen zwischen Wirk- Blind- und Scheinleistung}

\begin{center}
\boxed{P=UI\cdot\cos\varphi \quad\quad Q=UI\cdot\sin\varphi \quad\quad S=UI} 
\end{center}

\begin{multicols}{2}
\[\tan\varphi=\frac{Q}{P}=\frac{\sin\varphi}{\cos\varphi}\]

\begin{align*}
P&=\sqrt{S^2-Q^2} \\
 &=S\cdot\cos\varphi \\
 &=\frac{Q}{\tan\varphi}
\end{align*}

\begin{align*}
S&=\sqrt{P^2+Q^2} \\
 &=\frac{Q}{\sin\varphi} \\
 &=\frac{P}{\cos\varphi}
\end{align*}

\[P^2+Q^2=U^2\cdot I^2=S^2\]

\begin{align*}
\text{Leistungsfaktor} \\
\lambda = \frac{P}{S} = \cos\varphi
\end{align*}

\[Q =
\begin{cases}
>0 \rightarrow Q_{ind} = \sqrt{S^2-P^2} \\
<0 \rightarrow Q_{kap} = -\sqrt{S^2-P^2}
\end{cases}\]

\[Q=S\cdot\sin\varphi=P\cdot\tan\varphi\]

\begin{align*}
\varphi&=\arctan\frac{Q}{P} \\ 
       &=\arcsin\frac{Q}{S} \\
       &=\arccos\frac{P}{S}
\end{align*}

\end{multicols}

\subsection*{Die komplexe Leistung}
\begin{alignat*}{2}
\underline{S}&=\underline{U}\cdot\underline{I}^* &\quad\quad\quad& ^*\text{ - konjugiert Komplex} \\
	     &=U \cdot I \cdot e^{j\left(\varphi_u - \varphi_i\right)} \\
	     &=S \cdot e^{j\varphi} \\
	     &=\underbrace{S \cdot \cos\varphi}_P + j \cdot \underbrace{S \cdot \sin\varphi}_Q \\
	     &=P + j Q &\quad\quad& \left[\underline{S}\right] = VA \quad \left[P\right] = W \quad \left[Q\right] = var
\end{alignat*}

\subsubsection*{Zusammenhang mit dem komplexen Leitwert / Widerstand}
\begin{alignat*}{3}
\underline{S} &= I^2 \cdot \underline{Z} &\quad\quad\quad  P &= I^2 \cdot R = U^2 \cdot G &\quad\quad\quad Q &= I^2 \cdot X = -U^2 \cdot B
\end{alignat*}