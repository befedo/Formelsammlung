\section{Definitionen}

\subsection{periodische zeitabhängige Größen}
Allgemein \(x\left(t\right) \xrightarrow{}\) speziell \(u\left(t\right); i\left(t\right); q\left(t\right); \dots\) \\
es gillt \(x\left(t\right) = x\left(t + n \cdot T\right) ; \left(n \in \N^* \right) \)

\subsection{Wechselgrößen}
Allgemein \(x_{\sim} \left(t\right)\); periodisch sich ändernde Größe, deren Gleichanteil bzw. 
zeitlich linearer Mittelwert gleich Null ist. \\ \vspace{0mm} \\
Nachweis: \[\int_{t1}^{t1 + n \cdot T} x_{\sim} \left(t\right)dt = 0 \; ; \; \left(n \in \N^* \right) \;
; \; t1: \text{beliebiger Zeitwert}\]

\subsection{Mischgrößen}
Sind periodisch, Ihr Gleichanteil \(\overline{x}\) bzw. zeitlich linearer Mittelwert \\
jedoch ist ungleich Null. 
\begin{align*}
\text{Mischgröße} &= \text{Wechselgröße + Gleichanteil} \\
x\left(t\right) &= x_{\sim}\left(t\right) + \overline{x} \\
&= \text{gleichanteilbehaftete Wechselgröße}
\end{align*}

\subsection{Test}
