\section{Laplace / Fourier-Transformation}

\subsection*{Laplaceintegral}
\index{Laplaceintegral}
\begin{align*}
&X\left(p\right) \quad \Laplace \quad x\left(t\right)\\
&X\left(p\right) =\mathscr{L}\{x\left(t\right)\}=\int_{0}^{\infty}x\left(t\right)e^{-p\cdot t}\diff
t
\end{align*}

\subsection*{Fourierintegral}
\index{Fourierintegral}
\begin{align*}
&X\left(\omega\right)=\mathscr{F}\{x\left(t\right)\}=\int_{-\infty}^{\infty}x\left(t\right)e^{-j\omega t}\diff t\\
&X\left(\omega\right) \quad \Laplace \quad x\left(t\right)\\
&X\left(f\right) =\mathscr{F}\{x\left(t\right)\}=\int_{-\infty}^{\infty}x\left(t\right)e^{-j2\pi f
t}\diff t\\
&X\left(f\right) =\int_{-\infty}^\infty\left(x_{re}+jx_{im}\right)\cdot\left(\cos\left(2\pi f
t\right)-j\cdot\sin\left(2\pi f t\right)\right)\diff t\\
&X\left(f\right) \quad \Laplace \quad x\left(t\right)\\
&x\left(t\right) =\frac{1}{2\cdot\pi}\cdot\int_{-\infty}^{\infty}X\left(\omega\right)e^{j\omega
t}\diff \omega\\
&x\left(t\right) =\int_{-\infty}^{\infty}X\left(f \right)e^{j2\pi f t}\diff \omega
\end{align*}

\subsection*{Diskrete Fourier Transformation}
\index{Diskrete Fourier Transformation}
\begin{multicols}{2}
\textbf{1. Variante}
\begin{align*}
	X\left(l\right)&=\sum_{k=0}^{N-1}x\left(k\right)e^{-j2\pi\cdot l \cdot\frac{k}{N}}\\
	x\left(k\right)&=\frac{1}{N}\sum_{k=0}^{N-1}x\left(k\right)e^{j2\pi\cdot l \cdot\frac{k}{N}}
\end{align*}
\textbf{2. Variante}
\begin{align*}
	X\left(l\right)&=\frac{1}{N}\sum_{k=0}^{N-1}x\left(k\right)e^{-j2\pi\cdot l \cdot\frac{k}{N}}\\
	x\left(k\right)&=\sum_{k=0}^{N-1}x\left(k\right)e^{j2\pi\cdot l \cdot\frac{k}{N}}
\end{align*}
\textbf{3. Variante}
\begin{align*}
	X\left(l\right)&=\frac{1}{\sqrt{N}}\sum_{k=0}^{N-1}x\left(k\right)e^{-j2\pi\cdot l
	\cdot\frac{k}{N}}\\
	x\left(k\right)&=\frac{1}{\sqrt{N}}\sum_{k=0}^{N-1}x\left(k\right)e^{j2\pi\cdot l
	\cdot\frac{k}{N}}
\end{align*}
\vfill
\end{multicols}

\subsection*{DFT als Matrizen-Multiplikation}
\index{Diskrete Fourier Transformation!Matrizen-Multiplikation}
\begin{align*}
\left[X\left(l\right)\right]&=\frac{1}{\alpha}\cdot\left[F_{l,k}^N\right]\cdot\left[x\left(k\right)\right]&t&\Rightarrow
f\\
\left[x\left(k\right)\right]&=\frac{1}{\alpha}\cdot\left[f_{k,l}^N\right]\cdot\left[X\left(l\right)\right]&f&\Rightarrow
t\\
\left[f_{k,l}^N\right]&=\frac{1}{\alpha}\cdot\left[F_{l,k}^N\right]^*
\end{align*}

\(\alpha \) ist je nach Art der Transformationsvariante \(1, N \text{ oder } \sqrt{N}\) und
eigentlich schon in der Transformationsvorschrifft enthalten.

\[
F_{l,k}^N=e^{-j2\pi \cdot l\cdot \frac{k}{N}} = \cos\left(2 \pi l \frac{k}{N}\right)-j\sin\left(2
\pi l \frac{k}{N}\right)
\]

\newpage
\textbf{Additionssatz}
\index{Additionssatz}
\begin{align*}
x\left(t\right) &=x_1\left(t\right)+x_2\left(t\right)+\dots \quad \laplace \quad
X\left(f\right) &=X_1\left(f\right)+X_2\left(f\right)+\dots
\end{align*}

\textbf{Linearität}
\index{Linearität}
\begin{align*}
x\left(t\right) &=C\cdot x_1\left(t\right)\quad \laplace \quad X\left(f\right) =C\cdot
X_1\left(f\right)
\end{align*}

\textbf{Verschiebungssatz}
\index{Verschiebungssatz}
\begin{align*}
x\left(t\right) &=x_1\left(t-t_0\right)\quad \laplace \quad X\left(f\right) =X_1\left(f\right)\cdot
e^{-j2\pi f\cdot t_0}
\end{align*}

\textbf{Ähnlichkeitssatz}
\index{Aehnlichkeitssatz}
\begin{align*}
x\left(t\right) =x_1\left(a\cdot t\right)\quad &\laplace \quad X\left(f\right)
=\frac{1}{\left|a\right|}X_1\left(\frac{f}{a}\right)\\
x\left(t\right) =\frac{1}{\left|b\right|}x_1\left(\frac{t}{b}\right) \quad &\laplace \quad
X\left(f\right) =X_1\left(b\cdot f\right)
\end{align*}

\textbf{Differentiationssatz}
\index{Differentiationssatz}
\begin{align*}
x\left(t\right) =\frac{\diff x_1\left(t\right)}{\diff t} \quad &\laplace \quad X\left(f\right)
=j2\pi f\cdot X\left(f\right)\\
x\left(t\right) =\frac{\diff^K x_1\left(t\right)}{\diff t^K} \quad &\laplace \quad X\left(f\right)
=j^K\left(2\pi f\right)^K\cdot X\left(f\right)\\
x\left(t\right) =\frac{\diff^K x_1\left(t\right)}{\diff t^K} \quad &\laplace \quad
X\left(\omega\right) =j^K\left(\omega\right)^K\cdot X\left(\omega\right)
\end{align*}

\textbf{Integrationssatz}
\index{Integrationssatz}
\begin{align*}
x\left(t\right) =\int_\infty^tx_1\left(\tau\right)\diff\tau \quad &\laplace \quad X\left(f\right)
=\frac{1}{j2\pi f}\cdot X_1\left(f\right)+\frac{1}{2}X_1\left(f=0\right)\delta\left(f\right)\\
x\left(t\right) =\int_\infty^tx_1\left(\tau\right)\diff\tau \quad &\laplace \quad
X\left(\omega\right) =\frac{1}{j \omega}\cdot X_1\left(\omega\right)+\pi\cdot X_1\left(\omega=0\right)\delta\left(\omega\right)
\end{align*}

\textbf{Integrationssatz im Frequenzbereich}
\begin{align*}
x\left(t\right) =\frac{1}{-j2\pi t}\cdot x_1\left(t\right)+\frac{1}{2}x_1\left(t=0\right)\delta\left(t\right) \quad \laplace \quad X\left(f\right) =\int_{-\infty}^{f}X_1\left(\varphi\right)\diff \varphi\\
x\left(t\right) =\frac{1}{-j t}\cdot x_1\left(t\right)+\pi\cdot x_1\left(t=0\right)\delta\left(t\right) \quad \laplace \quad X\left(\omega\right) =\int_{-\infty}^{\omega}X_1\left(\varphi\right)\diff \varphi
\end{align*}

\textbf{Vertauschungssatz}
\index{Vertauschungssatz}
\begin{align*}
x\left(t\right) =x_1\left(t\right) \quad &\laplace \quad X\left(f\right) =X_1\left(f\right)\\
x\left(t\right) =X_1\left(t\right) \quad &\laplace \quad X\left(f\right) =x_1\left(-f\right)\\
x\left(t\right) =x_1\left(t\right) \quad &\laplace \quad X\left(\omega\right)
=X_1\left(\omega\right)\\
x\left(t\right) =X_1\left(t\right) \quad &\laplace \quad X\left(\omega\right) =2\pi\cdot
x_1\left(-\omega\right)
\end{align*}

\subsection*{Faltung}
\index{Faltung}
\begin{align*}
&x\left(t\right) =x_1\left(t\right) \quad \laplace \quad X\left(f\right) =\int_{-\infty}^\infty
X_1\left(\varphi\right)\cdot X_2\left(f-\varphi\right)\diff \varphi\\
&x\left(t\right) =x_1\left(t\right) \quad \laplace \quad X\left(\omega\right)
=\frac{1}{2\pi}\int_{-\infty}^\infty X_1\left(\varphi\right)\cdot X_2\left(\omega-\varphi\right)\diff \varphi\\
&x\left(t\right) =\int_{-\infty}^\infty x_1\left(\tau\right)\cdot x_2\left(t-\tau\right)\diff \tau 
\quad \laplace \quad X\left(f\right) =x_1\left(f\right)\cdot x_2\left(f\right)
\end{align*}

\subsection*{Delta-Impulsfläche}
\index{Delta Impulsfläche}
\begin{align*}
\text{\scha}_p\left(t\right) &=\sum_{k=-\infty}^{\infty}\delta\left(t-kt_p\right) s^{-1} \quad
\laplace \quad\text{\SCHA}_A\left(f\right)
=f_a \sum_{m=-\infty}^{\infty}\delta\left(f-mf_a\right) {Hz}^{-1}\\
f_a &=\frac{1}{t_p}\\
\text{\scha}_a\left(t\right) &=t_a\sum_{k=-\infty}^{\infty}\delta\left(t-kt_a\right)
s^{-1} \quad \laplace \quad\text{\SCHA}_P\left(f\right)
=\sum_{m=-\infty}^{\infty}\delta\left(f-mf_p\right) {Hz}^{-1}\\
f_p &=\frac{1}{t_a}
\end{align*}

\newpage
\subsection*{Periodifizierung}
\index{Signale!Periodifizierung}
\begin{align*}
x\left(t\right) =x_T\left(t\right)\ast\text{\scha}_p\left(t\right) \quad \laplace \quad X\left(f\right) =X_T\left(f\right)\cdot \text{\SCHA}_A\left(f\right)
\end{align*}

\subsection*{Abgetastete Funktionen}
\index{Signale!Abtastung}
\begin{align*}
x_\delta\left(t\right) =x\left(t\right)\cdot\text{\scha}_a\left(t\right) \quad &\laplace \quad
X_\delta\left(f\right) =X\left(f\right)\ast\text{\SCHA}_p\left(f\right)\\
x_\delta\left(t\right) =\sum_{k=-\infty}^{\infty}x\left(kt_a\right)\cdot t_a\cdot
\delta\left(t-kt_a\right) \quad &\laplace \quad X_\delta =\sum_{m=-\infty}^{\infty}X\left(f-mf_p\right)
\end{align*}

\subsection*{Abgetastete und Periodifizierte Funktionen}
\index{Signale!Abtastung + Periodifizierung}
\begin{align*}
x_{\delta p}\left(t\right)
&=\left(x_T\left(t\right)\ast\text{\scha}_p\left(t\right)\right)\cdot\text{\scha}_a\left(t\right) \\
x_{\delta p}\left(t\right) &=\sum_{m=-\infty}^\infty\sum_{k=-\infty}^\infty
x_T\left(kt_a-mt_p\right)\cdot t_a\cdot \delta\left(t-kt_a\right)\\
X_{\delta p}\left(t\right) &=\left(X_T\left(f\right)\cdot
\text{\SCHA}_a\left(f\right)\right)\ast\text{\SCHA}_p\left(f\right)\\
X_{\delta p}\left(t\right) &=\sum_{m=-\infty}^\infty\sum_{k=-\infty}^\infty
X_T\left(mf_a-kf_p\right)\cdot f_a\cdot \delta\left(f-mf_a\right)\\
f_a &=\frac{1}{t_p}\\
f_p &=\frac{1}{t_a}
\end{align*}

\subsection*{Korrespodenz}
\index{Transformationen!Korrespondenz}
\begin{align*}
x\left(t\right) =\hat{X}\mathrm{rect}_{T}\left(t\right) \quad &\laplace \quad X\left(j\omega\right)
=\hat{X}T\cdot\mathrm{si}\left({\omega\cdot\frac{T}{2}}\right)\\
x\left(t\right) =\hat{X}\Lambda_{T}\left(t\right) \quad &\laplace \quad X\left(j\omega\right)
=\hat{X}T\cdot\mathrm{si}^2\left({\omega\cdot\frac{T}{2}}\right)\\
x\left(t\right) =\hat{X}\sin\left(2\pi f_0 t\right) \quad &\laplace \quad X\left(f\right)
=\frac{\hat{jX}}{2}\left(\delta\left(f+f_0\right)-\delta\left(f-f_0\right)\right)\\
x\left(t\right) =\hat{X}\cos\left(2\pi f_0 t\right) \quad &\laplace \quad X\left(f\right)
=\frac{\hat{X}}{2}\left(\delta\left(f+f_0\right)+\delta\left(f-f_0\right)\right)
\end{align*}

\section{Spektrum}
\index{Spektrum}
\subsection*{Betragsspektrum}
\index{Betragsspektrum}
\begin{align*}
\left|X\left(f\right)\right|=\sqrt{\left(\text{Re}\left\{X\left(f\right)\right\}\right)^2+\left(\text{Im}\left\{X\left(f\right)\right\}\right)^2}
\end{align*}

\subsection*{Betragsquadratspektrum}
\index{Betragsquadratspektrum}
\begin{align*}
\left|X\left(f\right)\right|^2 =\left(\text{Re}\left\{X\left(f\right)\right\}\right)^2+\left(\text{Im}\left\{X\left(f\right)\right\}\right)^2
\end{align*}

\subsection*{Theorem von Parseval}
\index{Theorem von Parseval}
\begin{align*}
E =m_{i2}=\int_{-\infty}^\infty x^2\left(t\right)\diff t=\int_{-\infty}^\infty \left|X\left(f\right)\right|^2\diff f
\end{align*}

\section{Korrelation}
\index{Korrelation}
\subsection*{Kreuzkorrelationsfunktion}
\index{Kreuzkorrelationsfunktion}
\begin{align*}
E_{x_1x_2}\left(\tau\right) &=\int_{-\infty}^\infty x_2\left(t+\tau\right)\cdot
x_1\left(t\right)\diff t=\int_{-\infty}^\infty x_1\left(t-\tau\right)\cdot x_2\left(t\right)\diff t\\
E_{x_1x_2}\left(l\right) &=\sum_{k=-\infty}^\infty x_2\left(k+l\right)\cdot x_1\left(k\right)\diff t
\end{align*}

\subsection*{Normierte Kreuzkorrelationsfunktion}
\index{Kreuzkorrelationsfunktion!-normierte}
\begin{align*}
\mathring{x} &=\sqrt[n]{x_1\cdot x_2\cdot \dotsc \cdot x_n}\\
\mathring{E}_{x_1x_2} &=\sqrt{\int_{-\infty}^\infty x_1^2\left(t\right)\diff t\cdot
\int_{-\infty}^\infty x_2^2\left(t\right)\diff t}\\
r_{x_1x_2}\left(\tau\right)
&=\frac{E_{x_1x_2}\left(\tau\right)}{\mathring{E}_{x_1x_2}}=\frac{\int_{-\infty}^\infty x_2\left(t+\tau\right)\cdot x_1\left(t\right)\diff t}{\sqrt{\int_{-\infty}^\infty x_1^2\left(t\right)\diff t\cdot \int_{-\infty}^\infty x_2^2\left(t\right)\diff t}}\\
\left|r_{x_1x_2}\left(\tau\right)\right| &\leq 0
\end{align*}