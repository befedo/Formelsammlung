\section{Zahlensysteme}
\index{Binäre Rechenoperationen!Zahlensysteme}
\begin{multicols}{2}
	\subsection{Dualsystem}
	\begin{align*}
	&\text{Basis}:2\quad z\in\left(1;0\right)\\
	&\text{Format}:2^{n-1} \dots 2^{0},2^{-1} \dots 2^{-m}\\
	&\text{Zahlenwert}:\sum_{l=-m}^{n-1}z_l\cdot2^l
	\end{align*}
	
	\subsection{Ternärsystem}
	\begin{align*}
	\text{Basis}:2\quad z\in\left(1;0;-1\right)
	\end{align*}
	
	\subsection{Oktalsystem}
	\begin{align*}
	\text{Basis}:8\quad z\in\left(0;1;2;3;4;5;6;7\right)
	\end{align*}
	
	\subsection{Hexadezimalsystem}
	\begin{align*}
	\text{Basis}:16\quad z\in\left(0;1;2;3\dots d;e;f\right)
	\end{align*}
	
	\subsection{Dezimalsystem}
	\begin{align*}
	\text{Basis}:10\quad z\in\left(0;1;2;3\dots 8;9\right)
	\end{align*}
\end{multicols}

\newpage
\index{Binäre Rechenoperationen!Stellenberechnung}
\index{Binäre Rechenoperationen!Wertebereich}
\index{Binäre Rechenoperationen!Quantisierungsfehler}
\index{Binäre Rechenoperationen!Umwandlung Negativer Dualzahlen}
\subsection{Stellenberechnung}
\(
k: \text{Zahlenwert}\\
l: \text{Anzahl Nachkommastellen im Dezimalsystem}\\
n: \text{Anzahl Vorkommanstellen im Dualsystem}\\
m: \text{Anzahl Nachkommastellen im Dualsystem}
\)
\begin{align*}
&n\geq\text{ceil}\left(\frac{\lg\left(k+1\right)}{\lg\left(2\right)}\right)\\
&m\geq\text{ceil}\left(\frac{l}{\lg\left( 2\right)}\right)
\end{align*}

\subsection{Wertebereich und Quantisierungsfehler(Dualsystem)}
\(
n: \text{Anzahl Vorkommanstellen im Dualsystem}\\
m: \text{Anzahl Nachkommastellen im Dualsystem}
\)
\begin{align*}
&\text{Wertebereich}=0\dots 2^n-2^{-m}\\
&\text{Quantisierungsfehler}=\pm\frac{1}{2}\text{LSB}=\pm 2^{-m-1}\\
&\text{Quantisierungsstufen}=2^{-m}
\end{align*}

\subsection{Umwandlung Negativer Dualzahlen}
\begin{align*}
n&=\text{ceil}\left(1+\frac{\lg\left(|\text{Wert}|\right)}{\lg\left(2\right)}\right)\\
\Delta&=2^{n-1}-\left|\text{Wert}\right|
\end{align*}
Umwandlung von $\Delta$ bis MSB Stelle -1, danach setzten der richtigen
MSB-Stelle.

\section{Addition}
\index{Binäre Rechenoperationen!Addition!CLA}
\index{Binäre Rechenoperationen!Addition!Halbaddierer}
\index{Binäre Rechenoperationen!Addition!Volladdierer}
\subsection{Zwei Operanden}
\begin{multicols}{2}
\raggedcolumns
	\subsubsection{Carry-Look-Ahead}
	\(
	s_i: \text{Summe der Stelle i}\\
	p_{i+1}: \text{Propagate Übertrag an der Stelle i}\\
	g_{i+1}: \text{Generate Übertrag kompensation}
	\)
	\begin{align*}
	s_i&=a_i\nsim b_i \nsim c_i\\
	p_{i+1}&=a_i\nsim b_i\\
	g_{i+1}&=a_i\cdot b_i\\
	c_{i+1}&=p_{i+1}\cdot c_i +g_{i+1}
	\end{align*}

	\subsubsection{Halbaddierer}
	\(
	s_i: \text{Summe der Stelle i}\\
	c_{i+1}: \text{Übertrag der Stelle i+1}
	\)
	\begin{align*}
	s_i&=a_i\nsim b_i\\
	c_{i+1}&=a_i \cdot b_i
	\end{align*}
		
	\subsubsection{Volladdierer}
	\(
	s_i: \text{Summe der Stelle i}\\
	c_{i+1}: \text{Übertrag der Stelle i+1}\\
	c_{i}: \text{Übertrag der Stelle i}
	\)
	\begin{align*}
	s_i&=a_i\nsim b_i \nsim c_i\\
	c_{i+1}&=a_i \cdot b_i+\left(a_i\nsim b_i\right)\cdot c_i
	\end{align*}	
\end{multicols}

\subsection{Mehrere Operanden}
\index{Binäre Rechenoperationen!Addition!mehrere Operanden}
\begin{description}
  	\item [Ripple-Carry:] 	Jede Stufe addiert jeweils einen Operanden hinzu.
	\item [Baumstruktur:] 	Die einzelnen Operanden werden Baumförmig addiert.
	\item [Carry-Save:]		Nur der letzte Addierer ist Sequentiel aufgebaut. Daher
							wird der Übertrag des Vorgängers beim nächsten aufaddiert.
\end{description}

\subsection{Überlauf}
\index{Binäre Rechenoperationen!Addition!Überlauf}
\begin{multicols}{2}
	\subsubsection{Vermeidung}
	\(
	r: \text{Zusätzliche Summenstellen}\\
	p: \text{Operandenanzahl}
	\)
	\begin{align*}
	r=\text{ceil}\left(\frac{\lg\left(p\right)}{\lg\left(2\right)}\right)
	\end{align*}
	
	\subsubsection{Positive Operanden}
	\(
	s: \text{Summe}\\
	n: \text{Vorkommanstellen im Dualsystem}
	\)
	\begin{align*}
	s&=\left(a+b\right)\text{mod} 2^n
	\end{align*}
\end{multicols}
	
	\subsubsection{Negative Operanden}
	\(
	s: \text{Entstehender Summenwert}\\
	n: \text{Vorkommanstellen im Dualsystem}\\
	p: \text{Operandenanzahl}
	\)

	\[s=\left(a+b+2^n\text{ceil}\left(\frac{p-1}{2}\right)+2^{n-1}\right)\text{mod}
	2^n-2^{n-1}\]

\newpage
\subsection{Überlaufserkennung}
\index{Binäre Rechenoperationen!Addition!Überlauf!Erkennung}
	\textbf{Vergleich der MSB Stellen}
	\begin{align*}
	&\begin{aligned}
	&a_{MSB}\neq b_{MSB}\hfill&&\Rightarrow \text{Kein Überlauf möglich}\\
	&a_{MSB}=b_{MSB}=s_{MSB}&&\Rightarrow \text{Kein Überlauf möglich}\\
	&a_{MSB}=b_{MSB}=0 \quad\&\quad s_{MSB}=1&&\Rightarrow \text{Positiver Überlauf}\\
	&a_{MSB}=b_{MSB}=1 \quad\&\quad s_{MSB}=0&&\Rightarrow \text{Negativer Überlauf}
	\end{aligned}\\
	&\mathit{MIN}=a_{MSB}\\
	&\mathit{OVF}=\overline{a}_{MSB} \cdot s_{MSB}\left(\overline{b_{MSB}\nsim
	s_{MSB}}\right)+a_{MSB} \cdot \overline{s}_{MSB}\left(b_{MSB}\nsim
	s_{MSB}\right)
	\end{align*}
		
	\textbf{Vergleich des Carry}
	\begin{align*}
	&c_{I;MSB}= c_{O;MSB}&&\Rightarrow \text{Kein Überlauf möglich}\\
	&c_{I;MSB}=1\quad\&\quad c_{O;MSB}=0&&\Rightarrow \text{Positiver Überlauf}\\
	&c_{I;MSB}=0\quad\&\quad c_{O;MSB}=1&&\Rightarrow \text{Negativer Überlauf}\\
	&\mathit{MIN}=c_{O;MSB}\\
	&\mathit{OVF}=c_{I;MSB}\nsim c_{O;MSB}
	\end{align*}
		
	\textbf{Erweiterung der MSB Stelle}
	\begin{align*}
	&s_{MSB}= s_{MSB+1}&&\Rightarrow \text{Kein Überlauf möglich}\\
	&s_{MSB}=1\quad\&\quad s_{MSB+1}=0&&\Rightarrow \text{Positiver Überlauf}\\
	&s_{MSB}=0\quad\&\quad s_{MSB+1}=1&&\Rightarrow \text{Negativer Überlauf}\\
	&\mathit{MIN}=s_{MSB}\\
	&\mathit{OVF}=s_{MSB}\nsim s_{MSB+1}
	\end{align*}


\subsection{Sättigung}
\index{Binäre Rechenoperationen!Addition!Sättigungsverhalten}
\(
s_{MSB}: \text{Behandlung der höchsten Stelle}\\
s_{LSB}: \text{Behandlung der restlichen Stellen}
\)
\begin{align*}
s'_{MSB}&=s_{MSB} \cdot \overline{\mathit{OVF}}+\mathit{OVF} \cdot
\mathit{MIN}\\
s'_{LSB}&=s_{LSB} \cdot \overline{\mathit{OVF}}+\mathit{OVF} \cdot
\overline{{MIN}}
\end{align*}

\subsection{Umwandlung Ternärcode}

\begin{tabular}{lll|ll}
$h_i$& $b_{i+1}$&$b_i$&$c_i$&$h_{i+1}$\\
\hline
0&0&0&0&0\\
0&0&1&1&0\\
0&1&0&0&0\\
0&1&1&-1&1\\
1&0&0&1&0\\
1&0&1&0&1\\
1&1&0&-1&1\\
1&1&1&0&1
\end{tabular}